% \subsubsection{Data Model}
% \label{sec:entity-schema-model}
Table~\ref{table:entity-model} provides a formal specification of the schema elements.
Resources and course elements, except sessions, are referred to as \textit{entities}.
Entities are typed,
the set of sessions is cast as distinct type, 
and each type is modeled as a finite set.
% % :
% % the set of courses ${\COURSE}$, 
% % course parts ${\PART}$, 
% % classes ${\CLASS}$, 
% % teachers ${\TEACHER}$,
% % rooms ${\ROOM}$,
% % students ${\STUDENT}$, 
% % student groups ${\GROUP}$, 
% % and the domain of courses %${\COURSES}$ 
% % ${\COURSES}=\myset{\COURSE}$. 
% % ${\TYPE}$
% % (
% $
% {\TYPE}
% =
% \myset{
% {\COURSES}, 
% {\COURSE},
% {\PART},
% {\CLASS},
% {\ROOM},
% {\TEACHER},
% {\STUDENT},
% {\GROUP}
% }
% $
% % )
% denotes the set of entity types, 
% % ${\ENTITY}$
% ${\ENTITY}=\setunion{X}{\TYPE}{X}$
%  the set of entities,
% and ${\SESSION}$ the set of sessions.
The course element hierarchy
defines 1-to-many \textit{composition relations} over the pair of types
$(X,Y)$ corresponding to parent and child types
in the course element hierarchy. 
Each relation is modeled by a function ${\maptype{X}{Y}}:X\rightarrow2^{Y}$
mapping each object $i$ of type $X$
to the set ${\map{X}{Y}{i}}$ of its constitutive objects of type $Y$.
% For notational convenience, the table also represents the inverse ${\maptype{Y}{X}}:Y\rightarrow2^{X}$ of each function ${\maptype{X}{Y}}$.
% ($j\in\map{X}{Y}{i}\leftrightarrow i\in\map{Y}{X}{j}$)
For instance, 
${\maptype{\PART}{\CLASS}}$ 
models the classes of each part.
% and ${\maptype{\CLASS}{\PART}}$ 
% the (singleton) part of each class.
Each \textit{compatibility relation}
defining the allowed or assigned resources of a course element object
for a given resource type and course element type
defines a many-to-many relation
which we model the same way.
%by 2 inverse functions.
For instance, 
${\maptype{\PART}{\ROOM}}$
models the allowed rooms per part
and ${\maptype{\CLASS}{\GROUP}}$
the set of groups assigned to classes.

For notational convenience, 
the table also defines 
the maps resulting from 
the symmetric and transitive closure of the binary relation %(digraph)
merging the composition and compatibility maps.
This includes
the maps
computed over the course tree.
% \footnote{
% $\forall X,Y,Z \in {\TYPE}\cup\myset{\SESSION}:
% X\preceq^{*} Y\preceq^{*} Z 
% \Rightarrow 
% (\forall i \in X:
% \map{X}{Z}{i}=\setpartition{j}{\map{X}{Y}{i}}{\map{Y}{Z}{j}}
% $
% }
For instance, 
%function 
${\maptype{\CLASS}{\PART}}$ 
models the (singleton) part of each class,
and ${\maptype{\COURSE}{\SESSION}}$ 
the sessions of a course.
%resulting from the recursive set-union of the sessions of its parts' classes.
This also includes the 
inverse compatibility constraints
and those inherited along the course tree.
For instance, ${\maptype{\SESSION}{\ROOM}}$ models the rooms allowed for a session
which results from the composition of ${\maptype{\SESSION}{\CLASS}}$, ${\maptype{\CLASS}{\PART}}$ and ${\maptype{\PART}{\ROOM}}$.
% - 
% $
% j\in{\map{\SESSION}{\ROOM}{i}}
% \leftrightarrow 
% j\in{\map{\PART}{\ROOM}{k}}
% \wedge
% i\in{\map{\PART}{\SESSION}{k}}
% $
% )
Lastly, the table defines the constants (e.g., number of weeks),
scalar properties
(e.g., room capacity),
and remaining relations and sets (e.g. required resources, labels).
% and remaining relations (required resources) 
% and sets (labels, exclusive/inclusive sessions).

% $
% {\prec}
% =
% \myset{
% ({\COURSES},{\COURSE}),
% ({\COURSE},{\PART}),
% ({\PART},{\CLASS}),
% ({\CLASS},{\SESSION}),
% ({\TEACHER},{\PART}),
% ({\ROOM},{\PART}),
% ({\STUDENT},{\GROUP}),
% ({\GROUP},{\COURSE})
% }
% $
% denotes the relation over 
% ${\TYPE}\cup\myset{\SESSION}$ 
% that models the course hierarchy
% and the distribution of resource types over course components.

% ${\prec^{*}}$
% %${\preceq^{*}}$
% denotes the transitive %and reflexive 
% closure of
% ${\prec}$ 
% over
% ${\TYPE}\cup\myset{\SESSION}$
% and
% ${\maptype{X}{Y}}:X\rightarrow2^{Y}$
% denotes the function mapping each element of $X$ to its set of compatible elements in $Y$
% for each pair %$(X,Y)$ such that 
% %$X{\preceq^{*}}Y$.
% $X{\prec^{*}}Y$.
% For instance, 
% ${\maptype{\ROOM}{\PART}}$ 
% represents the distribution of rooms over course parts, 
% ${\maptype{\PART}{\CLASS}}$ 
% the decomposition of course parts into classes,
% ${\maptype{\CLASS}{\SESSION}}$ 
% the decomposition of classes into sessions,
% and ${\maptype{\ROOM}{\SESSION}}$ 
% the inferred distribution of rooms over sessions.
% The functions corresponding to the pairs of $\prec$
% are directly encoded in the entity model
% and the remaining functions are defined inductively using recursive aggregation. 

% We shall denote by ${\map{X}{Y}{i}}$ the image of entity $i$ of type $X$ over $2^Y$ %, i.e., the set of elements of type $Y$ compatible with $i$. 
% and by ${\maptype{Y}{X}}$ the inverse of ${\maptype{X}{Y}}$.

% Equation (\ref{model:hierarchy}) below models the hierarchical decomposition of course elements\footnote{$\disjunion$ denotes the disjoint union operation, i.e. set union over pairwise disjoint sets.},
% Equation (\ref{model:transitivity}) is the closure rule over 
% %$\preceq^{*}$. 
% $\prec^{*}$,
% %Note that each map $\maptype{\SESSION}{X}$ is the inverse of map $\maptype{X}{\SESSION}$.
% and Equation (\ref{model:inverse}) models inverse maps.

% \begin{align}
% %
% \forall (X,Y) \in 
% \myset{
% ({\COURSES},{\COURSE}),
% ({\COURSE},{\PART}),
% ({\PART},{\CLASS}),
% ({\CLASS},{\SESSION})
% }:
% Y=
% \setpartition{i}{X}{\map{X}{Y}{i}} 
% \label{model:hierarchy}
% \\
% %
% \forall X,Y,Z \in {\TYPE}\cup\myset{\SESSION}:
% X\preceq^{*} Y\preceq^{*} Z 
% \Rightarrow 
% (\forall i \in X:
% \map{X}{Z}{i}=\setpartition{j}{\map{X}{Y}{i}}{\map{Y}{Z}{j}}
% \label{model:transitivity})
% \\
% %
% \forall X,Y \in {\TYPE}:
% X\preceq^{*} Y 
% \Rightarrow 
% (\forall i \in X, j \in Y:
% j \in \map{X}{Y}{i} \Leftrightarrow i \in \map{Y}{X}{j}
% )
% \label{model:inverse}
% %\\
% %%
% %\forall X \in {\TYPE}\cup\myset{\SESSION},
% %i \in X:
% %\domarg{X}{X}{i} = \myset{i} \label{model:selfmap}
% %%
% \end{align}


%\footnote{
% The following rules apply. $\SLOT=\myset{i.\WEEKDAY.\DAILYSLOT+j.\DAILYSLOT+k\ |\ 0\leq i<\WEEK,0\leq j<\WEEKDAY,1\leq k\leq\DAILYSLOT}$.
% For each class $k$ in part $p$,
% $\myset{\sessionrank{s}\ |\ s\in\map{\CLASS}{\SESSION}{k}}=\myset{1,\ldots,\mycard{\map{\CLASS}{\SESSION}{k}}}$, 
% %$\mycard{\classparents{k}}\leq1$ 
% and $\classparents{k}\not\subset\map{\PART}{\CLASS}{p}$.
% For each pair of sessions $s,s'$, 
% $(s,s')\in\sessionranked$ iff $\map{\SESSION}{\CLASS}{s}=\map{\SESSION}{\CLASS}{s'}$ and $\sessionrank{s'}=\sessionrank{s}+1$.
% For each course part $p$,
% %$p\in\multiroomparts$ iff $\multiroompart{p}$; 
% %and 
% $\partteachermultiplicity{p}.\mycard{\map{\PART}{\SESSION}{p}}=\sum\limits_{l\in\map{\PART}{\TEACHER}{p}}{\partteacherservice{l,p}}$.}
%
%We shall denote by
%${\RANK}$
%the range of session ranks,
%${\maptype{\RANK}{\SESSION}}:\RANK\rightarrow2{^\SESSION}$
%the rank-based partitioning of sessions,
%and
%${\LABEL}$
%the set of labels 
%(${\LABEL}\subseteq2^{{\ENTITY}}$)
%completed 
%with the whole set of entities %to mock label optionality
%($\ENTITY\in{\LABEL}$)
%and singleton entities %to support identity-based selection
%($\myset{\myset{e}\ |\ e\in{\ENTITY}}\subseteq{\LABEL}$).
%As discussed in section~\ref{sec:rules},
%labels are optional filters used in rules to select entities
%hence the formal inclusion of $\ENTITY$ in ${\LABEL}$ to mock label optionality.
%Likewise, entity identifiers are used as an alternative to labels
%hence the inclusion of singleton entities in ${\LABEL}$.


\begin{table}[!t]
\begin{center}
\rowcolors{1}{gray!25}{white}
%\begin{tabular}{|r>{\columncolor{gray!25}}l|}
%\begin{tabular}{|ll|}
\begin{tabular}{|l|l|}
%\begin{tabular}{ll}


\hline
$\week$                                 & number of weeks
dividing the scheduling horizon
%in $\SLOT$\\
\\%in $\SLOT$\\
$\weekday$                               & number of weekdays
making a week \\
$\dailyslot$                              &  number of daily slots
making a 24-hour day% in $\SLOT$
%DAILY SLOT M pour minute
\\
$\WEEK =\myset{1,\ldots\week}$
&  range of weeks 
\\
$\WEEKDAY =\myset{1,\ldots\weekday}$
&  range of weekdays 
\\
$\DAILYSLOT =\myset{1,\ldots\dailyslot}$
&  range of daily slots 
\\
$\SLOT=\myset{1,\ldots\week\times\weekday\times\dailyslot}$
%$\SLOT$                                  
&  range of \timepoints{} (schedule horizon)
%defining the scheduling horizon
\\\hline
$\COURSE$%\subseteq\ENTITY$
                                            &  courses
\\
$\PART$%\subseteq\ENTITY$
                                            &  course parts
\\
$\CLASS$%\subseteq\ENTITY$
                                            &  part classes
\\
$\ROOM$%\subseteq\ENTITY$
                                            &  rooms
\\
$\TEACHER$%\subseteq\ENTITY$
                                            &  teachers
\\
$\STUDENT$%\subseteq\ENTITY$
                                            &  students
\\
$\GROUP$%\subseteq{\ENTITY}$
                                            &  groups of students

% \\
% $\ENTITY$                                   & the entities
\\\hline
$\COURSES=\myset{\COURSE}$%\subseteq\ENTITY$
                                            &  course domain
\\
$\TYPE=\myset{\COURSES,\COURSE,\PART,\CLASS,\ROOM,\TEACHER,\STUDENT,\GROUP}
$                                           &  types of entities
\\
${\ENTITY}=\setunion{X}{\TYPE}{X}$          &  set of entities
\\
$\map{X}{Y}{i}\subseteq{Y}, X \in \myset{\COURSES,\COURSE,\PART,\CLASS}$
                                            &  set of entities of type $Y$ tied to entity $i$ of type $X$
\\
$\map{X}{Y}{i}\subseteq{Y}, X \in \myset{\ROOM,\TEACHER,\STUDENT,\GROUP}$
                                            &  set of entities of type $Y$ associated with entity $i$ of type $X$
\\
% \\\hline
${\LABEL}\subseteq2^{{\ENTITY}}$            &  labels
\\\hline
% $\SESSION$                                  & the class sessions
% \\
$ \SESSIONEX$% \subseteq \SESSION $
                                            &  exclusive class sessions 
\\
$ \SESSIONINC$% \subseteq \SESSION $
                                            &  inclusive class sessions 
\\
$\SESSION=\SESSIONEX\disjunion\SESSIONINC$  &  class sessions
\\
$\partallowedslots{s}\subseteq{\SLOT}$      &  start times allowed for session $s$
\\
$\map{\SESSION}{X}{s}\subseteq{X}$
                                            &  set of entities of type $X$ tied to session $s$
\\
$\map{X}{\SESSION}{i}\subseteq{\SESSION}$
                                            &  set of sessions tied to entity $i$ of type $X$
% \\
% %UP
% $\umap{X}{\SESSION}{i}\subseteq{\SESSION},\; X\in\{\ROOM, \TEACHER\}$
%                                             & the set of sessions allowed for, or tied to, entity $i$ of type $X$
\\
%DOWN
$\dmap{X}{\SESSION}{i}\subseteq{\SESSION}$
                                            &  set of sessions required by resource entity $i$ of type $X$
\\
%$\disjunctiverooms\subseteq{\ROOM}$         & the set of disjunctive rooms
%\\
\hline
$\multiroompartmin{p}\in{\NATURAL}$          &  min number of rooms usable by each session of part $p$
%whether course part $p$ is multi-room or not
\\
$\multiroompartmax{p}\in{\NATURAL}$          &  max number of rooms usable by each session of part $p$
% whether course part $p$ is multi-room or not
\\
%$\partteachermultiplicity{p}\in{\NATURAL}$  & the number of lecturers usable by each session of part $p$
%\\
$\partteachermultiplicitymin{p}\in{\NATURAL}$  &  min number of lecturers usable by each session of part $p$
\\
$\partteachermultiplicitymax{p}\in{\NATURAL}$  &  max number of lecturers usable by each session of part $p$
\\
$\headcount{g}\in{\NATURAL}$               &  headcount of group $g$
\\
$\classcapacity{k}\in{\NATURAL}$            &  headcount of class $k$
\\
%$\multiteacherparts\subseteq{\PART}$           & the multi-teacher parts
%\\
%$\partteacherservice{l,p}\in{\NATURAL}$     & the number of sessions %required by lecturer $l$ in part $p$
%\\
$\roomcapacity{r}\in{\NATURAL}$             &  capacity of room $r$
\\
$\sessionduration{s}\in{\SLOT}$             &  duration of session $s$
\\
$\sessionrank{s}\in{\NATURAL^*}$            &  rank of session $s$ in its class
\\
$\partteacherservice{t,p}\in{\NATURAL}$     &  number of sessions required by teacher $t$ in part $p$
%\\
%$\sessionranked\subseteq{\SESSION\times\SESSION}$ & the pairs of sessions with consecutive ranks in a class
%\\
%$\sessionranked\subseteq{\SESSION\times\SESSION}$ & the pairs of sessions with consecutive ranks in a class
%$\classparents{k}\subseteq{\CLASS}$          & the parent classes of class $k$ if any
%\\
%\\
%$\virtualroom{r}\in{\BOOLEAN}$              & whether room $r$ is virtual or not
%\\
%$\virtualrooms\subseteq{\ROOM}$             & the virtual rooms
\\\hline
%$\PARTREQUIREDROOM \subseteq \CLASS$          & the classes requiring at least one room
% \\
%$\multiroomparts\subseteq{\PART}$           & the multi-room parts
%\\
%$\mandatoryrooms{p}\subseteq{\ROOM}$        & the mandatory rooms of part $p$
%\\
%
%
%$\multiroompart{p}\in{\BOOLEAN}$            & whether course part $p$ is multi-room or not
%\\
%\\\hline
\end{tabular}
\caption{Core data model.}
\label{table:entity-model}
\end{center}
\end{table}
%TODO \todo{for future work: the type graph/relation and its closure should be re-included in the text, and defined in the table and the maps `d` in the table explicitly restricted to pairs of the closed relation}