\section{Rules Syntax and Constraint Predicate Catalog}
\label{appendix:constraintcatalog}


Table~\ref{tab:rule-language}
provides the syntax of the rules language: e-maps, constraints, queries and rules.
\begin{center}
\begin{table}[!ht]
    \centering
    \begin{tabular}{|ll|}
    %\begin{tabular}{ll}
    \hline
       \grayrow $(e_i,S_i)$
        & e-map mapping entity $e_i$ to the set of sessions $S_i$
        \\
    \hline
        ${\EMAP}=\ENTITY\times{2^{\SESSION}}$  
        & the domain of e-maps
        \\
    \hline
      \grayrow  $c((e_1,S_1),..,(e_n,S_n),p_1,..,p_m)$
        &  constraint of predicate $c$, arity $n$, parameters $p_1,..,p_m$
        \\
    \grayrow    & and e-map arguments $(e_1,S_1),..,(e_n,S_n)\in{\EMAP}^{n}$
        \\
    \hline
    $\RANK$ = $1..\max\limits_{s\in\SESSION} \sessionrank{s}$ & the range of session ranks 
    \\
    %& i.e. ${\maptype{\RANK}{\SESSION}}:\RANK\rightarrow2{^\SESSION}$ is the rank-based partitioning of sessions s.t.
    %\\
    %&$s\in\map{\RANK}{\SESSION}{o}$ iff $\sessionrank{s}=o$
    %\\
    \hline
    ${\LABEL}^{*}={\LABEL}\cup\myset{\ENTITY}\cup\myset{\myset{e}\ |\ e\in{\ENTITY}}$
        &
the set of labels 

    \\\hline
   \grayrow $
{\SELECTOR}=\cup_{n\geq1}({\TYPE}\times{\LABEL}^*\times{2^{\RANK}})^{n}
$& the language of queries
    \\\hline
     \grayrow   $c\langle{\SELECTOR},p_1,\ldots,p_m\rangle$
        & rule of predicate $c$, query $\SELECTOR$ and parameters $p_1,\ldots p_m$
        \\
     \grayrow   & - $c$ predicate of arity $n$ and number of parameters $m$
        \\
     \grayrow   & - $\SELECTOR$ query sized to extract $n$ sets of e-maps
        \\
     \grayrow   & - $p_1,\ldots p_m$ values for the parameters of $c$
        \\
    \hline
    \end{tabular}
    \caption{Predicates, constraints, queries and rules.}
    \label{tab:rule-language}
\end{table}
\end{center}


% Let
% ${\RANK}$
% denote the range of session ranks,
% ${\maptype{\RANK}{\SESSION}}:\RANK\rightarrow2{^\SESSION}$
% the rank-based partitioning of sessions
% ($s\in\map{\RANK}{\SESSION}{o}$ iff $\sessionrank{s}=o$),
% and
% completed 
% with the whole set of entities to mock label optionality
% %($\ENTITY\in{\LABEL}^*$)
% and singleton entities to support identity-based selection,
% %($\myset{\myset{e}\ |\ e\in{\ENTITY}}\subseteq{\LABEL}^*$),
% the language of selectors
% is the set 
% $
% {\SELECTOR}=\cup_{n\geq1}({\TYPE}\times{\LABEL}^*\times{2^{\RANK}})^{n}
% $.
% %where each 
% Each selector 
% $
% d=((T_1,L_1,O_1),\ldots,(T_k,L_k,O_k))
% $
% %($k\geq1$)
% decomposes into a generator
% $
% (T_1,L_1,O_1)
% $
% and a possibly empty list of filters
% $
% ((T_2,L_2,O_2),\ldots,(T_k,L_k,O_k))
% $.
% $d$ matches any e-map
% whose entity has type $T_1$ and label $L_1$ and whose image includes any compatible session satisfying mask $O_1$ and any of the filters.
% The set of e-maps
% $\denote{d}$
% matched by $d$
% is defined by
% \begin{multline*}
% \denote{d}=
% \bigcup\limits_{e\in{T_1\cap L_1}}
% \Big \{(e,S')
% \ |\ 
% S'=
% \map{T_1}{\SESSION}{e}
% \;\bigcap
% %\\
% \bigcup\limits_{i=2\ldots k}
% {\Big(
% \mape{T_i}{\SESSION}{L_i}
% \bigcap
% \mape{\RANK}{\SESSION}{O_1\cap O_i}
% \Big)
% }
% \\
% \wedge
% S'\neq\emptyset
% \Big \}
% \end{multline*}

% where
% $
% {\mape{X}{Y}{X'}}
% =
% %$
% %denotes
% %$
% \bigcup\limits_{i\in X'}{\map{X}{Y}{i}}
% $
% with
% $X'\subseteq X$
% .


Table~\ref{tab:predicate_catalog}
lists and describes the predicates of the catalog.

\begin{table}[H]
\resizebox{\textwidth}{!}{%
\centering

\begin{tabular}{|l|l|}
%\begin{tabular}{ll}
\hline
\textbf{Predicate}               & \textbf{Description}\\ \hline

\grayrow{\ADJACENTROOMS}            & Sessions must be adjacent in the given room(s)\\ \hline

\ALLOWEDGRIDS{}  & Sessions may only start in the given time grid(s)\\ \hline
\grayrow\ALLOWEDROOMS{} & Sessions may only be hosted in the given room(s) \\ \hline
\ALLOWEDSLOTS{} & Sessions may only run in the given time slots\\ \hline
\grayrow\ALLOWEDTEACHERS{} & Sessions may only be taught by the given teacher(s)  \\ \hline
\ASSIGNROOMS{} & Sessions are hosted in the given room(s) \\ \hline
\grayrow\ASSIGNSLOT{} & Sessions start at the given time \\ \hline
\ASSIGNTEACHERS{} & Sessions are taught by the given teacher(s) \\ \hline
\grayrow\COMPACTNESS{} & The sessions makespan is bounded  \\ \hline
\DIFFERENTDAILYSLOT{} & Sessions start on different daily slots \\ \hline
\grayrow\DIFFERENTDAY{}  & Sessions start on different days \\ \hline
\DIFFERENTROOMS{} & Sessions are hosted in different rooms \\ \hline
\grayrow\DIFFERENTSLOT{} & Sessions start at different times \\ \hline
\DIFFERENTTEACHERS{} & Sessions are taught by different teachers \\ \hline
\grayrow\DIFFERENTWEEK{} & Sessions start on different week \\ \hline
\DIFFERENTWEEKDAY{}& Sessions start on different weekday \\ \hline
\grayrow\DIFFERENTWEEKLYSLOT{} & Sessions start on different weekly time points \\ \hline
\FORBIDDENROOMS{}         & Sessions cannot be hosted in the given room(s)\\ \hline
\grayrow\FORBIDDENSLOTS{} & Sessions cannot run in the given time slots \\ \hline
\FORBIDDENTEACHERS{}         & Sessions cannot be taught by the given teacher(s)\\ \hline

\GAP{}  &\cellcolor[gray]{.9}Gaps between sessions are bounded\\ \hline

\NOOVERLAP{}                & Sessions in the given set cannot overlap\\ \hline
\grayrow\PAIRWISENOOVERLAP{}     & Sessions cannot overlap if in different sets\\ \hline
\PERIODIC{}                & Sessions are periodic \\ \hline

\grayrow\REQUIREDROOMS{}   & Sessions must be hosted in the given room(s) \\ \hline

\REQUIREDTEACHERS{}     & Sessions must be taught by the given teacher(s) \\ \hline

\grayrow{\SAMEDAILYSLOT}   & Sessions start on the same daily slot\\ \hline
{\SAMEDAY}                  & Sessions start on the same day\\ \hline

\grayrow{\SAMEROOMS}                & Sessions are hosted in the same room(s)\\ \hline
{\SAMESLOT}                 & Sessions start at the same time\\ \hline
\grayrow{\SAMETEACHERS}             & Sessions are taught by the same teacher(s)\\ \hline

{\SAMEWEEKDAY}              & Sessions start on the same weekday\\ \hline
\grayrow{\SAMEWEEKLYSLOT}           & Sessions start on the same weekly time point\\ \hline
{\SAMEWEEK}                 & Sessions start on the same week\\ \hline

\grayrow\SEQUENCED{}                & Sessions run sequentially\\\hline

\WORKLOAD{sessions}       & The number of sessions per time frame is bounded\\ \hline

\WORKLOAD{times}      & The total duration of sessions per time frame is bounded\\ \hline



%{\FORBIDDENPERIOD}         & Sessions cannot start in the given time period\\ \hline
%{\ATMOSTDAILY}             & The number of sessions scheduled in the daily period is upper-bounded\\ \hline
%{\ATMOSTWEEKLY}            & The number of sessions scheduled in the weekly period is upper-bounded\\ \hline
%implicit\_sequenced\_sessions & 1 & \multicolumn{4}{|c|}{no} & All sessions in classes are sequenced\\ \hline
%{\SEQUENCED}                & $\geq2$   & no    & Sessions are sequenced\\ \hline
%{\WEEKLY}                   & Sessions are weekly \\ \hline

% \hline
%{\TRAVEL}                  & Travel time is factored in if sessions hosted in the given rooms\\ \hline




%{\TEACHERDISTRIBUTION}       & Distributes lecturer workload over classes\\ \hline

\end{tabular}
}
\caption{Catalog of {\UTP} constraint predicates.}
\label{tab:predicate_catalog}
\end{table}

Table~\ref{tab:constraintformel} provides the semantics of each predicate
once the scope is restricted to a tuple of sets of sessions
(obtained after joining a solution and a constraint built with the predicate).
Given
a $n$-ary predicate $c$ accepting $m$ parameters ($m\geq 0$)
and given a $n$-uple %made of sets of sessions
$(S'_1,\ldots,S'_n)\in{S}^n$,
%=\ENTITY\times{2^{\SESSION}}$$ 
we provide the semantics for $c(S'_1,\ldots,S'_n,p_1,\ldots,p_m)$
which denotes the evaluation of the predicate on the tuple.
% \begin{equation*}
% $c(S'_1,\ldots,S'_n,p_1,\ldots,p_m)$
% \end{equation*}


% $c((e_1,S_1),\ldots,(e_n,S_n),p_1,\ldots,p_m)$ %\label{rule:constraint}
% % \end{align}
% % \end{equation*}
% where 
% $c$ is a predicate of arity $n$,
% $(e_1,S_1),\ldots,(e_n,S_n)$ are e-maps ($(e_i,S_i)\in{\EMAP}$ for $i=1\ldots n$) 
% and 
% $p_1,\ldots,p_m$ are values for the parameters of $c$ ($m\geq0$).
 %lists the predicates of the language and indicates which are variadic or parametric.  
% The first predicates \SAMEDAILYSLOT, , same slot enforce common restrictions on the start times of the targeted sessions (e.g.,
% sessions starting the same day). Additionally, any start time interval may
% be forbidden by passing its start and end points as parameters to predicate
% forbidden period. Predicates at most daily and at most weekly upper-bound the number of sessions scheduled daily or weekly within the given
% time interval. sequenced is a n-ary predicate (n >= 2) which constrains the
% latest session of the i-th e-map to end before the earliest session of i + 1-th e-map (i = 1..n - 1). Predicate weekly ensures sessions are scheduled
% weekly without presuming any particular sequencing. Predicate no overlap
% ensures sessions do not overlap in time and is typically used to model disjunctive resources. Predicate travel factors in any travel time incurred between
% consecutive sessions hosted in distant rooms. The travel time matrix is a parameter of the predicate. same rooms, same students and same lecturers
% require that sessions be assigned to the same set of rooms, students or lecturers. Predicate adjacent rooms require that sessions be hosted in adjacent
% rooms based on an adjacency graph passed as a parameter. Lastly, predicate
% lecturer distribution distributes the volumes of sessions represented by
% the different e-map arguments among different lecturers. Lecturers and session volumes are parameters of the predicate.


%\newpage
% \begin{table}[H]
  %  \centering
\newcounter{rowcntrformal}[table]
\renewcommand{\therowcntrformal}{(\arabic{rowcntrformal})}
\setcounter{rowcntrformal}{0}
\begin{longtable}{|lr|}
%\begin{longtable}{lr}

    \hline%======================
        \textbf{Predicate} & \textbf{Parameters} \\ 
        \multicolumn{2}{|l|}{\textbf{Semantics}} \\
    \hline%======================
   \grayrow\textbf{\ADJACENTROOMS}(\SESSION',K$_1$,$\dots$,K$_n$) & $ K_i \subseteq \ROOM, i : 1..n $ \\
  \grayrow\multicolumn{2}{|l|}{  $\forall s_1, s_2 \in \SESSION', \var{\SESSION}{\SLOT}{s_1} = \var{\SESSION}{\SLOT}{s_2} $
    $\forall i \in  \{1..n\} : C_i =\{ \var{\SESSION}{\ROOM}{s} \mid s \in \SESSION' , \var{\SESSION}{\ROOM}{s} \in K_i \} \land$}\\
     \grayrow\multicolumn{2}{|l|}{ 
   $ \forall r,r' \in C_i : \exists path(r,r') \land |\{C_i \mid C_i \ne \emptyset , i = 1..n \}| \leq \sigma $}\refstepcounter{rowcntrformal} \therowcntrformal\label{formal:adjacentrooms}\\
   %\multicolumn{2}{|l|}{
    %$ \forall s1,s2 \in \SESSION', s_1 +1 = s_2, \var{\SESSION}{\ROOM}{s_1} \subseteq CR, \var{\SESSION}{\ROOM}{s_2} \subseteq CR,$}\\
    %\multicolumn{2}{|l|}{$
    %(\pi \circ \var{\SESSION}{\ROOM}{s_1}+1 = \pi \circ \var{\SESSION}{\ROOM}{s_2}) \vee (\pi \circ \var{\SESSION}{\ROOM}{s_1} = \pi \circ \var{\SESSION}{\ROOM}{s_2}+1)$}


    
   % Il n'existe pas de chemin entre deux composantes connexes de la solution.
    %Le nombre de composantes connexes doit être inférieur ou égale à la valeur du paramètre.
    %Paramètres : graphe de connexité des salles + nombre max de composantes connexes
    \hline%======================
   \textbf{\ALLOWEDGRIDS}(\SESSION',G)   & $G \in (W'\times D' \times M')^n, n \in \mathbb{N}^*, $ \\
    &$ S'\subseteq\SESSION, W'\subseteq \WEEK, D'\subseteq\WEEKDAY, M'\subseteq\DAILYSLOT$ \\
    \multicolumn{2}{|l|}{
    $\forall s\in S', \exists k \in \{1..n\}: \var{\SESSION}{\WEEK}{s}\in W'_k \wedge   \var{\SESSION}{\WEEKDAY}{s}\in D'_k \wedge  \var{\SESSION}{\DAILYSLOT}{s}\in M'_k$ }\refstepcounter{rowcntrformal} \therowcntrformal\label{formal:allowedgrids}    \\[-0.75em]
    \multicolumn{2}{|c|}{\tikz{\draw[dashed, line width=0.4pt, yshift=-0.5\arrayrulewidth] (0,0) -- (\linewidth,0);}} \\[-0.58ex]
    %\hdashline%======================
    \grayrow\textbf{\ALLOWEDROOMS}($\SESSION'$,$\ROOM'$) & $ S' \subseteq \SESSION, \ROOM' \subseteq \ROOM $ 
    \\
    \grayrow\multicolumn{2}{|l|}{
    $\forall s \in \SESSION'$, $\var{\SESSION}{\ROOM}{s} \subseteq \ROOM'$}
    \refstepcounter{rowcntrformal} \therowcntrformal\label{formal:allowedrooms}
        \\[-0.75em]
    \multicolumn{2}{|c|}{\tikz{\draw[dashed, line width=0.4pt, yshift=-0.5\arrayrulewidth] (0,0) -- (\linewidth,0);}} \\[-0.58ex]
    %\hdashline%======================
    \textbf{\ALLOWEDSLOTS}($\SESSION'$,$\SLOT'$)&  $\SLOT' \subseteq \SLOT,  $
    \\
    \multicolumn{2}{|l|}{
      $\forall s \in S' [\var{\SESSION}{\SLOT}{s},\var{\SESSION}{\SLOT}{s}+\sessionduration{s}] \subseteq \SLOT'$}\refstepcounter{rowcntrformal} \therowcntrformal\label{formal:allowedslots}\\[-0.75em]
    \multicolumn{2}{|c|}{\tikz{\draw[dashed, line width=0.4pt, yshift=-0.5\arrayrulewidth] (0,0) -- (\linewidth,0);}} \\[-0.58ex]
    %\hdashline%======================
     \grayrow\textbf{\ALLOWEDTEACHERS}((e,$\TEACHER'$),$\ROOM'$) &  $ S' \subseteq \SESSION, \TEACHER' \subseteq \TEACHER  $ \\
     \grayrow\multicolumn{2}{|l|}{
     $\forall s \in \SESSION'$, $\var{\SESSION}{\TEACHER}{s} \subseteq \TEACHER'$}\refstepcounter{rowcntrformal} \therowcntrformal\label{formal:allowedteachers}\\
    \hline%======================
    \textbf{\ASSIGNROOMS}($\SESSION'$,$\ROOM'$) & $ \ROOM' \subseteq \ROOM	$
    \\
    \multicolumn{2}{|l|}{
    $  \forall s \in \SESSION'$, $\var{\SESSION}{\ROOM}{s} = \ROOM' $}\refstepcounter{rowcntrformal} \therowcntrformal\label{formal:assignrooms}    
    \\[-0.75em]
    \multicolumn{2}{|c|}{\tikz{\draw[dashed, line width=0.4pt, yshift=-0.5\arrayrulewidth] (0,0) -- (\linewidth,0);}} \\[-0.58ex]
    %\hdashline%==========samegroup============
    \grayrow\textbf{\ASSIGNSLOT}($\SESSION'$,$\SLOT'$) & $ h \in \SLOT	$ 
    \\
    \grayrow\multicolumn{2}{|l|}{
    $  \forall s \in \SESSION'$, $\var{\SESSION}{\SLOT}{s} = h $}
    \refstepcounter{rowcntrformal} \therowcntrformal\label{formal:assignslot}
        \\[-0.75em]
    \multicolumn{2}{|c|}{\tikz{\draw[dashed, line width=0.4pt, yshift=-0.5\arrayrulewidth] (0,0) -- (\linewidth,0);}} \\[-0.58ex]
    %\hdashline%==========samegroup============
    \textbf{\ASSIGNTEACHERS} $\SESSION'$,$\TEACHER'$) 
    &
    $ \TEACHER' \subseteq \TEACHER$ 
    \\
    \multicolumn{2}{|l|}{
    $  \forall s \in \SESSION'$, $\var{\SESSION}{\TEACHER}{s} = \TEACHER' $
    }\refstepcounter{rowcntrformal} \therowcntrformal\label{formal:assignteachers}\\
    \hline%======================
   \grayrow\textbf{\COMPACTNESS}($\SESSION'$,$\sigma$)
    &
    $\sigma\in 0..|\SLOT'|$
    \\
    \grayrow\multicolumn{2}{|l|}{
     $ \forall d \in \WEEKDAY : \exists \SESSION'' = \{s \in \SESSION' : \var{\SESSION}{\WEEKDAY}{s}\} \land$ 
        }\\
     \grayrow\multicolumn{2}{|l|}{
    $ ((\max\limits_{s \in \SESSION''}(\var{\SESSION}{\SLOT}{s}+\sessionduration{s})-\min\limits_{s \in \SESSION'}(\var{\SESSION}{\SLOT}{s}))-\sum_{s \in \SESSION''}\sessionduration{s} )) / (|S''|-1)\leq \sigma$
    }
    \refstepcounter{rowcntrformal} \therowcntrformal\label{formal:compactness}\\
    \hline%======================
    \textbf{\DIFFERENTDAILYSLOT}($\SESSION'$) 
    &
    \\
    \multicolumn{2}{|l|}{
    $\forall s_1, s_2 \in \SESSION'$, $\var{\SESSION}{\DAILYSLOT}{s_1}  \ne \var{\SESSION}{\DAILYSLOT}{s_2}$
    }\refstepcounter{rowcntrformal} \therowcntrformal\label{formal:differentdailyslot}
    \\[-0.75em]
    \multicolumn{2}{|c|}{\tikz{\draw[dashed, line width=0.4pt, yshift=-0.5\arrayrulewidth] (0,0) -- (\linewidth,0);}} \\[-0.58ex]
    %\hdashline%==========samegroup============
    \grayrow\textbf{\DIFFERENTDAY}($\SESSION'$) & 
    \\
    \grayrow\multicolumn{2}{|l|}{
    $\forall s_1, s_2 \in \SESSION',\var{\SESSION}{\WEEKDAY}{s_1}  \ne \var{\SESSION}{\WEEKDAY}{s_2} \vee \var{\SESSION}{\WEEK}{s_1}  \ne \var{\SESSION}{\WEEK}{s_2} $
    }\refstepcounter{rowcntrformal} \therowcntrformal\label{formal:differentday}
        \\[-0.75em]
    \multicolumn{2}{|c|}{\tikz{\draw[dashed, line width=0.4pt, yshift=-0.5\arrayrulewidth] (0,0) -- (\linewidth,0);}} \\[-0.58ex]
    %\hdashline%==========samegroup============
    \textbf{\DIFFERENTROOMS}($\SESSION'$) 
    &
    \\
    \multicolumn{2}{|l|}{
    $\forall s_1, s_2 \in \SESSION'$, $\var{\SESSION}{\ROOM}{s_1}  \cap \var{\SESSION}{\ROOM}{s_2} = \emptyset	$}\refstepcounter{rowcntrformal} \therowcntrformal\label{formal:differentrooms}
        \\[-0.75em]
    \multicolumn{2}{|c|}{\tikz{\draw[dashed, line width=0.4pt, yshift=-0.5\arrayrulewidth] (0,0) -- (\linewidth,0);}} \\[-0.58ex]
    %\hdashline%==========samegroup============ 
    \grayrow\textbf{\DIFFERENTSLOT}($\SESSION'$)
    &
    \\
    \grayrow\multicolumn{2}{|l|}{
    $\forall s_1, s_2 \in \SESSION'$, $\var{\SESSION}{\SLOT}{s_1}   \ne \var{\SESSION}{\SLOT}{s_2}$}\refstepcounter{rowcntrformal} \therowcntrformal\label{formal:differentslot}
    \\[-0.75em]
    \multicolumn{2}{|c|}{\tikz{\draw[dashed, line width=0.4pt, yshift=-0.5\arrayrulewidth] (0,0) -- (\linewidth,0);}} \\[-0.58ex]
    %\hdashline%==========samegroup============
    \textbf{\DIFFERENTTEACHERS}($\SESSION'$) 
    &
    \\
    \multicolumn{2}{|l|}{
    $\forall s_1, s_2 \in \SESSION'$, $\var{\SESSION}{\TEACHER}{s_1}  \cap \var{\SESSION}{\TEACHER}{s_2} = \emptyset	$}\refstepcounter{rowcntrformal} \therowcntrformal\label{formal:differentteachers}%\\
    \\[-0.75em]
    \multicolumn{2}{|c|}{\tikz{\draw[dashed, line width=0.4pt, yshift=-0.5\arrayrulewidth] (0,0) -- (\linewidth,0);}} \\[-0.58ex]
    %\hdashline%==========samegroup============
    \grayrow\textbf{\DIFFERENTWEEK }($\SESSION'$)
    &
    \\
    \grayrow\multicolumn{2}{|l|}{
    $\forall s_1, s_2 \in \SESSION'$, $\var{\SESSION}{\WEEK}{s_1}  \ne \var{\SESSION}{\WEEK}{s_2}$}\refstepcounter{rowcntrformal} \therowcntrformal\label{formal:differentweek}%\\
        \\[-0.75em]
    \multicolumn{2}{|c|}{\tikz{\draw[dashed, line width=0.4pt, yshift=-0.5\arrayrulewidth] (0,0) -- (\linewidth,0);}} \\[-0.58ex]
    %\hdashline%==========samegroup============
    \textbf{\DIFFERENTWEEKDAY}($\SESSION'$)
    &
    \\
    \multicolumn{2}{|l|}{
    $\forall s_1, s_2 \in \SESSION',\var{\SESSION}{\WEEKDAY}{s_1}  \ne \var{\SESSION}{\WEEKDAY}{s_2} $}\refstepcounter{rowcntrformal} \therowcntrformal\label{formal:differentweekday}\\%$\forall s_1, s_2 \in \SESSION \times \SESSION$, $(1+((\var{\SESSION}{\SLOT}{s_1}-1) / \WEEKDAY   )) \ne (1+((\var{\SESSION}{\SLOT}{s_2}-1) / \WEEKDAY   ))$
    \\[-0.75em]
    \multicolumn{2}{|c|}{\tikz{\draw[dashed, line width=0.4pt, yshift=-0.5\arrayrulewidth] (0,0) -- (\linewidth,0);}} \\[-0.58ex]
    %\hdashline%==========samegroup============ 
    \grayrow\textbf{\DIFFERENTWEEKLYSLOT}($\SESSION'$)
    &
    \\
    \grayrow\multicolumn{2}{|l|}{
    $\forall s_1, s_2 \in \SESSION'$, $\var{\SESSION}{\DAILYSLOT}{s_1}  \ne \var{\SESSION}{\DAILYSLOT}{s_2} \vee \var{\SESSION}{\WEEKDAY}{s_1}  \ne \var{\SESSION}{\WEEKDAY}{s_2} $}\refstepcounter{rowcntrformal} \therowcntrformal\label{formal:differentweeklyslot}\\% $\forall s_1, s_2 \in \SESSION \times \SESSION$, $(1+((\var{\SESSION}{\SLOT}{s_1}-1) / \WEEKDAY   )) \ne (1+((\var{\SESSION}{\SLOT}{s_2}-1) / \WEEKDAY   ))$
      
    \hline%======================
    \textbf{\FORBIDDENROOMS}($\SESSION'$,$\ROOM'$)
    &  
    $\ROOM' \subseteq \ROOM $ 
    \\
    \multicolumn{2}{|l|}{
    $\forall s \in \SESSION'$, $\var{\SESSION}{\ROOM}{s} \subseteq \ROOM\setminus\ROOM'$}\refstepcounter{rowcntrformal} \therowcntrformal\label{formal:forbiddenrooms}
    \\[-0.75em]
    \multicolumn{2}{|c|}{\tikz{\draw[dashed, line width=0.4pt, yshift=-0.5\arrayrulewidth] (0,0) -- (\linewidth,0);}} \\[-0.58ex]
    
    %\hdashline%==========samegroup============
    \grayrow\textbf{\FORBIDDENSLOTS}($\SESSION'$,$\SLOT'$) 
    &
    $\SLOT' \subseteq \SLOT  $
    \\
    \grayrow\multicolumn{2}{|l|}{
    $\forall s \in S'  [\var{\SESSION}{\SLOT}{s},\var{\SESSION}{\SLOT}{s}+\sessionduration{s}[ \cap \SLOT' = \emptyset$}\refstepcounter{rowcntrformal} \therowcntrformal\label{formal:forbiddenslots}
    
    \\[-0.75em]
    \multicolumn{2}{|c|}{\tikz{\draw[dashed, line width=0.4pt, yshift=-0.5\arrayrulewidth] (0,0) -- (\linewidth,0);}} \\[-0.58ex]
    %\hdashline%==========samegroup============
    &\\
    \textbf{\FORBIDDENTEACHERS}($\SESSION'$,$\TEACHER'$)
    &
    $\TEACHER' \subseteq \TEACHER  $ 
    \\
    \multicolumn{2}{|l|}{
    $\forall s \in \SESSION'$, $\var{\SESSION}{\TEACHER}{s} \subseteq \TEACHER\setminus\TEACHER'$}\refstepcounter{rowcntrformal} \therowcntrformal\label{formal:forbiddenteachers}\\
      
     %\GAP & $N_{min},N_{max} \in \SLOT\cup\{0\} , $ $\forall s_1,s_2 \in \SESSION', $ $ (N_{min} \geq \var{\SESSION}{\SLOT}{s_2} - (\var{\SESSION}{\SLOT}{s_1} +\sessionduration{s_1})) \land  ((\var{\SESSION}{\SLOT}{s_2} +\sessionduration{s_2})-\var{\SESSION}{\SLOT}{1} \leq N_{max})$
    %%%%%%%%%%%%%%%%%%%%%%%%%%%%%%%%%%%%%%%%%%%%%%%%%%

     
    \hline%============== NEW GAP  ==============
    \grayrow\textbf{\GAPARG{min\_slot}}($\SESSION'$,$\sigma_{min}$)
    & 
    $\sigma_{min}\in 0..|\SLOT| , $ 
 \\%\\\hdashline
 \grayrow\multicolumn{2}{|l|}{
    %$ \NOOVERLAPARG{\SESSION'} \land$
    $\exists!\pi:\SESSION'\to[[\SESSION']] : \var{\SESSION}{\SLOT}{\piinv{i}} < \var{\SESSION}{\SLOT}{\piinv{j}}   $ \raisebox{0.2ex}{$\scriptscriptstyle(1\leq i < j \leq |\SESSION'|)$}}
    \\
    \grayrow\multicolumn{2}{|l|}{
    $\forall i \in 1..|\SESSION'|-1,$
    $\var{\SESSION}{\SLOT}{\piinv{i+1}} - (\var{\SESSION}{\SLOT}{\piinv{i}} + \sessionduration{\piinv{i}})\geq \sigma_{min}$}\refstepcounter{rowcntrformal} \therowcntrformal\label{formal:gapminslot}

    \\[-0.75em]
    \multicolumn{2}{|c|}{\tikz{\draw[dashed, line width=0.4pt, yshift=-0.5\arrayrulewidth] (0,0) -- (\linewidth,0);}} \\[-0.58ex]
%\hdashline%==========samegroup=============
    \textbf{\GAPARG{min\_day}}($\SESSION'$,$\sigma_{min}$)
    & 
    $\sigma_{min}\in 0..|\WEEKDAY|*|\WEEK| , $ 
    \\%\\\hdashline
    \multicolumn{2}{|l|}{
    %$ \NOOVERLAPARG{\SESSION'} \land$
    $\exists!\pi:\SESSION'\to[[\SESSION']] : \var{\SESSION}{\SLOT}{\piinv{i}} < \var{\SESSION}{\SLOT}{\piinv{j}}   $ \raisebox{0.2ex}{$\scriptscriptstyle(1\leq i < j \leq |\SESSION'|)$}
    }
    \\
    \multicolumn{2}{|l|}{
    $\land \forall i \in 1..|\SESSION'|-1,\var{\SESSION}{\WEEKDAY}{\piinv{i+1}} - (\var{\SESSION}{\WEEKDAY}{\piinv{i}}) \geq \sigma_{min}$}\refstepcounter{rowcntrformal} \therowcntrformal\label{formal:gapminday}

    \\[-0.75em]
    \multicolumn{2}{|c|}{\tikz{\draw[dashed, line width=0.4pt, yshift=-0.5\arrayrulewidth] (0,0) -- (\linewidth,0);}} \\[-0.58ex]
    %\hdashline%==========samegroup============
   \grayrow \textbf{\GAPARG{min\_week}}($\SESSION'$,$\sigma_{min}$)
    & 
    $\sigma_{min}\in 0..|\WEEK| , $ 
    \\%\\\hdashline
    \grayrow\multicolumn{2}{|l|}{
    %$ \NOOVERLAPARG{\SESSION'} \land$
    $\exists!\pi:\SESSION'\to[[\SESSION']] : \var{\SESSION}{\SLOT}{\piinv{i}} < \var{\SESSION}{\SLOT}{\piinv{j}}   $ \raisebox{0.2ex}{$\scriptscriptstyle(1\leq i < j \leq |\SESSION'|)$}
    }
    \\
    \grayrow\multicolumn{2}{|l|}{
    $\land \forall i \in 1..|\SESSION'|-1,\var{\SESSION}{\WEEK}{\piinv{i+1}} - (\var{\SESSION}{\WEEK}{\piinv{i}})\geq \sigma_{min}$}\refstepcounter{rowcntrformal} \therowcntrformal\label{formal:gapminweek}\\
\hline%======================
    \textbf{\GAPARG{max\_slot}}($\SESSION'$,$\sigma_{max}$)
    & 
    $\sigma_{max}\in 0..|\SLOT| , $ 
    \\%\\\hdashline
    \multicolumn{2}{|l|}{
    $s_1 = \min\limits_{s \in \SESSION'}(\var{\SESSION}{\SLOT}{s}), s_2 = \max\limits_{s \in \SESSION'}(\var{\SESSION}{\SLOT}{s}+\sessionduration{s}),$
    $\var{\SESSION}{\SLOT}{s_2} - (\var{\SESSION}{\SLOT}{s_1}+\sessionduration{s}) \leq \sigma_{max}$}\refstepcounter{rowcntrformal} \therowcntrformal\label{formal:gapmaxslot}
    \\[-0.75em]
    \multicolumn{2}{|c|}{\tikz{\draw[dashed, line width=0.4pt, yshift=-0.5\arrayrulewidth] (0,0) -- (\linewidth,0);}} \\[-0.58ex]
%\hdashline%==========samegroup============
   \grayrow \textbf{\GAPARG{max\_day}}($\SESSION'$,$\sigma_{max}$)  
    & 
    $\sigma_{max}\in 0..|\WEEKDAY|*|\WEEK| , $ 
    \\%\\\hdashline
    \grayrow\multicolumn{2}{|l|}{
    $s_1 = \min\limits_{s \in \SESSION'}(\var{\SESSION}{\SLOT}{s}), s_2 = \max\limits_{s \in \SESSION'}(\var{\SESSION}{\SLOT}{s}),$
    $\var{\SESSION}{\WEEKDAY}{s_2} - \var{\SESSION}{\WEEKDAY}{s_1} \leq \sigma_{max}$}\refstepcounter{rowcntrformal} \therowcntrformal\label{formal:gapmaxday}
    \\[-0.75em]
    \multicolumn{2}{|c|}{\tikz{\draw[dashed, line width=0.4pt, yshift=-0.5\arrayrulewidth] (0,0) -- (\linewidth,0);}} \\[-0.58ex]
%\hdashline%==========samegroup============

    \textbf{\GAPARG{max\_week}}($\SESSION'$,$\sigma_{max}$)
    & 
    $\sigma_{max}\in 0..|\WEEK| , $ 
    \\%\\\hdashline
    \multicolumn{2}{|l|}{$s_1 = \min\limits_{s \in \SESSION'}(\var{\SESSION}{\SLOT}{s}), s_2 = \max\limits_{s \in \SESSION'}(\var{\SESSION}{\SLOT}{s}),$
    $\var{\SESSION}{\WEEK}{s_2} - \var{\SESSION}{\WEEK}{s_1} \leq \sigma_{max}$}\refstepcounter{rowcntrformal} \therowcntrformal\label{formal:gapmaxweek}\\
\hline%======================

   \grayrow \textbf{\GAPARG{last\_first\_slot}}($\SESSION_1'$$\dots$,$\SESSION_n'$,$\sigma_{min}$,$\sigma_{max}$)  
    & 
    $\sigma_{min},\sigma_{max}\in 0..|\SLOT| , $ 
    \\%\\\hdashline
    %\multicolumn{2}{|l|}{$\SEQUENCED(\SESSION_1,...,\SESSION_n) \land $}\\
   \grayrow \multicolumn{2}{|l|}{$
    \forall i \in 1..n-1 ,\; s_i = \max\limits_{s \in \SESSION_i }(\var{\SESSION}{\SLOT}{s}+\sessionduration{s}),\; 
    s_{i+1} = \min\limits_{s \in \SESSION_{i+1}} (\var{\SESSION}{\SLOT}{s})$
    }
    \\
   \grayrow \multicolumn{2}{|l|}{
    $ \sigma_{min} \leq s_{i+1} - s_i \leq \sigma_{max} $ } \refstepcounter{rowcntrformal} \therowcntrformal\label{formal:gaplastfirstslot}
    \\[-0.75em]
    \multicolumn{2}{|c|}{\tikz{\draw[dashed, line width=0.4pt, yshift=-0.5\arrayrulewidth] (0,0) -- (\linewidth,0);}} \\[-0.58ex]
    %\hdashline%==========samegroup============
    \textbf{\GAPARG{last\_first\_day}}($\SESSION_1'$$\dots$,$\SESSION_n'$,$\sigma_{min}$,$\sigma_{max}$)   
    & 
    $\sigma_{min},\sigma_{max}\in 0..|\SLOT| , $ 
    \\%\\\hdashline
    %\multicolumn{2}{|l|}{$\SEQUENCED(\SESSION_1,...,\SESSION_n) \land $}\\
    \multicolumn{2}{|l|}{$
    \forall i \in 1..n-1 ,\; s_i = \max\limits_{s \in \SESSION_i }(\var{\SESSION}{\WEEKDAY}{s}),\;
    s_{i+1} = \min\limits_{s \in \SESSION_{i+1}} (\var{\SESSION}{\WEEKDAY}{s})$
    }
    \\
    \multicolumn{2}{|l|}{
    $ \sigma_{min} \leq s_{i+1} - s_i \leq \sigma_{max} $ } \refstepcounter{rowcntrformal} \therowcntrformal\label{formal:gaplastfirstday}

    \\[-0.75em]
    \multicolumn{2}{|c|}{\tikz{\draw[dashed, line width=0.4pt, yshift=-0.5\arrayrulewidth] (0,0) -- (\linewidth,0);}} \\[-0.58ex]
    %\hdashline%==========samegroup============
   \grayrow \textbf{\GAPARG{last\_first\_week}}($\SESSION_1'$$\dots$,$\SESSION_n'$,$\sigma_{min}$,$\sigma_{max}$)  
    & 
    $\sigma_{min},\sigma_{max}\in 0..|\SLOT| , $ 
    \\%\\\hdashline
   \grayrow \multicolumn{2}{|l|}{$
    \forall i \in 1..n-1 ,\; s_i = \max\limits_{s \in \SESSION_i }(\var{\SESSION}{\WEEK}{s}),\;
    s_{i+1} = \min\limits_{s \in \SESSION_{i+1}} (\var{\SESSION}{\WEEK}{s})$
    }
    \\
    \grayrow\multicolumn{2}{|l|}{
    $ \sigma_{min} \leq s_{i+1} - s_i \leq \sigma_{max} $ } \refstepcounter{rowcntrformal} \therowcntrformal\label{formal:gaplastfirstweek}\\
\hline%======================
    \textbf{\GAPARG{first\_last\_slot}}($\SESSION_1'$$\dots$,$\SESSION_n'$,$\sigma_{min}$,$\sigma_{max}$)  
    & 
    $\sigma_{min},\sigma_{max}\in 0..|\SLOT| , $ 
    \\%\\\hdashline
    %\multicolumn{2}{|l|}{$\SEQUENCED(\SESSION_1,...,\SESSION_n) \land $}\\
    \multicolumn{2}{|l|}{$
    \forall i \in 1..n-1 ,\; s_i = \min\limits_{s \in \SESSION_i }(\var{\SESSION}{\SLOT}{s}+\sessionduration{s}),\; 
    s_{i+1} = \max\limits_{s \in \SESSION_{i+1}} (\var{\SESSION}{\SLOT}{s})$
    }
    \\
    \multicolumn{2}{|l|}{
    $ \sigma_{min} \leq s_{i+1} - s_i \leq \sigma_{max} $ } \refstepcounter{rowcntrformal} \therowcntrformal\label{formal:gapfirstlastslot}

    \\[-0.75em]
    \multicolumn{2}{|c|}{\tikz{\draw[dashed, line width=0.4pt, yshift=-0.5\arrayrulewidth] (0,0) -- (\linewidth,0);}} \\[-0.58ex]
   %\hdashline%==========samegroup============
   \grayrow\textbf{\GAPARG{first\_last\_day}}($\SESSION_1'$$\dots$,$\SESSION_n'$,$\sigma_{min}$,$\sigma_{max}$)  
    & 
    $\sigma_{min},\sigma_{max}\in 0..|\SLOT| , $ 
    \\%\\\hdashline
    %\multicolumn{2}{|l|}{$\SEQUENCED(\SESSION_1,...,\SESSION_n) \land $}\\
   \grayrow \multicolumn{2}{|l|}{$
    \forall i \in 1..n-1 ,\; s_i = \min\limits_{s \in \SESSION_i }(\var{\SESSION}{\WEEKDAY}{s}),\;
    s_{i+1} = \max\limits_{s \in \SESSION_{i+1}} (\var{\SESSION}{\WEEKDAY}{s})$
    }
    \\
    \grayrow\multicolumn{2}{|l|}{
    $ \sigma_{min} \leq s_{i+1} - s_i \leq \sigma_{max} $ }\refstepcounter{rowcntrformal} \therowcntrformal\label{formal:gapfirstlastday} 

    \\[-0.75em]
    \multicolumn{2}{|c|}{\tikz{\draw[dashed, line width=0.4pt, yshift=-0.5\arrayrulewidth] (0,0) -- (\linewidth,0);}} \\[-0.58ex]
    %\hdashline%==========samegroup============
    \textbf{\GAPARG{first\_last\_week}}($\SESSION_1'$$\dots$,$\SESSION_n'$,$\sigma_{min}$,$\sigma_{max}$)  
    & 
    $\sigma_{min},\sigma_{max}\in 0..|\SLOT| , $ 
    \\%\\\hdashline
    \multicolumn{2}{|l|}{$
    \forall i \in 1..n-1 ,\; s_i = \min\limits_{s \in \SESSION_i }(\var{\SESSION}{\WEEK}{s}),\;
    s_{i+1} = \max\limits_{s \in \SESSION_{i+1}} (\var{\SESSION}{\WEEK}{s})$
    }
    \\
    \multicolumn{2}{|l|}{
    $ \sigma_{min} \leq s_{i+1} - s_i \leq \sigma_{max} $ } \refstepcounter{rowcntrformal} \therowcntrformal\label{formal:gapfirstlastweek}\\
    %\GAP LF naire 
    %&
    %$N_{min},N_{max} \in \SLOT\cup\{0\} , $
    %\\%\\\hdashline
   % \multicolumn{2}{|l|}{$ \forall i,j \in \mathbb{N}, j = i+1 \forall s_1 \in \SESSION_j , s_2 \in \SESSION_i , $ $ s.t.  \var{\SESSION}{\SLOT}{s_1} \leq  \var{\SESSION}{\SLOT}{s_2}, \var{\SESSION}{\SLOT}{s_2} -\var{\SESSION}{\SLOT}{s_1} +\sessionduration{s_1} \in [N_{min},N_{max}]  \wedge  \var{\SESSION}{\SLOT}{s_1} \ne  \var{\SESSION}{\SLOT}{s_2} $
    %$Si = argmin_{s \in Si} xs , Si+1  = argmax xs + len xs$}\\
    \hline%======================
   \grayrow \textbf{\NOOVERLAP}($\SESSION'$)  
    & \\%\\\hdashline
    \multicolumn{2}{|l|}{\cellcolor[gray]{0.9}
    $\bigwedge\limits_{\substack{s_1,s_2\in \SESSION'\\s_1\neq s_2}}(\var{\SESSION}{\SLOT}{s_1} + \sessionduration{s_1} \leq \var{\SESSION}{\SLOT}{s_2})\vee(\var{\SESSION}{\SLOT}{s_2} + \sessionduration{s_2} \leq \var{\SESSION}{\SLOT}{s_1})$}
    \refstepcounter{rowcntrformal} \therowcntrformal\label{formal:nooverlap}\\

    \hline%======================
    \textbf{\PAIRWISENOOVERLAP}($\SESSION_1'$,$\SESSION_2'$)    
    & \\%\\\hdashline
     \multicolumn{2}{|l|}{
    $\bigwedge\limits_{\substack{s_1\in \SESSION'_1, s_2\in \SESSION'_2\\s_1\neq s_2}}
    (\var{\SESSION}{\SLOT}{s_1} + \sessionduration{s_1}\leq\var{\SESSION}{\SLOT}{s_2})
    \vee
    (\var{\SESSION}{\SLOT}{s_2} + \sessionduration{s_2}\leq\var{\SESSION}{\SLOT}{s_1})$}\refstepcounter{rowcntrformal} \therowcntrformal\label{formal:pairwisenooverlap}\\
    
    \hline%======================
    \grayrow\textbf{\PERIODIC}($\SESSION'$,$n$)   
    &
    $n \in \mathbb{N}$ \\%\\\hdashline

    \grayrow\multicolumn{2}{|l|}{
    $ \exists\pi : \SESSION' \rightarrow 1..|\SESSION'|,  \forall i \in 1..|\SESSION'|-1,\var{\SESSION}{\SLOT}{{\pi^{-1}(i)}} +n = \var{\SESSION}{\SLOT}{{\pi^{-1}(i+1)}}
    %s-2 s_2 \in \SESSION' s.t. s_1 \ne s_2, $ $ |\var{\SESSION}{\SLOT}{s_1}-\var{\SESSION}{\SLOT}{s_2}|=n 
    $}\refstepcounter{rowcntrformal} \therowcntrformal\label{formal:periodic}\\
      
    \hline%======================
    \textbf{\REQUIREDROOMS}($\SESSION'$,$\ROOM'$)  
    &
    $ \ROOM' \subseteq \ROOM $\\%\\\hdashline
     \multicolumn{2}{|l|}{
    $\forall s \in \SESSION' $, $\ROOM' \subseteq \var{\SESSION}{\ROOM}{s}	$}\refstepcounter{rowcntrformal} \therowcntrformal\label{formal:requiredrooms}

    \\[-0.75em]
    \multicolumn{2}{|c|}{\tikz{\draw[dashed, line width=0.4pt, yshift=-0.5\arrayrulewidth] (0,0) -- (\linewidth,0);}} \\[-0.58ex]
    %\hdashline%==========samegroup============
    \grayrow\textbf{\REQUIREDTEACHERS}($\SESSION'$,$\TEACHER'$,$\Delta$*) 
    &
   $  \TEACHER' \subseteq \TEACHER, \Delta = \{\forall t \in \TEACHER'\mid\delta_{1,t},\delta_{2,t}\}$, \\%\\\hdashline
   \grayrow& $\forall t \in \TEACHER',\forall i \in \{1,2\},\;\delta_{i,t}\in\mathbb{N}$\\
     \grayrow\multicolumn{2}{|l|}{
     $ \forall s \in \SESSION',\;( \var{\SESSION}{\TEACHER}{s} \subseteq \TEACHER'  \land \forall t \in\TEACHER',\;\delta_{1,t} \leq\sum\limits_{s \in \SESSION'} (t \in \var{\SESSION}{\TEACHER}{s})\leq \delta_{2,t} )$}\refstepcounter{rowcntrformal} \therowcntrformal\label{formal:requiredteachers}\\
    \hline%======================
    \textbf{\SAMEDAILYSLOT} ($\SESSION'$) 
    & \\%\\\hdashline
     \multicolumn{2}{|l|}{
    $\forall s_1, s_2 \in \SESSION'$,
    $\var{\SESSION}{\DAILYSLOT}{s_1}  = \var{\SESSION}{\DAILYSLOT}{s_2}$}\refstepcounter{rowcntrformal} \therowcntrformal\label{formal:samedailyslot}

    \\[-0.75em]
    \multicolumn{2}{|c|}{\tikz{\draw[dashed, line width=0.4pt, yshift=-0.5\arrayrulewidth] (0,0) -- (\linewidth,0);}} \\[-0.58ex]
    %\hdashline%==========samegroup============
    \grayrow\textbf{\SAMEDAY}($\SESSION'$) 
    & \\%\\\hdashline
     \grayrow\multicolumn{2}{|l|}{
    $\forall s_1, s_2 \in \SESSION',\var{\SESSION}{\WEEKDAY}{s_1}  = \var{\SESSION}{\WEEKDAY}{s_2} \wedge \var{\SESSION}{\WEEK}{s_1}  = \var{\SESSION}{\WEEK}{s_2} $}\refstepcounter{rowcntrformal} \therowcntrformal\label{formal:sameday}

    \\[-0.75em]
    \multicolumn{2}{|c|}{\tikz{\draw[dashed, line width=0.4pt, yshift=-0.5\arrayrulewidth] (0,0) -- (\linewidth,0);}} \\[-0.58ex]
      %
    %\hdashline%==========samegroup============
    \textbf{\SAMEROOMS}($\SESSION'$) 
    & \\%\\\hdashline
     \multicolumn{2}{|l|}{
    $\forall s_1, s_2 \in \SESSION'$, 
    $\var{\SESSION}{\ROOM}{s_1}  = \var{\SESSION}{\ROOM}{s_2}	$}\refstepcounter{rowcntrformal} \therowcntrformal\label{formal:samerooms}
      %
      \\[-0.75em]
    \multicolumn{2}{|c|}{\tikz{\draw[dashed, line width=0.4pt, yshift=-0.5\arrayrulewidth] (0,0) -- (\linewidth,0);}} \\[-0.58ex]
    %\hdashline%==========samegroup============
    \grayrow\textbf{\SAMESLOT}($\SESSION'$) 
    & \\%\\\hdashline
     \grayrow\multicolumn{2}{|l|}{
    $\forall s_1, s_2 \in \SESSION'$, 
    $\var{\SESSION}{\SLOT}{s_1}  = \var{\SESSION}{\SLOT}{s_2}	$}\refstepcounter{rowcntrformal} \therowcntrformal\label{formal:sameslot}
    
      %
      \\[-0.75em]
    \multicolumn{2}{|c|}{\tikz{\draw[dashed, line width=0.4pt, yshift=-0.5\arrayrulewidth] (0,0) -- (\linewidth,0);}} \\[-0.58ex]
    %\hdashline%==========samegroup============
    \textbf{\SAMETEACHERS}($\SESSION'$)   
    & \\%\\\hdashline
     \multicolumn{2}{|l|}{
    $\forall s_1, s_2 \in \SESSION'$, 
    $\var{\SESSION}{\TEACHER}{s_1}  = \var{\SESSION}{\TEACHER}{s_2}	$}\refstepcounter{rowcntrformal} \therowcntrformal\label{formal:sameteachers}
      %
      \\[-0.75em]
    \multicolumn{2}{|c|}{\tikz{\draw[dashed, line width=0.4pt, yshift=-0.5\arrayrulewidth] (0,0) -- (\linewidth,0);}} \\[-0.58ex]
    %\hdashline%==========samegroup============
    \grayrow\textbf{\SAMEWEEK}($\SESSION'$) 
    & \\%\\\hdashline
    \grayrow \multicolumn{2}{|l|}{
    $\forall s_1, s_2 \in \SESSION'$, 
    $\var{\SESSION}{\WEEK}{s_1} = \var{\SESSION}{\WEEK}{s_2}$}\refstepcounter{rowcntrformal} \therowcntrformal\label{formal:sameweek}
    \\[-0.75em]
    \multicolumn{2}{|c|}{\tikz{\draw[dashed, line width=0.4pt, yshift=-0.5\arrayrulewidth] (0,0) -- (\linewidth,0);}} \\[-0.58ex]
    %\hdashline%==========samegroup============
    \textbf{\SAMEWEEKDAY}($\SESSION'$) 
    & \\%\\\hdashline
     \multicolumn{2}{|l|}{
    $\forall s_1, s_2 \in \SESSION',$  $\var{\SESSION}{\WEEKDAY}{s_1} = \var{\SESSION}{\WEEKDAY}{s_2}$}\refstepcounter{rowcntrformal} \therowcntrformal\label{formal:sameweekday}

    \\[-0.75em]
    \multicolumn{2}{|c|}{\tikz{\draw[dashed, line width=0.4pt, yshift=-0.5\arrayrulewidth] (0,0) -- (\linewidth,0);}} \\[-0.58ex]
    
    %\hdashline%==========samegroup============
    \grayrow\textbf{\SAMEWEEKLYSLOT}($\SESSION'$) 
    & \\%\\\hdashline
     \grayrow\multicolumn{2}{|l|}{
    $\forall s_1, s_2 \in \SESSION'$, $\var{\SESSION}{\DAILYSLOT}{s_1}  = \var{\SESSION}{\DAILYSLOT}{s_2} \wedge \var{\SESSION}{\WEEKDAY}{s_1}  = \var{\SESSION}{\WEEKDAY}{s_2} $}\refstepcounter{rowcntrformal} \therowcntrformal\label{formal:sameweeklyslot}\\
    
    \hline%======================
    \textbf{\SEQUENCED}($\SESSION'_1$$\dots$,$\SESSION'_n$) 
    & \\%\\\hdashline
     \multicolumn{2}{|l|}{
    $%\forall i \in \mathbb{N}, 
    \forall s_i \in \SESSION'_i,\forall s_{i+1} \in \SESSION'_{i+1},\; \var{\SESSION}{\SLOT}{s_{i+1}} \geq \var{\SESSION}{\SLOT}{s_{i}} + \sessionduration{s_i}	$}\refstepcounter{rowcntrformal} \therowcntrformal\label{formal:sequenced}\\
    \hline%======================
    \grayrow\textbf{\text{\WORKLOAD{time}}}($\SESSION'$,$w_1,w_2$) 
    &
    $ w_1,w_2 \in \SLOT \cup \{0\} $ \\%\\\hdashline
    \grayrow\multicolumn{2}{|l|}{$\forall d \in \WEEKDAY,\;  w_1 \leq \sum\limits_{s\in\SESSION'}\sessionduration{s} \times (\var{\SESSION}{\WEEKDAY}{s} = d) \leq w_2 $}\refstepcounter{rowcntrformal} \therowcntrformal\label{formal:timeworkload}

    \\[-0.75em]
    \multicolumn{2}{|c|}{\tikz{\draw[dashed, line width=0.4pt, yshift=-0.5\arrayrulewidth] (0,0) -- (\linewidth,0);}} \\[-0.58ex]
    %\hdashline%==========samegroup============
    \textbf{\text{\WORKLOAD{session}}}($\SESSION'$,$w_1,w_2$)   & $ w_1,w_2 \in 0..|\SESSION|$\\%\\\hdashline
     \multicolumn{2}{|l|}{$\forall d \in \WEEKDAY,\;  w_1 \leq |\{ s\in\SESSION' : \var{\SESSION}{\WEEKDAY}{s} = d \}| \leq w_2 $}\refstepcounter{rowcntrformal} \therowcntrformal\label{formal:sessionworkload}\\
     \hline%======================
    \caption{Semantics of \UTP{}  constraint predicates}
    \label{tab:constraintformel}
    \end{longtable}
 %   
% \end{table}
