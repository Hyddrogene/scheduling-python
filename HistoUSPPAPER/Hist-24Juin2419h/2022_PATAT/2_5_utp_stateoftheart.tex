%------------------------------------------------------------
%------------------------------------------------------------
\subsection{Related work}
\label{sec:related-work}

%Examination timetabling {1999schaerfAIR}: marginal differences with CTP => mandatory attendance for students, different workload constraints for students, can have more than one exam in a room (cumulative room), TAP {2022caselliESWA}: post-allocation of "tutors" to "workshops". BACP {2001castroARXIV,2011chiarandiniJH}: close to maquettage Student sectioning: {2010mullerAOR} => initial vs batch vs online. {2019schindlAOR} CB-CTTP {2010mccollumINFORMS,2012bettinelliAOR,2015bettinelliTOP}: week schedule  with time periods (no session duration), no modeling of student enrolments, hard: same-curriculum/teacher (disjunctive teacher/students), room occupancy, no-overlap(lectures of a course), teacher unavailabilities, soft: room capacity, room stability, ... PE-CTTP {2007lewisITC,2010mccollumINFORMS}: week schedule (45 slots) with time periods (no session duration), student sectioning factored in, each course is a single event (no series of class sessions), students and rooms disjunctive,  New/subsuming models - ITC-2019 {2018mullerPATAT}: - UCTP (survey {2021chenIEEEA}) - {2022zaulirMJFAS,2014aizamNACO} => essentially "new" workload/pattern constraints but probably encodable in ITC. General survey: {2019oudeAOR}:


We highlight here the main differences between the {\UTP} language and the {\ITC} language ({\ITC} for short).

%the two approaches use the same temporal representation but {\ITC} leaves the possibility to configure the granularity of daily slots in each instance (${\DAILYSLOT\in\myset{1\ldots 1440}}$)  while it is set to the minute in {\UTP}. Nevertheless, any granularity may be used in {\UTP} by filtering the series of allowed slots in course parts and by re-scaling session duration and travel time data.
A first difference between the two frameworks lies in the representation of the possible times a class can meet.
In {\UTP}, a class is defined by a single sequence of sessions of equal duration and the problem is to schedule each session.
In {\ITC}, a class is given alternative fixed session schedules ({\texttt{times}} elements in the {\XML} schema) and the problem is to choose one of the schedules for the class.
A schedule is the repetition over a set of weeks of one or more sessions that have the same duration and start on specific days of the week at the same predefined time (daily slot).
The two representations are not reducible to one another.
For instance, alternative schedules using different session durations cannot be modeled in {\UTP}.
Conversely, class schedules where sessions do not necessarily start on the same daily slot cannot be modeled in {\ITC}.
Nevertheless, basic class schedules may be represented in either approach by stating {\ITC} constraints or {\UTP} rules on classes.
For instance, a class meeting every week on the same day and the same daily slot, both being subject to time restrictions, may be modeled using \texttt{\SAMEDAILYSLOT}, \texttt{\WEEKLY} and \texttt{\FORBIDDENPERIOD} constraints.
The implementation of a more comprehensive reduction method %for alternative schedules 
%(based, for instance, on a dedicated {\UTP} predicate) 
will be the subject of future work.

%Another difference lies in the representation of resources and the constraints governing their distribution and allocation.
%On the one hand, 
Second, 
{\ITC} represents alternative course configurations
by introducing an intermediate layer in the course hierarchy 
that sits between courses and parts.
The configurations of a course typically differ in their number of (sub)parts
and are mutually exclusive from a student sectioning standpoint, that is, 
a registered student must be assigned a single configuration and attend all of its parts.
This feature is not currently supported in {\UTP}.
As for resources, {\UTP} explicitly represents lecturers on par with rooms %and allows to specify their workload in each course part %(i.e., the number of sessions to teach)
whereas {\ITC} only models rooms.
{\UTP} also provides the flexibility to allocate different resources within a class 
(and specify lecturer workload in particular) %(i.e., the number of sessions to teach)
whereas the same room must be allocated in {\ITC}.
Additionally, {\UTP} supports multi-resource sessions whereas {\ITC} is restricted to single-room sessions.

Lastly, the two constraint languages present important differences.
While {\ITC} constraint predicates apply to classes,
{\UTP} predicates apply to any set(s) of sessions
and may be used in particular on individual sessions, hence granting finer-grained control.
Besides, {\UTP} rules and the selector language allows to constrain any class of resources or course elements in a concise way.

Lastly, the {\ITC} schema addresses the timetabling problem as a combinatorial optimization problem.
It includes a cost function weighting 4 criteria which respectively penalize the choice of sessions and rooms for the classes, the violations of constraints and the overlapping of sessions per student.
In its current version, the {\UTP} language addresses the problem as a hard constraint satisfaction problem.
The integration of soft constraints and the possibility of aggregating penalties or preferences, either in solution generation or repair contexts, is under investigation.

%Student groups 
% near-identical course structure: no course configuration element
% no time elements in UTP classes. Alternative class times are fixed in ITC
% OK time elements in extension -> unusitable for loosely constrained class programs
% OK impossible to express k weekly slots with different starting slots. In UTP: use sameWeeklySlot with masks.
% OK impossible to enforce different constraints bettwen first period of a course (amorcage) and the rest => ok for us with masks
% KO: times with different #sessions and session length. 
% => solution per session (vs per class) : slot + romms + teachers


%Sectioning:: as itc (parent class)
%Ressources
%- rooms: travel in ITC => using constraint travel in UTP.
%- new: teachers.
%- students: no change.
%- OK groups. Admin and computational needs.
%- Domain constraints
%- by default, all resources are cumultative (explain). Avec contrainte (disjunctive):: no overlap on romms/etc. 
%- allowed slot (applies to all sessions of a part's classes) : ITC via time elements. Adequate for "grid systems". 
%- allowed rooms and allowed teachers (worklaod per part = prescribed number of sessions)
%- single or multi-room/teacher session in UTP.
%- possibly different resources between sessions of a class (unless addiiotnal rules:: sameRoom, sameTeacher, ...)
%Rules language
%- ITC: class-level constraints vs session-level constraints on UTP
%- Labelling: simplifies constraint expression to look up entities
%Solution
%seul ajout: group-to-class and student-to-group
%Résolution: UTP == SAT vs Opt/gestion prefs => pas de priorites, pondérations, etc
%
