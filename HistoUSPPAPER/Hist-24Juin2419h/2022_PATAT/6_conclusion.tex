%------------------------------------------------------------
%------------------------------------------------------------
\section{Conclusion and Perspectives}
\label{sec:conclusion}
We introduced in this paper a domain-specific language for university course timetabling. 
The language allows to model a wide variety of course timetabling problems
such as those encountered in French universities. 
%This is particularly the case in French universities where students are divided into groups that attend all the courses they are registered for.
It provides support for typical timetabling entities (students, sessions, lecturers, rooms, groups) and features (student sectioning, resource distribution, session scheduling, resource allocation) and includes a rules language to easily express constraints (sequencing, periodicity, etc.).
Rules allow to target any subset of domain entities and sessions and enforce timetabling-specific predicates.

We used the language to encode a real instance (Bachelor courses of a French university) and 
implemented a tool chain to convert the \XML\ instance files into solver-compatible formats.
In order to validate our approach, we implemented a \CSP\ model in \MINIZINC\ and \CHR\ and produced solutions for the considered instance.

We are currently working on different extensions of the language and the back-end solvers.
First, we intend to represent preferences and priorities in order to support timetable optimization and repair tasks.
Second, the current {\CP} models may be improved using dedicated scheduling constraints, search strategies and heuristics and take advantage of model simplication and reformulation techniques.
Another objective is to improve scalability by testing our solvers on large-scale instances aggregating different curriculae or converted from academic benchmarks.
Lastly, we intend to investigate the revision of timetables to manage unexpected events (e.g.  unavailability of a lecturer, late registration of students) or to support incremental solution construction. 




%\begin{acknowledgements}
%This work is funded by a research grant from the university of Angers.
%\marc{Déjà sur la première page ?}
%\end{acknowledgements}


% Authors must disclose all relationships or interests that 
% could have direct or potential influence or impart bias on 
% the work: 
%
% \section*{Conflict of interest}
%
% The authors declare that they have no conflict of interest.


