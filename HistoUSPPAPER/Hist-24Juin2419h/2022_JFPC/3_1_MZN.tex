%------------------------------------------------------------
%------------------------------------------------------------
\subsection{Modèle \MINIZINC{}}
\label{sec:model-minizinc}

%------------------------------------------------------------
%------------------------------------------------------------


%\newcolumntype{N}{>{\refstepcounter{rowcntr}\therowcntr}r}
\newcounter{rowcntr}[table]
\renewcommand{\therowcntr}{(\arabic{rowcntr})}
\setcounter{rowcntr}{0}

\begin{table*}[!ht]
\framebox[\textwidth][c]{%
\small
\begin{tabularx}{\textwidth}{>{\hsize=0.01\hsize\linewidth=\hsize}X>{\hsize=1.89\hsize\linewidth=\hsize}X>{\raggedleft\arraybackslash\hsize=.09\hsize\linewidth=\hsize}X}
%\begin{math}\STUDENT = \funcmzn{array\_union}(\xgroup) \end{math} & 
%partition\_set(\xgroup,\begin{math}\STUDENT\end{math}) & \refstepcounter{rowcntr} \therowcntr \label{mzn:grouppartition}\\
%
%
&$\forallmzn(u,v \inmzn \STUDENT \wmzn u\gqmzn v)$ %&\\
%&\hspace*{2,8em}
$(\arraymzn{student\_courses}[u]\neqmzn \arraymzn{student\_courses}[v] \arrowmzn \xstudent[u]\neqmzn \xstudent[v]) $&  \refstepcounter{rowcntr} \therowcntr \label{mzn:studentgrouping}\\
%
%
&$\forallmzn(u \inmzn \STUDENT,p \inmzn \funcmzn{student\_parts}[u])$%&\\
%&\hspace*{2,8em}
$(\existmzn(k \inmzn \arraymzn{part\_classes}[p])(\xstudent[u] \inmzn \xgroup[k]))$& \refstepcounter{rowcntr} \therowcntr \label{mzn:allparts}\\
%forallmzn(u \inmzn \STUDENT, g \inmzn \GROUP)(\xstudent[u] = g \arrowmzn forall(p in \funcmzn{student\_parts}[u])(p = g in  xgroup ))
%
&$\forallmzn(p \inmzn \PART,k1,k2 \inmzn \arraymzn{part\_classes}[p] \wmzn k1\gqmzn k2)$%&\\
%
%\hspace{2cm}all_disjoint
%&\hspace*{2,8em}
$(\xgroup[k1] \intermzn \xgroup[k2]=\{\})$& \refstepcounter{rowcntr} \therowcntr \label{mzn:exclusiveclass}\\
%
%
&$\forallmzn(k1 \inmzn \CLASS, k2 \inmzn \funcmzn{class\_parents}(k1))(\xgroup[k1] \subsetmzn \xgroup[k2])$ &  \refstepcounter{rowcntr} \therowcntr \label{mzn:parent}\\
%
&$\forallmzn(k \inmzn \CLASS)(\arraymzn{maxsize}[k] \geqmzn \summzn(g \inmzn \GROUP)$%&\\
%&\hspace*{2,8em}
$(\funcmzn{bool2int}(g \inmzn \xgroup[k])*\summzn(u \inmzn \STUDENT)(\funcmzn{bool2int}(\xstudent[u]=g)))$ &  \refstepcounter{rowcntr} \therowcntr \label{mzn:classcapacity}\\
% 
%
\hline
%
%
&$\forallmzn(s \inmzn \SESSION)(\xroom[s] \subsetmzn \arraymzn{part\_rooms}[\funcmzn{session\_part}[s]])$& \refstepcounter{rowcntr} \therowcntr \label{mzn:allowedrooms}\\
%
%
&$\forallmzn(s \inmzn \SESSION)(\xteacher[s] \subsetmzn \arraymzn{part\_lecturers}[\funcmzn{session\_part}[s]]) $ &  \refstepcounter{rowcntr} \therowcntr \label{mzn:allowedteachers} \\
%
%
%$\forallmzn(k \inmzn \CLASS)(((\arraymzn{part\_room\_use}[\funcmzn{class\_part}(k)]=\text{none}) \Longleftrightarrow (\xroom[k] = \{\})) $&\\
&$\forallmzn(s \inmzn \SESSION, p \inmzn \PART \wmzn p=\funcmzn{session\_part}[s])($&\\ &\hspace*{2,8em}$(\arraymzn{part\_room\_use}[p]=\text{none} \arrowmzn \xroom[s] = \{\}) \landmzn (\arraymzn{part\_room\_use}[p]=\text{single} \arrowmzn \funcmzn{card}(\xroom[s]) =1)$&\\
&\hspace*{1,4em}$ \landmzn(\arraymzn{part\_room\_use}[p]=\text{multiple} \arrowmzn \funcmzn{card}(\xroom[s]) \leqmzn 1))$
& \refstepcounter{rowcntr} \therowcntr \label{mzn:multiroom}\\
%&$\forallmzn(s \inmzn \SESSION, p \inmzn \PART \wmzn p=\funcmzn{session\_part}[s])($&\\ &\hspace*{2,8em}$(\arraymzn{part\_room\_use}[p]=\text{none} \arrowmzn \xroom[s] = \{\}) $&\\
%&\hspace*{1em}$\landmzn (\arraymzn{part\_room\_use}[p]=\text{single} \arrowmzn \funcmzn{card}(\xroom[s]) = 1)  $&\\
%&\hspace*{1em}$\landmzn(\arraymzn{part\_room\_use}[p]=\text{multiple} \arrowmzn \funcmzn{card}(\xroom[s]) \leqmzn 1))$
%& \refstepcounter{rowcntr} \therowcntr \label{mzn:multiroom}\\
%
%
&$\forallmzn(s \inmzn \SESSION)( \funcmzn{card}(\xteacher[s]) = \funcmzn{team}[\funcmzn{session\_part}[s]])$ & \refstepcounter{rowcntr} \therowcntr 
\label{mzn:multiteacher}\\
%
%
&$\forallmzn(p \inmzn \PART,l \inmzn \arraymzn{part\_lecturers}[p])$%&\\
%&\hspace*{2,8em}
$(\summzn{}(s \inmzn \funcmzn{part\_sessions}[p]))(\funcmzn{bool2int}(l \inmzn \xteacher[s]) = \arraymzn{service}[l,p])) $& \refstepcounter{rowcntr} \therowcntr \label{mzn:partteacherservice}\\
%
%
\hline
%
%
%$\forallmzn(k \inmzn \CLASS)(\xroom[k] \subsetmzn \arraymzn{part\_rooms}[\funcmzn{class\_part}(k))$ & \refstepcounter{rowcntr} \therowcntr  \label{mzn:allowedsrooms}\\
%
%$\forallmzn(s \inmzn \SESSION)(\xteacher[s] \subsetmzn \arraymzn{part\_teachers}[\funcmzn{session\_part}[s]])$  & \refstepcounter{rowcntr} \therowcntr  \label{mzn:allowedsteacher}\\
%
&$\forallmzn(p \inmzn \PART,s \inmzn \funcmzn{part\_sessions}[p])$&\\
&\hspace*{2,8em}$(\funcmzn{week}(\xslot[s]) \inmzn \arraymzn{weeks}[p]\landmzn \funcmzn{weekday}(\xslot[s]) \inmzn \arraymzn{weekdays}[p]\landmzn \funcmzn{dailyslot}(\xslot[s]) \inmzn \arraymzn{dailyslots}[p])$& \refstepcounter{rowcntr} \therowcntr \label{mzn:allowedslots}\\
%&$\forallmzn(p \inmzn \PART,s \inmzn \funcmzn{part\_sessions}(p))$&\\
%&\hspace*{2,8em}$(\funcmzn{week}(\xslot[s]) \inmzn \arraymzn{weeks}[p]$ &\\
%&\hspace*{1em}$\landmzn \funcmzn{weekday}(\xslot[s]) \inmzn \arraymzn{weekdays}[p]$&\\
%&\hspace*{1em}$\landmzn \funcmzn{dailyslot}(\xslot[s]) \inmzn \arraymzn{dailyslots}[p])$& \refstepcounter{rowcntr} \therowcntr \label{mzn:allowedslots}\\
%
%
&$\forallmzn(s \inmzn \SESSION)((\xslot[s] - 1) \divmzn \gconst{nr\_daily\_slots} =(\xslot[s] + \funcmzn{length}[s] - 1) \divmzn \gconst{nr\_daily\_slots})$& \refstepcounter{rowcntr} \therowcntr \label{mzn:nopreemption}\\
%&$\forallmzn(s \inmzn \SESSION)$&\\
%&\hspace*{2,8em}$((\xslot[s] - 1) \divmzn \gconst{nr\_slots\_per\_day} =$&\\
%&\hspace*{2,8em}$(\xslot[s] + \funcmzn{length}[s] - 1) \divmzn \gconst{nr\_slots\_per\_day})$& \refstepcounter{rowcntr} \therowcntr \label{mzn:nopreemption}\\
%
%
&$ \forallmzn(k \inmzn \CLASS, s1,s2 \inmzn \funcmzn{class\_sessions}[k] \wmzn \funcmzn{rank}(s1) \gqmzn \funcmzn{rank}(s2))$ %&\\ 
%&\hspace*{2,8em}
$(\xslot[s1]+\funcmzn{length}[s] \leqmzn \xslot[s2]) $& \refstepcounter{rowcntr} \therowcntr \label{mzn:classsequencing}\\
%
%
%&$\forallmzn(s1 \inmzn \SESSION, r \inmzn \funcmzn{part\_rooms}[\funcmzn{session\_part}[s1]],s2 \inmzn %\funcmzn{room\_sessions}[r] $&\\
%&\hspace*{2,8em}$\wmzn \funcmzn{is\_multi\_rooms}[\funcmzn{session\_part}[s1]]) \landmzn s1 \neqmzn %s2)($&\\
%&\hspace*{3em}$\funcmzn{disjunctive}([\xslot[s1],\xslot[s2]],$&\\
%%
%&\hspace*{3em}$[\funcmzn{bool2int}(r \inmzn \xroom[s1])*\funcmzn{length}[s1],\funcmzn{bool2int}(r %\inmzn \xroom[s2])*\funcmzn{length}[s2]]))$ & \refstepcounter{rowcntr} \therowcntr  %\label{mzn:multiroomscheduling}\\
&$\forallmzn(p \inmzn \PART, s1 \inmzn \funcmzn{part\_sessions}[p], r \inmzn \funcmzn{part\_rooms}[p],s2 \inmzn \funcmzn{room\_sessions}[r] $%&\\
%&\hspace*{2,8em}
$\wmzn \funcmzn{is\_multi\_rooms}[p] \landmzn s1 \neqmzn s2)$&\\
&\hspace*{3em}$(\funcmzn{disjunctive}([\xslot[s1],\xslot[s2]],$&\\
%
&\hspace*{3em}$[\funcmzn{bool2int}(r \inmzn \xroom[s1])*\funcmzn{length}[s1],\funcmzn{bool2int}(r \inmzn \xroom[s2])*\funcmzn{length}[s2]]))$ & \refstepcounter{rowcntr} \therowcntr  \label{mzn:multiroomscheduling}\\
%
%
&$\forallmzn(p \inmzn \PART, s \inmzn \funcmzn{part\_sessions}[p] \wmzn \funcmzn{is\_multi\_rooms}[p])$&\\
&\hspace*{2,8em}$(\summzn(r \inmzn \arraymzn{part\_rooms}[p])(\funcmzn{bool2int}(r \inmzn \xroom[s]) * \arraymzn{capacity}[r])$&\\
&\hspace*{2,8em}$\geqmzn\summzn(g \inmzn \GROUP)(\funcmzn{bool2int}(g \inmzn \xgroup[\funcmzn{session\_class}[s]] )*\funcmzn{card}(\arraymzn{group\_students}[g])))$& \refstepcounter{rowcntr} \therowcntr \label{mzn:multiroomcapacity}\\
%
%
%$\forallmzn(s \inmzn \SESSION, r \inmzn \funcmzn{session\_rooms}[s])($&\\
%\multicolumn{1}{|c}{$\arraymzn{room\_capacity}[r] \geq sum(g \inmzn %\funcmzn{session\_group}[s])(\funcmzn{card}(\arraymzn{group\_students}[g]))))$} & \refstepcounter{rowcntr} \therowcntr  
%\label{mzn:cumulativeroomcapacity}\\
%
%
&$\forallmzn(p \inmzn \PART, s \inmzn \funcmzn{part\_sessions}[p])(\funcmzn{mandatory\_rooms}[p] \subsetmzn \xroom[s])$& \refstepcounter{rowcntr} \therowcntr \label{mzn:mandatoryrooms}\\
%
%
%&$\forallmzn(r \inmzn \ROOM  \wmzn \notmzn(\funcmzn{virtual}[r]))$&\\
%&\hspace*{2,8em}$(\funcmzn{cumulative}([\xslot[s] | s \inmzn \funcmzn{room\_sessions}[r]],$&\\
%&\hspace*{2,8em}$[\funcmzn{bool2int}(r \inmzn \xroom[s])* \funcmzn{length}[s] | s \inmzn \funcmzn{room\_sessions}[r]  ],$&\\
%&\hspace*{3em}$[\summzn(g \inmzn \GROUP)(\funcmzn{bool2int}(g \inmzn \xgroup[\funcmzn{session\_class}[s]])) * \summzn(u \inmzn \STUDENT)($&\\
%&$\funcmzn{bool2int}(g = \xstudent[u]))| s \inmzn \funcmzn{room\_sessions}[r]],\arraymzn{capacity}[r]))$& \refstepcounter{rowcntr} \therowcntr  
%\label{mzn:roomuse}\\
&$\forallmzn(r \inmzn \ROOM  \wmzn \notmzn(\funcmzn{virtual}[r]))(\funcmzn{let \{}\funcmzn{set of \SESSION: RS= room\_sessions}[r]\intermzn\funcmzn{single\_room\_sessions;}\}\inmzn$&\\
&\hspace*{2,8em}$(\funcmzn{cumulative}([\xslot[s] | s \inmzn RS],[\funcmzn{bool2int}(r \inmzn \xroom[s])* \funcmzn{length}[s] | s \inmzn RS],$&\\
%&$\forallmzn(r \inmzn \ROOM  \wmzn \notmzn(\funcmzn{virtual}[r]))($&\\
%&\hspace*{2,8em}$\funcmzn{let \{}\funcmzn{set of \SESSION: RS= room\_sessions}[r]\intermzn\funcmzn{single\_room\_sessions;}\}\inmzn$&\\
%&\hspace*{2,8em}$(\funcmzn{cumulative}([\xslot[s] | s \inmzn RS],[\funcmzn{bool2int}(r \inmzn \xroom[s])* \funcmzn{length}[s] | s \inmzn RS],$&\\
%&\hspace*{2,8em}$(\funcmzn{cumulative}([\xslot[s] | s \inmzn RS],$&\\
%&\hspace*{2,8em}$[\funcmzn{bool2int}(r \inmzn \xroom[s])* \funcmzn{length}[s] | s \inmzn RS],$&\\
&\hspace*{2,8em}$[\summzn(g \inmzn \GROUP)(\funcmzn{bool2int}(g \inmzn \xgroup[\funcmzn{session\_class}[s]])) * \summzn(u \inmzn \STUDENT)($&\\
&$\funcmzn{bool2int}(g = \xstudent[u]))| s \inmzn RS],\arraymzn{capacity}[r]))$& \refstepcounter{rowcntr} \therowcntr \label{mzn:roomuse}\\
%&\hspace*{2em}$[\summzn(g \inmzn \GROUP)(\funcmzn{bool2int}(g \inmzn \xgroup[\funcmzn{session\_class}[s]])) * \summzn(u \inmzn \STUDENT)($&\\
%&$\funcmzn{bool2int}(g = \xstudent[u]))| s \inmzn RS],\arraymzn{capacity}[r]))$& \refstepcounter{rowcntr} \therowcntr \label{mzn:roomuse}\\
%
%
%\forallmzn(t \inmzn \TEACHER)(\funcmzn{cumulative}( & \\
%\multicolumn{1}{|c}{$[\xslot[s] | s \inmzn \funcmzn{teacher\_sessions}(t)],[\funcmzn{bool2int}(t \subsetmzn \xteacher[s])* \funcmzn{session\_length}[s] | s \inmzn \funcmzn{teacher\_sessions}(t)],$}&\\
%\multicolumn{1}{|c}{$[1| s \inmzn \funcmzn{teacher\_sessions}(t)],teacher\_max\_use[t])))$}& \refstepcounter{rowcntr} \therowcntr  
%\label{mzn:teacheruse}\\
%
%
%$\forallmzn(g \inmzn \GROUP)(\funcmzn{cumulative}($&\\
%\multicolumn{1}{|c}{$[\xslot[s] | s \inmzn \funcmzn{group\_sessions}(g)],[\funcmzn{session\_length}[s] | s \inmzn \funcmzn{group\_sessions}(g)],$}&\\
%\multicolumn{1}{|c}{$[1| s \inmzn \funcmzn{group\_sessions}(g)],group\_max\_use[g])))$}& \refstepcounter{rowcntr} \therowcntr  
%\label{mzn:groupuse}\\
%
%
\hline
%
%
&$\FORBIDDENPERIOD((r,S'),h1,h2) = \forallmzn(i \inmzn S')( r \inmzn \xroom[i] \arrowmzn (\xslot[i]+\funcmzn{length}[i] \geqmzn h_1 \lormzn  \xslot[i]\lqmzn h_2)) $& \refstepcounter{rowcntr} \therowcntr \label{mzn:forbiddenperiod}\\
%&$\FORBIDDENPERIOD((r,S'),h1,h2) = \forallmzn(i \inmzn S')($&\\
%&\hspace*{2,8em}$ r \inmzn \xroom[i] \arrowmzn (\xslot[i]+\funcmzn{length}[i] \geqmzn h_1 \lormzn  \xslot[i]\lqmzn h_2)) $& \refstepcounter{rowcntr} \therowcntr \label{mzn:forbiddenperiod}\\
%
%
&$\SAMEWEEKDAY((r,S')) = \forallmzn(i,j \inmzn S' \wmzn i\gqmzn j)($&\\
&\hspace*{2,8em}$ (r\inmzn \xroom[i] \intermzn \xroom[j]) \arrowmzn(\xslot[i] \divmzn \gconst{nr\_weekly\_slots}=\xslot[j] \divmzn  \gconst{nr\_weekly\_slots}))$ & \refstepcounter{rowcntr} \therowcntr 
\label{mzn:sameweekday}\\
%&$\SAMEWEEKDAY((r,S')) = \forallmzn(i,j \inmzn S' \wmzn i\gqmzn j)($&\\
%&\hspace*{2,8em}$ (r\inmzn \xroom[i] \intermzn \xroom[j]) \arrowmzn $&\\
%&\hspace*{2,8em}$(\xslot[i] \divmzn \gconst{nr\_weekly\_slots}=\xslot[j] \divmzn  \gconst{nr\_weekly\_slots}))$ & \refstepcounter{rowcntr} \therowcntr \label{mzn:sameweekday}\\
%
%
&$\SAMEROOMS((r,S')) = \forallmzn(i,j \inmzn S' \wmzn i\gqmzn j)(($&\\
&\hspace*{2,8em}$r\inmzn \xroom[i] \intermzn \xroom[j]) \arrowmzn \xroom[i] = \xroom[j])$ & \refstepcounter{rowcntr} \therowcntr  \label{mzn:samerooms}\\
%
%
&$\SEQUENCED((r1,S1),(r2,S2)) = \forallmzn(i \inmzn S1,j \inmzn S2)($&\\
&\hspace*{3em}$(r1\inmzn \xroom[i] \landmzn r2\inmzn \xroom[j]) \arrowmzn\xslot[i]\texttt{+}\funcmzn{length}[i] \geqmzn \xslot[j])$ & \refstepcounter{rowcntr} \therowcntr \label{mzn:sequenced}\\
%
%
&$\NOOVERLAP((r,S'))=\funcmzn{disjunctive}([\xslot[i]|i \inmzn S'],[\funcmzn{length}[i]*\funcmzn{bool2int}(r \inmzn \xroom[i])|i \inmzn S'])$ & \refstepcounter{rowcntr} \therowcntr \label{mzn:nooverlap}\\
%
%&$\NOOVERLAP((r,S'))=$&\\
%&\hspace*{2,8em}$\funcmzn{disjunctive}([\xslot[i]|i \inmzn S'],[\funcmzn{length}[i]*\funcmzn{bool2int}(r \inmzn \xroom[i])|i \inmzn S'])$ & \refstepcounter{rowcntr} \therowcntr \label{mzn:nooverlap}\\
%
\end{tabularx}%
}%
\caption{Contraintes et prédicats du modèle }
\label{table:mzn-contraintes}
\end{table*}

%\corentin{}


% \begin{table*}[!ht]
% %\resizebox{\textwidth}{!}{%
% \small
% \centering
% \begin{tabular}{|lr|}
% \hline
% %\begin{math}\STUDENT = \funcmzn{array\_union}(\xgroup) \end{math} & 
% %partition\_set(\xgroup,\begin{math}\STUDENT\end{math}) & \refstepcounter{rowcntr} \therowcntr \label{ctr:grouppartition}\\
% %
% %
% $\forallmzn(u,v \inmzn \STUDENT \wmzn u<v)(%$&\\
% ( \arraymzn{student\_courses}[u]!=\arraymzn{student\_courses}[v] ) \arrowmzn ( \xstudent[u]!=\xstudent[v] )) $&  \refstepcounter{rowcntr} \therowcntr \label{ctr:studentgrouping}\\
% %
% %
% $\forallmzn(u \inmzn \STUDENT,p \inmzn \funcmzn{student\_parts}(u))(\existmzn(k \inmzn \arraymzn{part\_classes}[p])(\xstudent[u] \inmzn \xgroup[k])))$& \refstepcounter{rowcntr} \therowcntr \label{ctr:allparts}\\
% %forallmzn(u \inmzn \STUDENT, g \inmzn \GROUP)(\xstudent[u] = g \arrowmzn forall(p in \funcmzn{student\_parts}(u))(p = g in  xgroup ))
% %
% $\forallmzn(p \inmzn \PART,k_1,k_2 \inmzn \arraymzn{part\_classes}[p] \wmzn k_1<k_2)($%&\\
% %
% %\hspace{2cm}all_disjoint
% $ \xgroup[k_1] \intermzn \xgroup[k_2] = \{\})$& \refstepcounter{rowcntr} \therowcntr \label{ctr:exclusiveclass}\\
% %
% %
% $\forallmzn(k_1 \inmzn \CLASS, k_2 \inmzn \funcmzn{class\_parent}(k_1))(\xgroup[k_1] \subsetmzn \xgroup[k_2])$ &  \refstepcounter{rowcntr} \therowcntr \label{ctr:parent}\\
% %
% $\forallmzn(k \inmzn \CLASS)(\arraymzn{  class\_capacity}[k] \geqmzn \summzn(g \inmzn \GROUP)( \funcmzn{ bool2int}(g \inmzn \xgroup[k]) * \summzn(u \inmzn \STUDENT)(\funcmzn{bool2int}(\xstudent[u] == g))))$ &  \refstepcounter{rowcntr} \therowcntr \label{ctr:classcapacity}\\
% % 
% %
% \hline
% %
% %
% $\forallmzn(s \inmzn \SESSION)(\xroom[s] \subsetmzn \arraymzn{part\_rooms}[\funcmzn{session\_part}(s)])$& \refstepcounter{rowcntr} \therowcntr \label{ctr:allowedrooms}\\
% %
% %
% $\forallmzn(s \inmzn \SESSION)(\xteacher[s] \subsetmzn \arraymzn{part\_teachers}[\funcmzn{session\_part}(s)]) $ &  \refstepcounter{rowcntr} \therowcntr \label{ctr:allowedteachers} \\
% %
% %
% %$\forallmzn(k \inmzn \CLASS)(((\arraymzn{part\_room\_use}[\funcmzn{class\_part}(k)]=\text{none}) \Longleftrightarrow (\xroom[k] = \{\})) $&\\
% $\forallmzn(s \inmzn \SESSION)((( $\hspace{1,3cm}$\arraymzn{part\_room\_use}[\funcmzn{session\_part}(s)]=\text{none}) \equimzn (\xroom[s] = \{\})) $&\\
% \multicolumn{1}{|c}{$\landmzn ((\arraymzn{part\_room\_use}[\funcmzn{session\_part}(s)]=\text{single}) \arrowmzn (\funcmzn{card}(\xroom[s]) = 1))  $}&\\
% \multicolumn{1}{|c}{$\landmzn((\arraymzn{part\_room\_use}[\funcmzn{session\_part}(s)]=\text{multiple}) \arrowmzn (\funcmzn{card}(\xroom[s])>=1)))$}
% & \refstepcounter{rowcntr} \therowcntr 
% \label{ctr:multiroom}\\
% %
% %
% $\forallmzn(s \inmzn \SESSION)( \funcmzn{card}(\xteacher[s]) = \funcmzn{teacher\_per\_session}[s])$ & \refstepcounter{rowcntr} \therowcntr 
% \label{ctr:multiteacher}\\
% %
% %
% \davidl{use GCC for (10)}
% $\forallmzn(p \inmzn \PART,t \inmzn \arraymzn{part\_teachers}[p])(\summzn{}(s \inmzn \funcmzn{part\_sessions}(p))(\funcmzn{bool2int}(t \inmzn \xteacher[s])) = \arraymzn{part\_teacher\_service}[p,t]) $& \refstepcounter{rowcntr} \therowcntr 
% \label{ctr:partteacherservice}\\
% %
% %
% \hline
% %
% %
% %$\forallmzn(k \inmzn \CLASS)(\xroom[k] \subsetmzn \arraymzn{part\_rooms}[\funcmzn{class\_part}(k))$ & \refstepcounter{rowcntr} \therowcntr  \label{ctr:allowedsrooms}\\
% %
% %$\forallmzn(s \inmzn \SESSION)(\xteacher[s] \subsetmzn \arraymzn{part\_teachers}[\funcmzn{session\_part}(s)])$  & \refstepcounter{rowcntr} \therowcntr  \label{ctr:allowedsteacher}\\
% %
% $\forallmzn(p \inmzn \PART,s \inmzn \funcmzn{part\_sessions}(p))(($\hspace{0.5cm}$\funcmzn{slot\_week}(\xslot[s]) \inmzn \arraymzn{part\_weeks}[p])$ &\\
%   \multicolumn{1}{|c}{$\landmzn (\funcmzn{slot\_weekday}(\xslot[s]) \inmzn \arraymzn{part\_days}[p])$} &\\
%   \multicolumn{1}{|c}{$\landmzn (\funcmzn{slot\_dailyslot}(\xslot[s]) \inmzn \arraymzn{part\_dailyslots}[p]))$}& \refstepcounter{rowcntr} \therowcntr 
% \label{ctr:allowedslots}\\
% %
% %
% $\forallmzn(s \inmzn \SESSION)((\xslot[s] - 1) \divmzn \gconst{nr\_slots\_per\_day} = (\xslot[s] + \funcmzn{session\_length}(s) - 1) \divmzn \gconst{nr\_slots\_per\_day})$& \refstepcounter{rowcntr} \therowcntr 
% \label{ctr:nopreemption}\\
% %
% %
% $ \forallmzn(k \inmzn \CLASS, s_1,s_2 \inmzn \funcmzn{class\_sessions}[k] \wmzn \funcmzn{session\_rank}(s_1)<\funcmzn{session\_rank}(s_2) )$ &\\
% \multicolumn{1}{|c}{$(\xslot[s_1]+\funcmzn{session\_length}(s) \leqmzn \xslot[s_2])) $}& \refstepcounter{rowcntr} \therowcntr 
% \label{ctr:classsequencing}\\
% %
% %
% $\forallmzn( s \inmzn \SESSION, r \inmzn \funcmzn{class\_rooms}(\funcmzn{session\_class}(s)),s_p \inmzn \funcmzn{room\_sessions}(r)] \wmzn \funcmzn{is\_multi\_rooms}(s)) \landmzn s != s_p)($&\\
% \multicolumn{1}{|c}{$\funcmzn{disjunctive}([\xslot[s],\xslot[s_p] ],$}&\\
% %
% \multicolumn{1}{|c}{$[\funcmzn{bool2int}(r \inmzn \xroom[s])* \funcmzn{session\_length}(s),\funcmzn{bool2int}(r \inmzn \xroom[s_p])* \funcmzn{session\_length}(s_p)])))$} & \refstepcounter{rowcntr} \therowcntr  \label{ctr:multiroomscheduling}\\
% %
% %
% $\forallmzn(s \inmzn \SESSION \wmzn \funcmzn{is\_multi\_rooms}(s))( \summzn(r \inmzn \arraymzn{session\_rooms}[s])($&\\
% $\funcmzn{bool2int}(r \inmzn \xroom[s]) * \arraymzn{room\_capacity}[r]) \geqmzn \summzn(g \inmzn \arraymzn{class\_group}[\funcmzn{session\_class}(s)])(\funcmzn{card}(\arraymzn{group\_students}[g]))$& \refstepcounter{rowcntr} \therowcntr  
% \label{ctr:multiroomcapacity}\\
% %
% %
% %$\forallmzn(s \inmzn \SESSION, r \inmzn \funcmzn{session\_rooms}(s))($&\\
% %\multicolumn{1}{|c}{$\arraymzn{room\_capacity}[r] \geq sum(g \inmzn %\funcmzn{session\_group}(s))(\funcmzn{card}(\arraymzn{group\_students}[g]))))$} & \refstepcounter{rowcntr} \therowcntr  
% %\label{ctr:cumulativeroomcapacity}\\
% %
% %
% $\forallmzn(s \inmzn \SESSION \wmzn \funcmzn{has\_mandatory\_room}(s))(\xroom[s] \intermzn \funcmzn{session\_mandatory}[s] != \{\})$& \refstepcounter{rowcntr} \therowcntr 
% \label{ctr:mandatoryrooms}\\
% %
% %
% $\forallmzn(r \inmzn \ROOM)($&\\
% \multicolumn{1}{|c}{$\funcmzn{cumulative}([\xslot[s] | s \inmzn \funcmzn{room\_sessions}(r)],[\funcmzn{bool2int}(r \inmzn \xroom[s])* \funcmzn{session\_length}(s) | s \inmzn \funcmzn{room\_sessions}(r)  ],$}&\\
% \multicolumn{1}{|c}{$[\summzn(g \inmzn \GROUP)(\funcmzn{bool2int}(g \inmzn \xgroup[\funcmzn{session\_class}(s)]) * \summzn(u \inmzn \STUDENT)(\funcmzn{bool2int}(g = \xstudent[u]))| s \inmzn \funcmzn{room\_session}(r)],$}&\\
% $\arraymzn{room\_capacity}(r))))$& \refstepcounter{rowcntr} \therowcntr  
% \label{ctr:roomuse}\\
% %
% %
% %\forallmzn(t \inmzn \TEACHER)(\funcmzn{cumulative}( & \\
% %\multicolumn{1}{|c}{$[\xslot[s] | s \inmzn \funcmzn{teacher\_sessions}(t)],[\funcmzn{bool2int}(t \subsetmzn \xteacher[s])* \funcmzn{session\_length}(s) | s \inmzn \funcmzn{teacher\_sessions}(t)],$}&\\
% %\multicolumn{1}{|c}{$[1| s \inmzn \funcmzn{teacher\_sessions}(t)],teacher\_max\_use[t])))$}& \refstepcounter{rowcntr} \therowcntr  
% %\label{ctr:teacheruse}\\
% %
% %
% %$\forallmzn(g \inmzn \GROUP)(\funcmzn{cumulative}($&\\
% %\multicolumn{1}{|c}{$[\xslot[s] | s \inmzn \funcmzn{group\_sessions}(g)],[\funcmzn{session\_length}(s) | s \inmzn \funcmzn{group\_sessions}(g)],$}&\\
% %\multicolumn{1}{|c}{$[1| s \inmzn \funcmzn{group\_sessions}(g)],group\_max\_use[g])))$}& \refstepcounter{rowcntr} \therowcntr  
% %\label{ctr:groupuse}\\
% %
% %
% \hline
% %
% %
% $\FORBIDDENPERIOD((e,S_p),h_1,h_2) = \forallmzn(i \inmzn S_p )( e \inmzn \xroom[i] \arrowmzn (\xslot[i] < h_1) \landmzn  (\xslot[i]>h_2)) $& \refstepcounter{rowcntr} \therowcntr 
% \label{ctr:forbiddenperiod}\\
% %
% %
% $\SAMEWEEKDAY((e,S_p)) = \forallmzn(i,j \inmzn S_p \wmzn i<j)($&\\
% \multicolumn{1}{|c}{$ (e\inmzn \xroom[i] \landmzn e\inmzn \xroom[j]) \arrowmzn (\xslot[i] \divmzn nr\_weekly\_slots) = (\xslot[j] \divmzn nr\_weekly\_slots))$} & \refstepcounter{rowcntr} \therowcntr 
% \label{ctr:sameweekday}\\
% %
% %
% $\SAMEROOMS((e,S_p)) = \forallmzn(s_1,s_2 \inmzn S_p \wmzn s_1<s_2 )(($&\\
% %
% \multicolumn{1}{|c}{$e\inmzn \xroom[s_1] \landmzn e\inmzn \xroom[s_2]) \arrowmzn \xroom[s_1] = \xroom[s_2])$} & \refstepcounter{rowcntr} \therowcntr 
% \label{ctr:samerooms}\\
% %
% %
% $\SEQUENCED((e_1,S_1),(e_2,S_2)) = \forallmzn(i \inmzn S_1,j \inmzn S_2)($&\\
% \multicolumn{1}{|c}{$(e_1\inmzn \xroom[i] \landmzn e_2\inmzn \xroom[j]) \arrowmzn \xslot[i]+\funcmzn{session\_length}(i) \leqmzn \xslot[j])$} & \refstepcounter{rowcntr} \therowcntr 
% \label{ctr:sequenced}\\
% %
% %
% $\NOOVERLAP((e_1,S_1))= \funcmzn{disjunctive}([\xslot[s]|s \inmzn S_1 ],[\funcmzn{session\_length}(s)*\funcmzn{bool2int}(e_1 \inmzn \xroom[s])|s \inmzn S_1])$ & \refstepcounter{rowcntr} \therowcntr 
% \label{ctr:nooverlap}\\
% \hline
% \end{tabular}
% %}
% \caption{Contraintes et prédicats du modèle \MINIZINC{}}
% \label{table:mzn-contraintes}
% \end{table*}

%------------------------------------------------------------
%------------------------------------------------------------
%\subsection{Student Sectioning}
%\label{sec:model-sectioning}

{\MINIZINC} est un langage de modélisation haut-niveau de problèmes d'optimisation sous contraintes \cite{MZN,MINIZINC}. 
Les modèles {\MINIZINC} sont traduits dans le langage cible {\FLATZINC} \cite{FLATZINC} qui permet d'interfacer différents types de solveurs dont les solveurs de programmation par contraintes sur domaines finis tels  {\GECODE} \cite{GECODE}.
{\MINIZINC} intègre de nombreuses contraintes globales et le modèle {\UTP} présenté en Table~\ref{table:mzn-contraintes} et utilisant les variables de décision présentée en Table~\ref{table:cp-variables} s'appuie sur quelques contraintes dédiées aux problèmes d'ordonnancement.

%\davidl{Expliquer fichier DZN + fichier MZN avec implémentation prédicats UTP}
Les contraintes de sectionnement répartissent
les étudiants dans les groupes et assigne chacun de ces groupes à différentes classes 
conformément aux règles de sectionnement et aux seuils d'effectifs.
%La contrainte~\ref{mzn:grouppartition} partitionne les étudiants en groupes.
La contrainte \ref{mzn:studentgrouping} n'autorise le regroupement
d'étudiants que s'ils sont inscrits aux mêmes cours.
\ref{mzn:allparts} impose que tout étudiant, assimilé à son groupe, assiste à toute partie de cours dans lequel il est inscrit.
\ref{mzn:exclusiveclass} assure que les classes d'une partie de cours n'ont aucun groupe en commun.
\ref{mzn:parent} implémente la relation de parenté entre classes.
Enfin, \ref{mzn:classcapacity} vérifie que l'effectif cumulé des groupes attribués à une classe ne dépasse pas le seuil autorisé. 
%A noter que cette contrainte utilise des variables auxiliaires pseudo-booléennes.

%------------------------------------------------------------
%------------------------------------------------------------
%\subsubsection{Resource Distribution}
%\label{sec:model-distribution}

La distribution des ressources s'appuie sur des contraintes de domaine, de cardinalité et de sommes.
Les contraintes~\ref{mzn:allowedrooms} et \ref{mzn:allowedteachers} définissent les salles et enseignants allouables à chaque séance.
\ref{mzn:multiroom} contraint le nombre de salles allouées à une séance selon que sa partie de cours est sans salles, mono-salle, ou multi-salles.
\ref{mzn:multiteacher} attribue le nombre attendu d'enseignants à chaque séance
et \ref{mzn:partteacherservice} vérifie que chaque enseignant dispense le volume de séances requis par partie de cours où il est pré-positionné.

%------------------------------------------------------------
%------------------------------------------------------------
%\subsubsection{Session Scheduling and Resource Allocation}
%\label{sec:model-scheduling}

La programmation des séances et l'allocation des ressources met en jeu des contraintes de positionnement, de séquencement, de non-chevauchement et de capacité. 
%Constraint~\ref{mzn:allowedslots} defines the allowed slots per session
%and 
La contrainte~\ref{mzn:allowedslots} définit les créneaux autorisés pour chaque séance.
\ref{mzn:nopreemption} interdit qu'une séance soit à cheval sur 2 journées. 
\ref{mzn:classsequencing} séquence les séances d'une classe selon leur rangs.
Les contraintes \ref{mzn:multiroomscheduling} et \ref{mzn:multiroomcapacity} modélisent les séances multi-salles et l'accès exclusif à leurs ressources.
\ref{mzn:multiroomscheduling} impose qu'une ressource allouée à une séance multi-salles soit disjonctive le temps de son utilisation.
\ref{mzn:multiroomcapacity} assure que le nombre d'étudiants attendus n'excède pas la capacité cumulée des salles allouées.
À noter que cette contrainte est purement quantitative et autorise toute répartition d'étudiants dans les salles indépendamment de la structure de groupes.
\ref{mzn:mandatoryrooms} modélise les salles à allouer obligatoirement à toute séance d'une partie de cours.
\ref{mzn:roomuse} modélise la contrainte de capacité cumulative qui s'applique par défaut à toute salle allouée hors séances multi-salles.
%\ref{mzn:groupuse}

%------------------------------------------------------------
%------------------------------------------------------------

%\subsubsection{{\UTP} Predicates}
%\label{sec:model-predicates}

%%%%{\ADJACENTROOMS}
%%%%{\ATMOSTDAILY}
%%%%{\ATMOSTWEEKLY}
%%%%{\TRAVEL}
%{\FORBIDDENPERIOD}
%{\NOOVERLAP}
%%%%{\SAMEDAILYSLOT}
%%%%{\SAMEDAY}
%{\SAMEROOMS}
%%%%{\SAMESLOT}
%%%%{\SAMESTUDENTS}
%%%%{\SAMETEACHERS}
%{\SAMEWEEKDAY}
%%%%{\SAMEWEEKLYSLOT}
%%%%{\SAMEWEEK}
%{\SEQUENCED}
%%%%{\TEACHERDISTRIBUTION}
%%%%{\WEEKLY}

%Due to lack of space, we just present a few {\UTP} constraint predicates, namely,
La table~\ref{table:mzn-contraintes} présente les variantes de quelques prédicats \UTP{} dans le cas où les entités ciblées sont des salles.
%, respectivement,
%\texttt{\FORBIDDENPERIOD} (\ref{mzn:forbiddenperiod}),
%\texttt{\SAMEWEEKDAY} (\ref{mzn:sameweekday}), 
%\texttt{\SAMEROOMS} (\ref{mzn:samerooms}),
%\texttt{\NOOVERLAP} (\ref{mzn:nooverlap}) 
%et \texttt{\SEQUENCED} (~\ref{mzn:sequenced}).
%Note that \texttt{\FORBIDDENPERIOD} accepts start and end point parameters.
\ref{mzn:forbiddenperiod} implémente le prédicat \texttt{\FORBIDDENPERIOD} qui prend en paramètres les 2 créneaux modélisant la période interdite. 
\ref{mzn:sameweekday}, \ref{mzn:samerooms} et \ref{mzn:sequenced} modélisent de manière directe les prédicats \texttt{\SAMEWEEKDAY}, \texttt{\SAMEROOMS} et \texttt{\SEQUENCED}.
\ref{mzn:nooverlap} implémente le prédicat \texttt{\NOOVERLAP} en s'appuyant sur la contrainte globale \texttt{disjunctive}. %de \MINIZINC{}.


% Note that entailed constraints may be enforced. 
% For instance, a room $r$ is necessarily disjunctive if a non-overlapping constraint is enforced on its set of compatible sessions, %\footnote{
%Il faut noter que des contraintes peuvent être renforcés. 
%Par exemple, une salle $r$ est nécessairement disjonctive si une contrainte de non-chevauchement est appliquée à son ensemble de sessions possibles, c'est-à-dire si l'instance inclut la contrainte ${no\_overlap}$.
% i.e., if the instance includes constraint ${no\_overlap}((r,\map{\ROOM}{\SESSION}{r}))$.
% If so, the default cumulative capacity constraint may be safely replaced with  Constraint~\ref{mzn:disjunctiveroomcapacity} for such resources where
% $\disjunctiverooms\subseteq{\ROOM}$         denotes the set of disjunctive rooms.
%Dans ce cas, la contrainte cumulative par défaut peut être remplacée en toute sécurité par la contrainte~\ref{mzn:disjunctiveroomcapacity} pour les ressources concernés.
