Le schéma {\UTP} découpe la modélisation %modele
d'une instance d'emploi du temps en 3 
composants %fragment 
syntaxiques qui sont un modèle d'entités, un ensemble de règles et une solution.
%The {\UTP} schema splits the modeling of a timetabling instance into three syntactic components, namely, an entity model, a rules set and a solution.
Le modèle d'entités définit l'horizon de temps, les cours et les ressources de l'instance. 
Cela encode la compatibilité des ressources avec des éléments de cours, la capacité des ressources, %et les flux
les contraintes de séquencement de séances et la répartition des étudiants en groupes.
%
%The entity model defines the schedule horizon, courses and resources of the instance. It also encodes core %domain and cardinality
%compatibility, capacity and flow 
%constraints relating to session scheduling, resource allocation, and student sectioning.
%Rules express additional constraints meant to capture stakeholder requirements on particular aspects of the problem.
%
Les règles expriment des contraintes supplémentaires destinées à capturer les exigences des différentes entités sur des aspects particuliers du problème.
%
%To this end, the schema provides a catalog of timetabling-specific constraint predicates and embeds a rules language to apply predicates on chosen entities.
%
Pour cela, le schéma fournit un catalogue de contraintes spécifiques au problème d'emploi du temps et un langage de règles pour appliquer les prédicats sur les entités sélectionnées.
%
%The solution component includes a list of choices made for some or all of the decisions at stake (e.g., start time of a session).
L'élément solution contient une liste de décisions qui peuvent être totalement ou partiellement effectuées pour des entités (e.g. le créneau d'une séance).
%
%Note that rules set and solution components may be omitted. 
%
L'ensemble de règles et l'élément solution peuvent être omis si non nécessaire.
% Besides, the listed solution is not required to be consistent with the constraints enforced by the entity model or the rules set.
% This allows to tackle subproblems using separate instances and to support timetable generation or repair tasks.

%The rules language has generative semantics in the sense that each rule is the intensional representation of a collection of constraints sharing the same predicate.

Le langage de règles a une sémantique générative dans le sens où chacune des règles est une représentation en intention d'une collection de contraintes qui partage le même prédicat.
%
%{\UTP} instances may therefore be compiled into lower-level representations that lists all constraints and whose format is tailored to back-end solvers.
%
Les instances {\UTP}  peuvent alors être compilées dans des formats en extension qui sont des représentations bas-niveau sous forme de liste de contraintes dans un format lisible pour les solveurs.

%The schema is currently implemented as an {\XML} language.
%It comes with a parser that converts instances into {\JSON} and {\DZN} formats %({\Minizinc} input data format) 
%and can flatten rules into constraint collections or leave them as is.
%The detailed XML syntax and alternative formats are beyond the scope of this paper and may be found in \cite{USPsite}.

Le schéma est actuellement implémenté en {\XML}. Il est accompagné d'un parser capable de convertir une instance au format {\JSON} ou {\DZN}, et génère une collection de contraintes à partir des règles. % capable décomposer les règles sous forme de collection de contrainte. 
%
Le détail de la syntaxe {\XML} et des formats alternatif dépassent le cadre de cet article et sont disponibles sur \cite{USPsite}.
%{\UTP} predicates and the entity meta-model may be directly implemented as a constraint-based model in any {\CP} language.
%Instantiating such a model simply requires listing all constraints and entities in a compatible format.
%{\UTP} instances are thus compilable into {\CP} instances by .
 
%motivate design choices 
%We introduce below the conceptual model,
%provide set-theoretic semantics for the rules language, 
%and highlight differences with related work.
%the catalog of predicates 
% and the rules language.
%We also motivate design choices
% introduce the catalog of predicates,
% and describe the generative semantics of the rules language.


%Nous introduisons maintenant le modèle conceptuel, fournissons une sémantique de la théorie des ensembles  pour le langage de règle, et soulignons les différences avec des travaux similaires.

% The {\UTP} schema splits the modeling of timetabling problem instances into three components, namely, an entity model, a rules set and a solution.
% It embeds a specific language to declare rules on entities based on a %predefined 
% catalog of timetabling-specific predicates.
% Each rule is an intensional representation of a collection of constraints applying to different entities.
% {\UTP} instances may thus be compiled to lower-level representations that explicitly declare all constraints and whose format is compatible with {\CP} languages.
% We provide below an informal description of the syntactic components
% %including the {\UTP} constraint predicate catalog
% (the detailed XML syntax of the schema may be found in \cite{USPsite}),
% motivating design choices and describing the generative semantics of the rules language,
% and we highlight differences with related work.