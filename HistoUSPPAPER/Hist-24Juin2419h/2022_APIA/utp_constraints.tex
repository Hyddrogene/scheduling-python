%------------------------------------------------------------
%------------------------------------------------------------
\subsection{Predicates and constraints}
\label{sec:constraints}
{\UTP} constraints apply to pairs, called e-maps, which associate an entity with a non-empty subset of its compatible sessions.
%which we call e-maps. 
Constraints are built with predicates whose signature includes e-map variables%ranging over the set of e-maps
, the number of which is referred to as the arity of the predicate. 
Note that some predicates may also accept parameters.
Let 
${\EMAP}=
\setunion{X}{\TYPE}
\myset{(e,S')\ |\ e\in X,S'\subseteq\map{X}{\SESSION}{e}\wedge S'\neq\emptyset}$
denote the set of e-maps,
a {\UTP} constraint has the form
\begin{align}
c((e_1,S_1),\ldots,(e_m,S_m),p_1,\ldots,p_n) \label{rule:constraint}
\end{align}
where 
$c$ is a predicate symbol of arity $m$,
$(e_1,S_1)$, $\ldots$, $(e_m,S_m)$ are e-maps ($(e_i,S_i)\in{\EMAP}$, $i=1\ldots m$) 
and 
$p_1,\ldots,p_n$ are values for the parameters of $c$ ($n\geq0$).


Every predicate may be used indistinctly with e-maps defined on course elements or on resources.
E-maps defined on resources are interpreted as conditional session-to-resource assignments
when checking constraints 
whereas e-maps defined on course elements are unconditional assignments since they model constitutive sessions.
In other words, 
a constraint is only evaluated
on the sessions whose assignments in the considered solution are consistent with its e-map arguments.\footnote{Formally, let $\var{E}{\SESSION}{e}$ be the variable denoting the set of sessions assigned to entity $e$ and $S'_1,\ldots,S'_m$ be sets of sessions, the conditionality of a constraint $c$ is stated as follows: 
$(\var{E}{\SESSION}{e_1}=S'_1 \wedge \var{E}{\SESSION}{e_m}=S'_m)
\Rightarrow
(c((e_1,S_1),\ldots,(e_m,S_m),p_1,\ldots,p_n)
\Leftrightarrow
c((e_1,S_1\cap S'_1),\ldots,(e_m,S_m\cap S'_m),p_1,\ldots,p_n))$.
}
It follows that 
a constraint is evaluated on every session that is mapped to a course element.
In particular, constraints that apply exclusively to course elements are unconditional. 
Note also that the use of e-maps that model the whole set of sessions compatible with an entity 
will necessarily constrain any session that may be assigned to this entity.


%Every predicate may be used indistinctly with e-maps defined on course elements or on resources which we call c-maps and r-maps, respectively. R-maps are interpreted conditionally since they map a resource to some of its possible sessions whereas c-maps model unconditional assignments since they model constitutive sessions of course elements. In other words, a constraint must be evaluated on every session of every c-map in its scope but only on the sessions of its r-maps whose resource assignment is compatible with the proposed solution.\footnote{Formally, let $\var{E}{\SESSION}{e}$ be the variable denoting the set of sessions assigned to entity $e$ and $S'_1,\ldots,S'_m$ be sets of sessions, the conditionality of a constraint $c$ is stated as follows:  $(\var{E}{\SESSION}{e_1}=S'_1 \wedge \var{E}{\SESSION}{e_m}=S'_m) \Rightarrow (c((e_1,S_1),\ldots,(e_m,S_m),p_1,\ldots,p_n) \Leftrightarrow c((e_1,S_1\cap S'_1),\ldots,(e_m,S_m\cap S'_m),p_1,\ldots,p_n))$.}
%effectively assigns to the resource of the r-map.
%(see Rule (\ref{rule:conditionality})).
%It follows that constraints applying exclusively to c-maps are unconditional. Besides, the scoping of e-maps that model the whole set of sessions compatible with an entity will constrain every session assigned to a resource or constitutive of a course element.
%%The rule below %(\ref{rule:conditionality}) models the conditionality of constraints.

%{\footnotesize{
%\begin{multline}
%\forall S'_1,\ldots,S'_m\in{\SESSION}:
%(\var{E}{\SESSION}{e_1}=S'_1 \wedge \var{E}{\SESSION}{e_m}=S'_m)
%\Rightarrow\\
%(c((e_1,S_1),\ldots,(e_m,S_m),p_1,\ldots,p_n)
%\Leftrightarrow
%c((e_1,S_1\cap S'_1),\ldots,(e_m,S_m\cap S'_m),p_1,\ldots,p_n))
%\label{rule:conditionality}
%\end{multline}
%}}

\begin{table*}[!ht]
\resizebox{\textwidth}{!}{%
\centering
\begin{tabular}{|l|l|l|l|}
\hline
\textbf{Name}               & \textbf{Arity} & \textbf{Parametric} & \textbf{Semantics}\\ \hline

%assign\_slot               & 1         & yes   & Assign a slot or slot tuple to a session\\ \hline
%assign\_room               & 1         & yes   & Assign a set of room to session in entry\\ \hline

%allocation\_group           & 1        & no    & Domain allocation for class with group in the solution\\ \hline
%part\_schedule              & 1        & no    & Allowed start time slots for sessions\\ \hline
%domain\_class\_group        & 1        & no    & Allowed groups for classes (solution input)\\ \hline
%domain\_session\_teacher    & 1        & no    & Allowed teachers for sessions\\ \hline
%domain\_class\_room         & 1        & no    & Allowed rooms for sessions\\ \hline

{\SAMEDAILYSLOT}            & 1         & no    & Sessions start on the same daily slot\\ \hline
{\SAMEWEEKDAY}              & 1         & no    & Sessions start on the same weekday\\ \hline
{\SAMEWEEKLYSLOT}           & 1         & no    & Sessions start on the same weekly slot\\ \hline
{\SAMEWEEK}                 & 1         & no    & Sessions start the same week\\ \hline
{\SAMEDAY}                  & 1         & no    & Sessions start the same day\\ \hline
{\SAMESLOT}                 & 1         & no    & Sessions start at the same time\\ \hline
{\FORBIDDENPERIOD}          & 1         & yes   & Sessions cannot start in the time period\\ \hline
{\ATMOSTDAILY}              & 1         & yes   & The number of sessions scheduled in the daily period is upper-bounded\\ \hline
{\ATMOSTWEEKLY}             & 1         & yes   & The number of sessions scheduled in the weekly period is upper-bounded\\ \hline
%implicit\_sequenced\_sessions & 1 & \multicolumn{4}{|c|}{no} & All sessions in classes are sequenced\\ \hline
{\SEQUENCED}                & $\geq2$   & no    & Sessions are sequenced\\ \hline
{\WEEKLY}                   & 1         & no    & Sessions are weekly \\ \hline

{\NOOVERLAP}                & 1         & no    & Sessions cannot overlap\\ \hline
{\TRAVEL}                   & 1         & yes   & Travel time is factored in if sessions hosted in the rooms\\ \hline

{\SAMEROOMS}                & 1         & no    & Sessions are hosted in the same room(s)\\ \hline
{\SAMESTUDENTS}             & 1         & no    & Sessions are attended by the same student(s)\\ \hline
{\SAMETEACHERS}             & 1         & no    & Sessions are taught by the same teacher(s)\\ \hline

{\ADJACENTROOMS}            & 1         & yes   & Sessions are hosted in the adjacent rooms\\ \hline

{\TEACHERDISTRIBUTION}      & $\geq2$   & yes   & Distributes teacher workload over classes\\ \hline

\end{tabular}
}
\caption{Catalog of {\UTP} predicates}
\label{tab:predicate_catalog}
\end{table*}


% 
% \newcolumntype{M}[1]{>{\raggedright}m{#1}}
%\begin{table}[!h]
%    \centering
%    \begin{tabular}{|l|M{2cm}|*{7}{c|}}
%        \hline
%        \multirow{2}{4em}{Name} & \multirow{2}{4em}{Entity} & \multirow{2}{1cm}{Arity} & \multicolumn{4}{|c|}{Parameter} & \multirow{2}{6em}{Conditional} & \multirow{2}{10em}{Explication}   \\
%        \cline{4-7}
%           & & &name& type& number& type & &    \\
%        \hline
%
%
%assign\_slot & All & max 1 & slot & max 1 & min 1 & slots  & yes & Assign a slot or slot tuple to a session\\ \hline
%
%allocation\_group&Part& max 1  & \multicolumn{4}{|c|}{no}  & no  & Domain allocation for class with group in the solution\\ \hline
%
%assign\_room  & Course, Part, Class, Sessions, Teacher, Student &max 1 &rooms &  1 & min 1 &room & yes & Assign a set of room to session in entry\\ \hline
%
%\multirow{2}{6.5em}{at\_most\_daily} & \multirow{2}{2cm}{Course, Part, Class, Teacher, Room, Student }& \multirow{2}{4em}{max 1} &count & 1 & 1 & slot &\multirow{2}{4em}{ yes} &\multirow{2}{4em}{ Limit a number of session in intervalle } \\ 
%\cline{4-7}
%  & & &first& 1& 1& slot & &    \\
%  \cline{4-7}
%  & & &last& 1& 1& slot & &    \\
%\hline
%at\_most\_weekly & Course, Part, Class, Teacher, Room, Student & max 1 & count &  1 & 1 & slot & yes & Limit a number of session in intervalle \\ \hline
%
%connected\_room &  Course, Part, Class, Teacher, Room, Student & max 1 & roomChain & min 1  & min 2 & room [ordered]  & yes & Session need connected rooms \\ \hline
%
%disjunctive\_group & Student & max 1 & \multicolumn{4}{|c|}{no} & yes & A group cant have overlap of 2 sessions  \\ \hline
%
%disjunctive\_room & Room   & max 1 & \multicolumn{4}{|c|}{no} & yes & A room cant host 2 sessions at same moment  \\ \hline
%
%disjunctive\_teacher & Teacher & max 1 & \multicolumn{4}{|c|}{no} & yes & A teacher cant gives  classes at same moment\\ \hline
%
%domain\_class\_group & Class & max 1 & \multicolumn{4}{|c|}{no} & no & A subset of group to classes (need solution)\\ \hline
%
%domain\_session\_teacher & Session & max 1 & \multicolumn{4}{|c|}{no} & no & A subset of teacher for sessions\\ \hline
%
%domain\_class\_room &Class & max 1 & \multicolumn{4}{|c|}{no} & no & A subset of room for class \\ \hline
%
%\multirow{2}{5.5em}{forbidden\_slot} & \multirow{2}{4em}{All} & \multirow{2}{4em}{max 1} & first & 1 & 1& slot & \multirow{2}{4em}{yes} & \multirow{2}{4em}{A session cant take slot in intervalle} \\
%  \cline{4-7}
%  & & &last& 1& 1& slot & &    \\
%\hline
%
%implicite\_sequenced\_sessions & Class & max 1 & \multicolumn{4}{|c|}{no} & no & All sessions in classes are sequenced\\ \hline
%
%\multirow{2}{9em}{not\_consecutive\_rooms} &\multirow{2}{2cm}{ Course, Part, Class, Teacher, Student} &\multirow{2}{4em}{ max 1} & minGap & 1 & 1 & slot & \multirow{2}{4em}{yes} & \multirow{2}{4em}{If 2 sessions have rooms in tuple then need a gap of mingap to walk from one the other} \\ 
%  \cline{4-7}
%  & & &rooms& 2 & min 1 & room,label & &   \\
%\hline
%
%part\_schedule & all & max 1 & \multicolumn{4}{|c|}{no}  & yes & we allowed time part value\\ \hline
%
%same\_daily\_slot & all & min 1 & \multicolumn{4}{|c|}{no} & yes & all slots of  selected sessions  are equal to the same daily slot \\ \hline
%
%same\_day  & all & min 1 & \multicolumn{4}{|c|}{no} & yes & all slots of  selected sessions  are equal to the same day \\ \hline
%
%same\_rooms & Course, Part, Class, Session, Teacher, Student  & min 1 & \multicolumn{4}{|c|}{no} & yes & all set rooms of  selected sessions  are equal \\ \hline
%
%same\_slots & all  & min 1 & \multicolumn{4}{|c|}{no} & yes & all slots of  selected sessions  are equal \\ \hline
%
%same\_teachers & Course, Part, Class, Session, Room, Student  & min 1 & \multicolumn{4}{|c|}{no} & yes & all set teachers of  selected sessions  are equal \\ \hline
%
%same\_week & all & min 1& \multicolumn{4}{|c|}{no} & yes & all slots of  selected sessions  are equal to the same week \\ \hline
%
%same\_weeklyday & all & min 1 & \multicolumn{4}{|c|}{no} & yes & all slots of  selected sessions  are equal to the same weekly day \\ \hline
%
%same\_weeklyslot & all & min 1 & \multicolumn{4}{|c|}{no} & yes & all slots of  selected sessions  are equal to the same weekly slot \\ \hline
%
%sequenced & Course, Part, Class, Session & min 1 & \multicolumn{4}{|c|}{no} & no & Sessions are ordered in the horizon slot (i.e i < j slot[session[i]] < slot[session[j]] \\ \hline
%
%teacher\_repartition & Class & min 2& class & min 2 & 1 & option  & no & repartition of teacher into a differentes classes of part \\ \hline
%
%weekly & Course, Part, Class, Session & min 1& \multicolumn{4}{|c|}{no} & no & A session tuple is weekly \\ \hline
%    \end{tabular}
%    \caption{Catalog of {\UTP} predicates}
%    \label{tab:catalog_constraint}
%\end{table}

Table \ref{tab:predicate_catalog} lists the predicates of the language
and indicates which are variadic or parametric.
The first predicates 
\texttt{\SAMEDAILYSLOT},
\ldots,
%\texttt{\SAMEWEEKDAY},
%\texttt{\SAMEWEEKLYSLOT},
%\texttt{\SAMEWEEK},
%\texttt{\SAMEDAY} and
\texttt{\SAMESLOT}
enforce common restrictions on the start times of the targeted sessions (e.g., sessions starting the same day).
Additionally,
any start time interval may be forbidden 
by passing its start and end points 
as parameters to 
predicate \texttt{\FORBIDDENPERIOD}.
Predicates \texttt{\ATMOSTDAILY}
and
\texttt{\ATMOSTWEEKLY}
upper-bound
the number of sessions
scheduled daily or weekly
within the given time interval.
\texttt{\SEQUENCED}
is a n-ary predicate ($n\geq2$)
which constrains
the latest session of the $i$-th e-map 
to end before
the earliest session of $i+1$-th e-map ($i=1..n-1$).
Predicate 
\texttt{\WEEKLY}
ensures sessions
are scheduled %weekly
on the same weekly slot
without presuming any particular sequencing.
Predicate
\texttt{\NOOVERLAP}
ensures sessions do not overlap in time
and is typically used to model disjunctive resources.
Predicate \texttt{\TRAVEL}
factors in any travel time
incurred between consecutive sessions
hosted in distant rooms.
The travel time matrix is a parameter of the predicate.
\texttt{\SAMEROOMS},
\texttt{\SAMESTUDENTS}
and
\texttt{\SAMETEACHERS}
require that sessions be assigned to the same set of rooms,
students or teachers.
Predicate 
\texttt{\ADJACENTROOMS}
require that sessions be hosted in 
adjacent rooms 
based on an adjacency graph passed as a parameter.
Lastly, 
predicate \texttt{\TEACHERDISTRIBUTION}
distributes the volumes of sessions represented by the different e-map arguments 
among different teachers. Teachers and session volumes are parameters of the predicate.
%\marc{Je pense qu'un exemple ici pourrait être intéressant ?
%"Two teachers may share a class where one does the first half of the sessions and the other one the last half."}

%\marc{Je remonterai le tableau pour qu'il apparaisse plus tôt ?}
