%%%%
%\newcolumntype{N}{>{\refstepcounter{rowcntr}\therowcntr}r}
\newcounter{rowcntr}[table]
\renewcommand{\therowcntr}{(\arabic{rowcntr})}
\setcounter{rowcntr}{0}


\begin{table*}[!ht]
\framebox[\textwidth][c]{%
\small
\begin{tabularx}{\textwidth}{>{\hsize=0.01\hsize\linewidth=\hsize}X>{\hsize=1.89\hsize\linewidth=\hsize}X>{\raggedleft\arraybackslash\hsize=.09\hsize\linewidth=\hsize}X}
%\begin{math}\STUDENT = \funcmzn{array\_union}(\xgroup) \end{math} & 
%partition\_set(\xgroup,\begin{math}\STUDENT\end{math}) & \refstepcounter{rowcntr} \therowcntr \label{cp:grouppartition}\\
%
%
%&$\forall(p \in \PART,k1,k2 \in \arraymzn{part\_classes}[p] \wmzn k1\gqmzn k2)$&\\
%
%\hspace{2cm}all_disjoint
%&\hspace*{2,8em}$(\xgroup[k1] \intermzn \xgroup[k2]=\{\})$& \refstepcounter{rowcntr} \therowcntr \label{cp:exclusiveclass}\\
%
%
%&$\forall(k1 \in \CLASS, k2 \in \funcmzn{class\_parents}(k1))(\xgroup[k1] \subseteq \xgroup[k2])$ &  \refstepcounter{rowcntr} \therowcntr \label{cp:parent}\\
%
%&$\forall(k \in \CLASS)(\arraymzn{maxsize}[k] \geqmzn \summzn(g \in \GROUP)$&\\
%&\hspace*{2,8em}$(\funcmzn{bool2int}(g \in \xgroup[k])*\summzn(u \in \STUDENT)(\funcmzn{bool2int}(\xstudent[u]=g)))$ &  \refstepcounter{rowcntr} \therowcntr \label{cp:classcapacity}\\
% 
%
%\hline
%
%
%%%%%%%%%%%%%%%%%%%%%%%%%%%%%%
%&Overlap sameDuration synchronous&\\
&no\_overlap($\SESSION'$)=  &\\
&\hspace*{2,8em}all\_different\_except0($\{\forall s \in \SESSION' \mid \var{\SESSION}{\SLOT}{s} * (e \in \var{\SESSION}{\ENTITY}{s})\} $)& \refstepcounter{rowcntr} \therowcntr \label{cp:allowedrooms-relaxed}\\
%
% 
%&cumulativeRoom($\SESSION'$)= $\forall r \in \ROOM,$&\\
&$\forall r \in \ROOM,$&\\
&\hspace*{2,8em}$\;$binPacking($\{0..|\SESSION|\}\cup\{\forall h \in \SLOT \mid 0..\roomcapacity{r}\},$&\\
&\hspace*{5em}$\{ \forall s \in \SESSION \mid(r \in \var{\SESSION}{\ROOM}{s})*\var{\SESSION}{\SLOT}{s}\},\{\forall s \in \SESSION \mid \classcapacity{\map{\SESSION}{\CLASS}{s}}\}	$)&   \refstepcounter{rowcntr} \therowcntr \label{cp:allowedteachers-relaxed} \\
%
%
%$\forall(k \in \CLASS)(((\arraymzn{part\_room\_use}[\funcmzn{class\_part}(k)]=\text{none}) \Longleftrightarrow (\xroom[k] = \{\})) $&\\
%&Group same size&\\
&$\forall h \in \SLOT,\;   $&\\
&\hspace*{2,8em}gcc($  \{\forall s \in\SESSION\mid \var{\SESSION}{\ROOM}{s} * (\var{\SESSION}{\SLOT}{s} = h)\},\{0\}\cup\{\forall r \in \ROOM \mid r\},\{nrCapacity\} $)
& \refstepcounter{rowcntr} \therowcntr \label{cp:multiroom-relaxed}\\
%
%&Group same size&\\
%&$\forall r \in \ROOM,\;   $&\\
%&\hspace*{2,8em}gcc($  \{\forall s \in\SESSION\mid \var{\SESSION}{\SLOT}{s} * (\var{\SESSION}{\ROOM}{s} = r)\},\{0\}\cup\{\forall h \in \SLOT \mid h\},\{nrCapacity\} $)
%& \refstepcounter{rowcntr} \therowcntr \label{cp:multiteacher-relaxed}\\
%
%
%& Cumulative by room type&\\
&$\forall l \in \LABEL' ,\;  Soit \SESSION'\subseteq\map{\LABEL}{\SESSION}{l} $&\\
&\hspace*{2,8em}cumulative($  \{\forall s \in\SESSION'\mid \var{\SESSION}{\SLOT}{s}\},\{\forall s \in \SESSION' \mid |\var{\SESSION}{\ROOM}{s}|\},|\{\forall r \in \ROOM \land l \in \map{\ROOM}{\LABEL}{r} \mid r\}| $)
& \refstepcounter{rowcntr} \therowcntr \label{cp:cumulativebytype}\\
%
%
\end{tabularx}%
}%
%}
\caption{
%Contraintes et prédicats du modèle \MINIZINC{}
Constraints and predicates relaxed for the \CP{} model.
}
\label{table:cp-contraintes-relaxed}
\end{table*}
%Dans le cas ou certaine features sont validées par l'instance nous pouvons réécrire voir simplifier certaines contraintes.
%
In cases where certain features are validated by the instance, we can rewrite or simplify constraints. 
%
%Dans le cas \hyperref[feat:nooverlap]{``no-overlap''} nous pouvons réécrire la contrainte cumulative~\ref{ctr:} pour la simplifier.
%
%Dans le cas full-exclusif (\hyperref[featmodel:exclusivesession]{``full-exclusive''}), on applique la contrainte disjonctive plutôt que cumulative en effet la capacité ne nécessite plus d'être prise en compte.
%
In the full-exclusive case (\hyperref[featmodel:exclusivesession]{``full-exclusive''}), we apply the disjunctive constraint instead of cumulative because capacity is no longer a concern. 
%On peut appliquer des contraintes de bin packing et all\_different dans le cas de no\_overlap (\hyperref[feat:nooverlap]{``no-overlap''}) et same\_duration. Ces constraintes globales offrent un meilleur filtrage et dans le cas \UTP{} l'ordre entre les séances impactant peu( ormis la contrainte sequenced et periodic) cela permet de mieux rentabiliser le filtrage de all\_different.
%
In the case of no\_overlap (\hyperref[feat:nooverlap]{``no-overlap''}) and same\_duration, we can apply bin\_packing and all\_different constraints. These global constraints provide better filtering, and in the case of \UTP{}, where the order between sessions has little impact (except for the sequenced and periodic constraints), this allows for better utilization of the filtering of all\_different. 
%
%Si nous avons des groupes et des salles de tailles modulaires, nous pouvons appliquer des contraintes de comptage (gcc) et cumulative~\ref{cp:cumulativebytype} qui, au lieu d'affecter une salle, affectent un type de salle par cours et vérifient qu'à un instant T, les capacités de type de salles ne sont pas dépassées.
%
If we have groups and modular-sized rooms, we can apply counting constraints (gcc) and cumulative constraints~\ref{cp:cumulativebytype} that, instead of assigning a room, assign a room type per course and ensure that at a given time, the capacities of room types are not exceeded.
%
%Ensuite il s'agit d'une affectation polynomial des salles pour les .
%https://link.springer.com/article/10.1007/s10601-018-9282-9