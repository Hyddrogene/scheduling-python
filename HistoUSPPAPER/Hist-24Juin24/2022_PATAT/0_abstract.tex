\begin{abstract}
We present a domain-specific modeling language for a %broad 
class of university timetabling problems ({\UTP}) 
that involve course scheduling, resource allocation and student sectioning.
%that reduce to hard %TODO hard?
%constraint satisfaction problems.
The {\UTP} language combines a formal domain model and a rules formalism to state constraints. %address common requirements. % relating to course scheduling, resource allocation and student sectioning. 
The model is based on 
a multi-scale schedule horizon (i.e., weeks, weekdays and daily slots),
a hierarchical course structure (i.e., course parts, part classes and class sessions),
and an extended set of resources (i.e., rooms, lecturers, students and student groups).
Student groups must be formed to populate classes and class sessions are to be scheduled individually and allocated single or multiple rooms and lecturers.
The model encodes sectioning constraints on classes, core scheduling constraints on sessions as well as compatibility, capacity and cardinality constraints on resource allocation. 
%and the default session-to-resource assignment policy is to consider all resources as cumulative.
Rules allow to state conjunctions of constraints on selected sets of entities and sessions using a catalog of timetabling predicates and a syntax to group, filter and bind entities and sessions.
%Rules rely on a  constraints on using timetabling predicates and %a comprehension syntax a selector syntax to target specific sessions, resources and course elements.
%Rules are built with timetabling predicates and filters to forge constraints on selected classes of resources or course elements. 
%{\UTP} 
As for implementation, the \UTP{} language is based on {\XML} 
and %the language implementation includes 
%an {\XML} schema % used to encode instances, 
%and 
comes with a tool chain that flattens rules into constraints and converts instances to solver-compatible 
%{\JSON}
formats. 
We present here the abstract syntax of the \UTP{} language and alternative constraint programming models developed in {\MINIZINC} and {\CHR} %to generate solutions. 
%%As an early proof of concept, we 
%and report early 
together with preliminary experiments on a real case study.
%instances modeling undergraduate curricula in a french university.

%Insert your abstract here. Include keywords, PACS and mathematical subject classification numbers as needed.
\keywords{University Timetabling \and Domain-Specific Modeling Language \and Constraint Programming \and Resource Scheduling}
% \PACS{PACS code1 \and PACS code2 \and more}
%\subclass{MSC code1 \and MSC code2 \and more}
\end{abstract}