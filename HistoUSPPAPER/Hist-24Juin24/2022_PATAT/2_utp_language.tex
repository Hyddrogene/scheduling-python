%------------------------------------------------------------
%------------------------------------------------------------
\section{University Timetabling Problem}
\label{sec:schema}
A \UTP{} instance is defined by an entity model and a rules set. %and a solution. % and a pre-assignment. 
%The entity model defines the schedule horizon, courses and resources of the instance and encodes core constraints relating to session scheduling, resource allocation, and student sectioning.
%Rules express additional constraints meant to capture stakeholder requirements on particular aspects of the problem.
%A rule is a conjunction of constraints built from a catalog of timetabling-specific predicates.
%The solution is a list of choices made for some or all of the decisions at stake (e.g., start time of a session).
%Note that a \UTP{} instance may have no rules.% rules set and solution components may be omitted. 
% Besides, the listed solution is not required to be consistent with the constraints enforced by the entity model or the rules set.
% This allows to tackle subproblems using separate instances and to support timetable generation or repair tasks.
% Each rule is an intensional representation of a collection of constraints applying to different entities.
% {\UTP} instances may thus be compiled to lower-level representations that explicitly declare all constraints and whose format is compatible with {\CP} languages.
%{\UTP} instances may therefore be compiled into lower-level representations that lists all constraints and whose format is tailored to back-end solvers.
%
A solution to a \UTP{} instance is a list of choices made for all the decisions at stake %(e.g., start time of a session).
that satisfies the core constraints of the entity model and the constraints expressed by the rules.
%Note that a \UTP{} instance may have no rules and an input solution may be provided based on context.
%We defined \acronym{XUTP}, an embedded version of the \UTP{} language into \XML{}.
%We do not provide here the detailed \XML{} specification of \acronym{XUTP} (see~\cite{uspSite}).
We provide in this section an informal description and set-theoretic semantics for the \UTP{} language components, namely the entity model (Section~\ref{sec:entity-model}), constraints (Section~\ref{sec:constraints}), rules (Section~\ref{sec:rules}) and solution (Section~\ref{sec:solution}).
Section~\ref{sec:related-work} draws a comparison between the \UTP{} language and the \ITC{} schema.


%
%We present theses components in turn, 
%discuss the type of features and requirements that may be factored in,
%and provide set-theoretic semantics for the rules language and flattening process.
%The reader is referred to \cite{uspSite} for the detailed {\XML} specification of {\XUTP} and the {\JSON/\DZN} instance formats.
%\cite{uspSite} also provides access to the {\MINIZINC} and {\CHR} models, the tool suite, and a benchmark of instances.
%motivate design choices,
%whose {\CP} implementation is addressed in Section~\ref{sec:model}.

%motivate design choices, 
%and highlight differences with related work.

%------------------------------------------------------------
%------------------------------------------------------------
\subsection{Entity model}
\label{sec:entity-model}
The entity model of a {\UTP} instance defines its schedule horizon, course structure and resources, as well as properties of entities and relational maps (see 
Figure~\ref{fig:utp-entity-model} for a sketch of the meta-model and Figure~\ref{fig:utp-rule-1} for a toy example).  
First, the entity model uses a time grid that decomposes into weeks, weekdays and daily slots. %, the number of which is instance-specific. 
Weeks share the same weekdays and weekdays the same daily slots. The latter make up 24 hours and have the same duration. %measured in minutes. 
Note that neither successive weeks nor successive weekdays are assumed to be consecutive. %and that Monday is the first weekday by convention. 
The schedule horizon is implicitly defined by the series of time slots mapping to week, weekday and daily slot combinations. Slots hence serve as time points to represent start and end times of course sessions and to measure session duration, travel time and any gap between sessions.

% % For one-column wide figures use
% \begin{figure}[h]
% % \includegraphics{utp-time-grid.eps}
% \caption{{\UTP} 3-layered time grid}
% \label{fig:utp-time-grid}
% \end{figure}

Courses have a tree-structure wherein each course (e.g., Algorithms) decomposes into parts (e.g., Lecture and Lab), parts into classes (e.g., lecture classes A and B), and classes into sessions (e.g., sessions 1 to 10 for each lecture class). Class sessions are the elementary tasks to schedule when solving a {\UTP} instance and the model fixes their number, duration and sequencing. First, the classes of a course part are decomposed into an identical number of sessions of equal duration, both constants being part-specific. Although this approach forbids classes using different session durations in a course part, 
%it %provides flexibility for handling sessions independently wrt. scheduling and resource allocation.
%Fixed decompositions also 
it is paramount to capture requirements that rely on clear-cut sessions (e.g., starting lab classes after 2 lecture sessions, synchronizing the 5th sessions of the lab classes for a joint examination). Second, the sessions of a class are ranked in the model and must be sequenced accordingly in any solution (session 1 before session 2 \ldots). Note that sessions are considered uninterruptible and, in particular, may not overlap two days. 

\begin{figure}[ht]
\includegraphics[scale=0.35]{img/utp_entity_model.png}
\caption{Entity meta-model.}
\label{fig:utp-entity-model}
\end{figure}

{\UTP} resources fall into 4 types, namely, rooms, lecturers, students and (student) groups.
All the resources of an instance, except groups (see Section~\ref{sec:solution}), are declared and typed in the entity model. In practice, upstream processes and decisions %constrain the resourcing and timing of courses.
%Basic restrictions come in the form of compatibility constraints that list 
determine the suitable rooms, eligible lecturers, candidate students and allowed times for the different courses (e.g., faculties prescribing degree-specific time grids, departments implementing room pooling policies and naming lecturers for courses, students registering to courses). %Such constraints are built in the entity model but scoped differently depending on resource types. %Specifically, 
These compatibility constraints are modeled by associating sets of possible start times, rooms and lecturers to each course part and a set of registered students to each course. Each session then inherits the sets of allowed resources from the course part and the course it belongs to.
%and which are implicitly the sessions.
%Student registrations are listed separately and %the possible students for a session are those registered to the course it sits in.
%any student registered to a course is considered a possible candidate for each of its sessions.

The entity model also encodes flow constraints that govern the distribution of resources over courses based on student registrations and capacity planning decisions (e.g., workload distribution between lecturers). First, each lecturer is allocated a fixed number of sessions in each course part he is eligible for, leaving lecturer-to-session assignment decisions to solvers. Second, each room allowed in a course part may be freely allocated to any session of the part (possibly none) but the model provides the flexibility to mark a room as mandatory in which case it will host or co-host all the sessions. As for students, the sectioning policy is implicit and complies with the course structure, i.e., each student must be assigned to a single class in each part of a course he has registered to and attend all sessions of these classes. In addition, the model supports group nesting constraints between classes to implement course-specific policies (e.g., aggregating student groups bottom-up from labs to lectures) or cross-course sectioning (e.g., imposing the same groups between classes of different courses of a curriculum).
 
Resource utilization is naturally subject to demand and capacity constraints.
Since modalities differ from one environment to the next, the language supports disjunctive and cumulative resources. The default policy is to consider all students, groups, lecturers and rooms as cumulative resources, i.e., they can attend, teach or host simultaneous sessions. Note though that rules may be stated to make some resources fully disjunctive or to prevent specific sessions from overlapping. Support for cumulative resources is paramount to address flexible attendance requirements (e.g., students assigned optional tutoring sessions that may overlap with compulsory courses) or to handle multi-class events (e.g., rooms hosting several classes for an exam or a conference). The model imposes no limits on the number of parallel sessions lecturers and students may attend. Rooms however may only host class sessions whose cumulated headcount is within their capacity. Upper bounds on room capacity and class size are encoded for all rooms and classes and the model also allows uncapacitated rooms to cater for the case of virtual rooms.

The language also supports sessions using multiple resources of the same type. %at any point in time.
The need for multiple rooms or lecturers arises in practical situations (e.g., multi-room sessions for hybrid teaching, joint supervision of practical work sessions, exams requiring several monitors). %and the number of resources required per session is course part dependent.
%such restrictions are expressed in the entity model through cardinality constraints. %that are lifted to course parts. 
To this end, the model associates to each course part the number of lecturers required per session 
and indicates whether the sessions are single- or multi-rooms.
Note that sessions without lecturers or rooms are allowed (e.g., unsupervised student project sessions).
The model enforces specific constraints to handle multi-room sessions which override the default room allocation policy. Specifically, students attending the session may be freely dispatched in rooms irrespectively of the group structure, the cumulated capacity of the allocated rooms is taken into account for hosting, uncapacitated rooms cannot be allocated, and 
the allocated rooms are considered disjunctive for the time of the session.
%a {\UTP} instance may freely mix single- and multi-resource sessions as well as disjunctive and cumulative resources. 

Note finally that the language provides users with the ability to label resources and course elements to define their own classes of entities (e.g., teams of lecturers, blocks of rooms). Labels together with built-in entity types and identifiers are used to filter entities and to scope rules appropriately.
 


We formalize below the entity model and introduce notations that will be used thereafter.
Let ${\ENTITY}$ denote the set of entities
and ${\SESSION}$ the set of sessions.
${\ENTITY}$ is partitioned into   
a set of courses ${\COURSE}$, 
a set of course parts ${\PART}$, 
a set of classes ${\CLASS}$, 
a set of students ${\STUDENT}$, 
a set of lecturers ${\TEACHER}$,
a set of rooms ${\ROOM}$,
and the singleton domain of courses ${\COURSES}$ 
(${\COURSES}=\myset{\COURSE}$). 
Let 
%${\COURSES}$ denote the course domain, 
%${\COURSE}$ the set of courses (${\COURSES}=\myset{\COURSE}$), 
%${\PART}$ the set of course parts, 
%${\CLASS}$ the set of classes, 
%${\SESSION}$ the set of sessions, 
%${\STUDENT}$ the set of students, 
%${\TEACHER}$ the set of teachers, 
%and 
%${\ROOM}$ the set of rooms.
$
{\TYPE}
=
\myset{
{\COURSES}, 
{\COURSE},
{\PART},
{\CLASS},
{\STUDENT},
{\TEACHER},
{\ROOM}
}
$
denote the set of entity types
(${\ENTITY}=\setunion{X}{\TYPE}{X}$)
and 
$
{\prec}
=
\myset{
({\COURSES},{\COURSE}),
({\COURSE},{\PART}),
({\PART},{\CLASS}),
({\CLASS},{\SESSION}),
({\STUDENT},{\COURSE}),
$
$
({\TEACHER},{\PART}),
({\ROOM},{\PART})
}
$
denote the relation over 
${\TYPE}\cup\myset{\SESSION}$ 
that models the course hierarchy
and the distribution of resource types over course components.

${\prec^{*}}$
%${\preceq^{*}}$
denotes the transitive %and reflexive 
closure of
${\prec}$ 
over
${\TYPE}\cup\myset{\SESSION}$
and
${\maptype{X}{Y}}:X\rightarrow2^{Y}$
denotes the function mapping each element of $X$ to its set of compatible elements in $Y$
for each pair %$(X,Y)$ such that 
%$X{\preceq^{*}}Y$.
$X{\prec^{*}}Y$.
For instance, 
${\maptype{\ROOM}{\PART}}$ 
represents the distribution of rooms over course parts, 
${\maptype{\PART}{\CLASS}}$ 
the decomposition of course parts into classes,
${\maptype{\CLASS}{\SESSION}}$ 
the decomposition of classes into sessions,
and ${\maptype{\ROOM}{\SESSION}}$ 
the inferred distribution of rooms over sessions.
The functions corresponding to the pairs of $\prec$
are directly encoded in the entity model
and the remaining functions are defined inductively using recursive aggregation. 

We shall denote by ${\map{X}{Y}{i}}$ the image of entity $i$ of type $X$ over $2^Y$ %, i.e., the set of elements of type $Y$ compatible with $i$. 
and by ${\maptype{Y}{X}}$ the inverse of ${\maptype{X}{Y}}$.
Equation (\ref{model:hierarchy}) below models the hierarchical decomposition of course elements\footnote{$\sqcup$ denotes the disjoint union operation, i.e. set union over pairwise disjoint sets.},
Equation (\ref{model:transitivity}) is the closure rule over 
%$\preceq^{*}$. 
$\prec^{*}$,
%Note that each map $\maptype{\SESSION}{X}$ is the inverse of map $\maptype{X}{\SESSION}$.
and Equation (\ref{model:inverse}) models inverse maps.

\begin{align}
%
\forall (X,Y) \in 
\myset{
({\COURSES},{\COURSE}),
({\COURSE},{\PART}),
({\PART},{\CLASS}),
({\CLASS},{\SESSION})
}:
Y=
\setpartition{i}{X}{\map{X}{Y}{i}} 
\label{model:hierarchy}
\\
%
\forall X,Y,Z \in {\TYPE}\cup\myset{\SESSION}:
X\preceq^{*} Y\preceq^{*} Z 
\Rightarrow 
(\forall i \in X:
\map{X}{Z}{i}=\setpartition{j}{\map{X}{Y}{i}}{\map{Y}{Z}{j}}
\label{model:transitivity})
\\
%
\forall X,Y \in {\TYPE}:
X\preceq^{*} Y 
\Rightarrow 
(\forall i \in X, j \in Y:
j \in \map{X}{Y}{i} \Leftrightarrow i \in \map{Y}{X}{j}
)
\label{model:inverse}
%\\
%%
%\forall X \in {\TYPE}\cup\myset{\SESSION},
%i \in X:
%\domarg{X}{X}{i} = \myset{i} \label{model:selfmap}
%%
\end{align}



\begin{table}[ht]
\begin{center}
\begin{tabular}{|rl|}
\hline
$(\WEEK,\WEEKDAY,\DAILYSLOT)$               & the number of weeks $\WEEK$, weekdays $\WEEKDAY$ and daily slots $\DAILYSLOT$
\\
$\SLOT$                                     & the time slots 
\\\hline
$\ENTITY$                                   & the entities
\\
$\COURSES\subseteq\ENTITY$                  & the course domain
\\
$\COURSE\subseteq\ENTITY$                   & the courses
\\
$\PART\subseteq\ENTITY$                     & the course parts
\\
$\CLASS\subseteq\ENTITY$                    & the classes
\\
$\ROOM\subseteq\ENTITY$                     & the rooms
\\
$\TEACHER\subseteq\ENTITY$                  & the lecturers
\\
$\STUDENT\subseteq\ENTITY$                  & the students
\\
$\map{X}{Y}{i}\subseteq{Y}$                 & the entities of type $Y$ associated with entity $i$ of type $X$
\\\hline
${\LABEL}\subseteq2^{{\ENTITY}}$            & the labels
\\\hline
$\GROUP\subseteq{2^{\STUDENT}}$             & the groups of students
\\\hline
$\SESSION$                                  & the sessions
\\
$\map{X}{\SESSION}{i}\subseteq{\SESSION}$   & the sessions compatible with entity $i$ of type $X$
\\
$\map{\SESSION}{X}{s}\subseteq{X}$          & the entities of type $X$ compatible with session $s$
%\\
%$\disjunctiverooms\subseteq{\ROOM}$         & the set of disjunctive rooms
\\
$\partallowedslots{s}\subseteq{\SLOT}$      & the start times allowed for session $s$
\\
$\sessionduration{s}\in{\SLOT}$             & the duration of session $s$
\\
$\sessionrank{s}\in{\NATURAL^*}$            & the rank of session $s$ in its class
\\
$\sessionranked\subseteq{\SESSION\times\SESSION}$ & the pairs of sessions with consecutive ranks in a class
\\\hline
$\classparents{k}\subseteq{\CLASS}$          & the parent classes of class $k$ if any
\\
$\classcapacity{k}\in{\NATURAL}$            & the maximum size of class $k$
\\\hline
$\roomcapacity{r}\in{\NATURAL}$             & the capacity of room $r$
\\
$\virtualroom{r}\in{\BOOLEAN}$              & whether room $r$ is virtual or not
\\
$\virtualrooms\subseteq{\ROOM}$             & the virtual rooms
\\\hline
$\multiroompart{p}\in{\BOOLEAN}$            & whether course part $p$ is multi-room or not
\\
$\multiroomparts\subseteq{\PART}$           & the multi-room parts
\\
$\mandatoryrooms{p}\subseteq{\ROOM}$        & the mandatory rooms of part $p$
\\
$\partteachermultiplicity{p}\in{\NATURAL}$  & the number of lecturers required by every session of part $p$
\\
$\partteacherservice{l,p}\in{\NATURAL}$     & the number of sessions required by lecturer $l$ in part $p$
\\
\hline
\end{tabular}
\caption{Entity model: constants, sets, maps and relations.}
\label{table:model-maps}
\end{center}
\end{table}


Table~\ref{table:model-maps} provides the full list of constants, sets, properties and relational maps encoded in the entity model.\footnote{
The following rules apply. $\SLOT=\myset{i.\WEEKDAY.\DAILYSLOT+j.\DAILYSLOT+k\ |\ 0\leq i<\WEEK,0\leq j<\WEEKDAY,1\leq k\leq\DAILYSLOT}$.
For each class $k$ in part $p$,
$\myset{\sessionrank{s}\ |\ s\in\map{\CLASS}{\SESSION}{k}}=\myset{1,\ldots,\mycard{\map{\CLASS}{\SESSION}{k}}}$, 
%$\mycard{\classparents{k}}\leq1$ 
and $\classparents{k}\not\subset\map{\PART}{\CLASS}{p}$.
For each pair of sessions $s,s'$, 
$(s,s')\in\sessionranked$ iff $\map{\SESSION}{\CLASS}{s}=\map{\SESSION}{\CLASS}{s'}$ and $\sessionrank{s'}=\sessionrank{s}+1$.
For each course part $p$,
%$p\in\multiroomparts$ iff $\multiroompart{p}$; 
%and 
$\partteachermultiplicity{p}.\mycard{\map{\PART}{\SESSION}{p}}=\sum\limits_{l\in\map{\PART}{\TEACHER}{p}}{\partteacherservice{l,p}}$.}

%We shall denote by
%${\RANK}$
%the range of session ranks,
%${\maptype{\RANK}{\SESSION}}:\RANK\rightarrow2{^\SESSION}$
%the rank-based partitioning of sessions,
%and
%${\LABEL}$
%the set of labels 
%(${\LABEL}\subseteq2^{{\ENTITY}}$)
%completed 
%with the whole set of entities %to mock label optionality
%($\ENTITY\in{\LABEL}$)
%and singleton entities %to support identity-based selection
%($\myset{\myset{e}\ |\ e\in{\ENTITY}}\subseteq{\LABEL}$).
%As discussed in section~\ref{sec:rules},
%labels are optional filters used in rules to select entities
%hence the formal inclusion of $\ENTITY$ in ${\LABEL}$ to mock label optionality.
%Likewise, entity identifiers are used as an alternative to labels
%hence the inclusion of singleton entities in ${\LABEL}$.

%------------------------------------------------------------
%------------------------------------------------------------
\subsection{Predicates and constraints}
\label{sec:constraints}
{\UTP} constraints apply to pairs, called e-maps, which associate an entity with a non-empty subset of its compatible sessions.
%which we call e-maps. 
Constraints are built with predicates whose signature includes e-map variables%ranging over the set of e-maps
, the number of which is referred to as the arity of the predicate. 
Note that some predicates may also accept parameters.
Let 
${\EMAP}=
\setunion{X}{\TYPE}
\myset{(e,S')\ |\ e\in X,S'\subseteq\map{X}{\SESSION}{e}\wedge S'\neq\emptyset}$
denote the set of e-maps,
a {\UTP} constraint has the form
\begin{align}
c((e_1,S_1),\ldots,(e_m,S_m),p_1,\ldots,p_n) \label{rule:constraint}
\end{align}
where 
$c$ is a predicate symbol of arity $m$,
$(e_1,S_1),\ldots,(e_m,S_m)$ are e-maps ($(e_i,S_i)\in{\EMAP}$, $i=1\ldots m$) 
and 
$p_1,\ldots,p_n$ are values for the parameters of $c$ ($n\geq0$).
Three constraints (\ref{constraint-example-1}, \ref{constraint-example-2}, \ref{constraint-example-3}) are illustrated in Figure~\ref{fig:utp-rule-1}.


Every predicate may be used indistinctly with e-maps defined on course elements or on resources.
E-maps defined on resources are interpreted as conditional session-to-resource assignments
when checking constraints 
whereas e-maps defined on course elements are unconditional assignments since they model constitutive sessions.
In other words, 
a constraint is only evaluated
on the sessions for which its e-map arguments and the considered solution propose the same entity assignment.\footnote{Formally, let $\var{E}{\SESSION}{e}$ be the variable denoting the set of sessions assigned to entity $e$ and $S'_1,\ldots,S'_m$ be sets of sessions, the conditionality of a constraint $c$ is stated as follows: 
$(\var{E}{\SESSION}{e_1}=S'_1 \wedge\ldots\wedge\var{E}{\SESSION}{e_m}=S'_m)
\Rightarrow
(c((e_1,S_1),\ldots,(e_m,S_m),p_1,\ldots,p_n)
\Leftrightarrow
c((e_1,S_1\cap S'_1),\ldots,(e_m,S_m\cap S'_m),p_1,\ldots,p_n))$.}

It follows that 
a constraint is evaluated on every session that is mapped to a course element by one of its e-map arguments.
Constraints that apply exclusively to course elements are therefore unconditional. 
Note also that the use of e-maps that model the whole set of sessions compatible with an entity 
will necessarily constrain any session that may be assigned to this entity.


%Every predicate may be used indistinctly with e-maps defined on course elements or on resources which we call c-maps and r-maps, respectively. R-maps are interpreted conditionally since they map a resource to some of its possible sessions whereas c-maps model unconditional assignments since they model constitutive sessions of course elements. In other words, a constraint must be evaluated on every session of every c-map in its scope but only on the sessions of its r-maps whose resource assignment is compatible with the proposed solution.\footnote{Formally, let $\var{E}{\SESSION}{e}$ be the variable denoting the set of sessions assigned to entity $e$ and $S'_1,\ldots,S'_m$ be sets of sessions, the conditionality of a constraint $c$ is stated as follows:  $(\var{E}{\SESSION}{e_1}=S'_1 \wedge \var{E}{\SESSION}{e_m}=S'_m) \Rightarrow (c((e_1,S_1),\ldots,(e_m,S_m),p_1,\ldots,p_n) \Leftrightarrow c((e_1,S_1\cap S'_1),\ldots,(e_m,S_m\cap S'_m),p_1,\ldots,p_n))$.}
%effectively assigns to the resource of the r-map.
%(see Rule (\ref{rule:conditionality})).
%It follows that constraints applying exclusively to c-maps are unconditional. Besides, the scoping of e-maps that model the whole set of sessions compatible with an entity will constrain every session assigned to a resource or constitutive of a course element.
%%The rule below %(\ref{rule:conditionality}) models the conditionality of constraints.

%{\footnotesize{
%\begin{multline}
%\forall S'_1,\ldots,S'_m\in{\SESSION}:
%(\var{E}{\SESSION}{e_1}=S'_1 \wedge \var{E}{\SESSION}{e_m}=S'_m)
%\Rightarrow\\
%(c((e_1,S_1),\ldots,(e_m,S_m),p_1,\ldots,p_n)
%\Leftrightarrow
%c((e_1,S_1\cap S'_1),\ldots,(e_m,S_m\cap S'_m),p_1,\ldots,p_n))
%\label{rule:conditionality}
%\end{multline}
%}}

\input{2_2_utp_predicates}

Table \ref{tab:predicate_catalog} lists the predicates of the language
and indicates which are variadic or parametric.
The first predicates 
\texttt{\SAMEDAILYSLOT},
\ldots,
%\texttt{\SAMEWEEKDAY},
%\texttt{\SAMEWEEKLYSLOT},
%\texttt{\SAMEWEEK},
%\texttt{\SAMEDAY} and
\texttt{\SAMESLOT}
enforce common restrictions on the start times of the targeted sessions (e.g., sessions starting the same day).
Additionally,
any start time interval may be forbidden 
by passing its start and end points 
as parameters to 
predicate \texttt{\FORBIDDENPERIOD}.
Predicates \texttt{\ATMOSTDAILY}
and
\texttt{\ATMOSTWEEKLY}
upper-bound
the number of sessions
scheduled daily or weekly
within the given time interval.
\texttt{\SEQUENCED}
is a n-ary predicate ($n\geq2$)
which constrains
the latest session of the $i$-th e-map 
to end before
the earliest session of $i+1$-th e-map ($i=1..n-1$).
Predicate 
\texttt{\WEEKLY}
ensures sessions
are scheduled weekly
without presuming any particular sequencing.
Predicate
\texttt{\NOOVERLAP}
ensures sessions do not overlap in time
and is typically used to model disjunctive resources.
Predicate \texttt{\TRAVEL}
factors in any travel time
incurred between consecutive sessions
hosted in distant rooms.
The travel time matrix is a parameter of the predicate.
\texttt{\SAMEROOMS},
\texttt{\SAMESTUDENTS}
and
\texttt{\SAMETEACHERS}
require that sessions be assigned to the same set of rooms,
students or lecturers.
Predicate 
\texttt{\ADJACENTROOMS}
require that sessions be hosted in 
adjacent rooms 
based on an adjacency graph passed as a parameter.
Lastly, 
predicate \texttt{\TEACHERDISTRIBUTION}
distributes the volumes of sessions represented by the different e-map arguments 
among different lecturers. Lecturers and session volumes are parameters of the predicate.


\input{2_3_utp_rules}
\input{2_4_utp_solution}
%------------------------------------------------------------
%------------------------------------------------------------
\subsection{Related work}
\label{sec:related-work}

%Examination timetabling {1999schaerfAIR}: marginal differences with CTP => mandatory attendance for students, different workload constraints for students, can have more than one exam in a room (cumulative room), TAP {2022caselliESWA}: post-allocation of "tutors" to "workshops". BACP {2001castroARXIV,2011chiarandiniJH}: close to maquettage Student sectioning: {2010mullerAOR} => initial vs batch vs online. {2019schindlAOR} CB-CTTP {2010mccollumINFORMS,2012bettinelliAOR,2015bettinelliTOP}: week schedule  with time periods (no session duration), no modeling of student enrolments, hard: same-curriculum/teacher (disjunctive teacher/students), room occupancy, no-overlap(lectures of a course), teacher unavailabilities, soft: room capacity, room stability, ... PE-CTTP {2007lewisITC,2010mccollumINFORMS}: week schedule (45 slots) with time periods (no session duration), student sectioning factored in, each course is a single event (no series of class sessions), students and rooms disjunctive,  New/subsuming models - ITC-2019 {2018mullerPATAT}: - UCTP (survey {2021chenIEEEA}) - {2022zaulirMJFAS,2014aizamNACO} => essentially "new" workload/pattern constraints but probably encodable in ITC. General survey: {2019oudeAOR}:


We highlight here the main differences between the {\UTP} language and the {\ITC} language ({\ITC} for short).

%the two approaches use the same temporal representation but {\ITC} leaves the possibility to configure the granularity of daily slots in each instance (${\DAILYSLOT\in\myset{1\ldots 1440}}$)  while it is set to the minute in {\UTP}. Nevertheless, any granularity may be used in {\UTP} by filtering the series of allowed slots in course parts and by re-scaling session duration and travel time data.
A first difference between the two frameworks lies in the representation of the possible times a class can meet.
In {\UTP}, a class is defined by a single sequence of sessions of equal duration and the problem is to schedule each session.
In {\ITC}, a class is given alternative fixed session schedules ({\texttt{times}} elements in the {\XML} schema) and the problem is to choose one of the schedules for the class.
A schedule is the repetition over a set of weeks of one or more sessions that have the same duration and start on specific days of the week at the same predefined time (daily slot).
The two representations are not reducible to one another.
For instance, alternative schedules using different session durations cannot be modeled in {\UTP}.
Conversely, class schedules where sessions do not necessarily start on the same daily slot cannot be modeled in {\ITC}.
Nevertheless, basic class schedules may be represented in either approach by stating {\ITC} constraints or {\UTP} rules on classes.
For instance, a class meeting every week on the same day and the same daily slot, both being subject to time restrictions, may be modeled using \texttt{\SAMEDAILYSLOT}, \texttt{\WEEKLY} and \texttt{\FORBIDDENPERIOD} constraints.
The implementation of a more comprehensive reduction method %for alternative schedules 
%(based, for instance, on a dedicated {\UTP} predicate) 
will be the subject of future work.

%Another difference lies in the representation of resources and the constraints governing their distribution and allocation.
%On the one hand, 
Second, 
{\ITC} represents alternative course configurations
by introducing an intermediate layer in the course hierarchy 
that sits between courses and parts.
The configurations of a course typically differ in their number of (sub)parts
and are mutually exclusive from a student sectioning standpoint, that is, 
a registered student must be assigned a single configuration and attend all of its parts.
This feature is not currently supported in {\UTP}.
As for resources, {\UTP} explicitly represents lecturers on par with rooms %and allows to specify their workload in each course part %(i.e., the number of sessions to teach)
whereas {\ITC} only models rooms.
{\UTP} also provides the flexibility to allocate different resources within a class 
(and specify lecturer workload in particular) %(i.e., the number of sessions to teach)
whereas the same room must be allocated in {\ITC}.
Additionally, {\UTP} supports multi-resource sessions whereas {\ITC} is restricted to single-room sessions.

Lastly, the two constraint languages present important differences.
While {\ITC} constraint predicates apply to classes,
{\UTP} predicates apply to any set(s) of sessions
and may be used in particular on individual sessions, hence granting finer-grained control.
Besides, {\UTP} rules and the selector language allows to constrain any class of resources or course elements in a concise way.

Lastly, the {\ITC} schema addresses the timetabling problem as a combinatorial optimization problem.
It includes a cost function weighting 4 criteria which respectively penalize the choice of sessions and rooms for the classes, the violations of constraints and the overlapping of sessions per student.
In its current version, the {\UTP} language addresses the problem as a hard constraint satisfaction problem.
The integration of soft constraints and the possibility of aggregating penalties or preferences, either in solution generation or repair contexts, is under investigation.

%Student groups 
% near-identical course structure: no course configuration element
% no time elements in UTP classes. Alternative class times are fixed in ITC
% OK time elements in extension -> unusitable for loosely constrained class programs
% OK impossible to express k weekly slots with different starting slots. In UTP: use sameWeeklySlot with masks.
% OK impossible to enforce different constraints bettwen first period of a course (amorcage) and the rest => ok for us with masks
% KO: times with different #sessions and session length. 
% => solution per session (vs per class) : slot + romms + teachers


%Sectioning:: as itc (parent class)
%Ressources
%- rooms: travel in ITC => using constraint travel in UTP.
%- new: teachers.
%- students: no change.
%- OK groups. Admin and computational needs.
%- Domain constraints
%- by default, all resources are cumultative (explain). Avec contrainte (disjunctive):: no overlap on romms/etc. 
%- allowed slot (applies to all sessions of a part's classes) : ITC via time elements. Adequate for "grid systems". 
%- allowed rooms and allowed teachers (worklaod per part = prescribed number of sessions)
%- single or multi-room/teacher session in UTP.
%- possibly different resources between sessions of a class (unless addiiotnal rules:: sameRoom, sameTeacher, ...)
%Rules language
%- ITC: class-level constraints vs session-level constraints on UTP
%- Labelling: simplifies constraint expression to look up entities
%Solution
%seul ajout: group-to-class and student-to-group
%Résolution: UTP == SAT vs Opt/gestion prefs => pas de priorites, pondérations, etc
%

