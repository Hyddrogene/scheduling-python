%------------------------------------------------------------
%------------------------------------------------------------
\subsection{Reformulation}
\label{sec:model-reformulation}

The model presented above is generic and may be adapted on a per instance basis depending on the features and rules at stake.
We discuss here a few variants of the {\UTP} problem which provide opportunities for model reformulation.

When instances only involve single-room sessions ($\multiroomparts=\emptyset$), %\vee|\map{\PART}{\ROOM}{p}|=1\ (p\in{\PART})$).
one may adopt integer or enumerated room allocation variables
instead of set variables $\var{\SESSION}{\ROOM}{s}$ and rewrite constraints accordingly. 
In the same way, lecturer assignment variables and constraints may be adapted when a single lecturer is required per session.
Note that hybrid models mixing single or multi-resource session variables may be considered too.
The temporal model may also be simplified 
when the time grid 
%(i.e., the complete set of start times allowed for sessions) 
is coarse-grain and guarantees no session
can span over consecutive start times
($\forall s\in\SESSION, \sessionduration{s}\leq\min(\myset{h'-h\ |\  h,h'\in\mape{\PART}{\SLOT}{\PART}\wedge h<h'}$).
This situation occurs in institutions that impose a common time grid to ensure sessions (with any travel time incurred) necessarily fit in each time slot.
If so, sessions may be handled as time points rather than time intervals
and temporal predicates and constraints may be adapted. 
%, e.g., Constraint~(\ref{ctr:differentsessions}) can be substituted to Constraint~(\ref{ctr:disjointsessions}).
%Note that worst-case travel time scenarios may have to be factored in if \texttt{\TRAVEL} constraints are stated.
Capacity constraints may also be simplified for disjunctive rooms. 
A room is disjunctive if a \texttt{\NOOVERLAP} constraint is stated on the whole set of its compatible sessions
($r\in\disjunctiverooms\leftrightarrow{no\_overlap(r,\map{\ROOM}{\SESSION}{r})}$ where $\disjunctiverooms\subseteq{\ROOM}$ denotes the set of disjunctive rooms).
If so, the default cumulative constraint (\ref{ctr:cumulativeroomcapacity}) may be overridden by Constraint~(\ref{ctr:disjunctiveroomcapacity}).
%\davidl{symmetry constraints: groups, interchangeable/substitutable resources}




\begin{flalign}
%&Let\ s,s'\in\SESSION:
%\disjoint{s}{s'}
%\leftrightarrow
%\var{\SESSION}{\SLOT}{s} \neq\var{\SESSION}{\SLOT}{s'}
%&\label{ctr:differentsessions}
%\\
&\forall r\in\disjunctiverooms:
\bigwedge_{\substack{h\in\SLOT\\k\in\map{\ROOM}{\CLASS}{r}}}
\roomcapacity{r}
\geq
\max_{\substack{s\in\map{\CLASS}{\SESSION}{k}}}
{\roomuse{r}{k}{s}{h}}
&
\label{ctr:disjunctiveroomcapacity}
\end{flalign}
