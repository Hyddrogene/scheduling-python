%------------------------------------------------------------
%------------------------------------------------------------
\section{Constraint Programming Implementation}
\label{sec:cp-model}

\setcounter{equation}{0}
%Nous présentons dans cette section deux modèles d'instances {\UTP} développés en \MINIZINC{} et \CHR{}.
%Ces modèles mettent en jeu des contraintes relatives
%au partitionnement des étudiants en groupes et à l'attribution des groupes aux classes,
%à la distribution des ressources sur les séances,
%à l'ordonnancement des séances,
%et à l'allocation de leurs ressources.
%Nous présentons tout d'abord les données d'instance
%ainsi que les variables de décision qui sont communes aux deux modèles.
%%La table~\ref{table:cp-core-sets} liste les plages d'entiers identifiant les différents ensembles d'objets manipulés.
%%(p. ex., ensemble des séancesset of compatible sessions for rooms) and metric properties (e.g., room capacity). 
%La table~\ref{table:cp-instance-data} liste les plages d'entiers identifiant les différents ensembles d'objets manipulés et définit les structures utilisées pour représenter les données d'instance.
%
In this section, we present two constraint-based models for \UTP{} instances developed in \MINIZINC{} and \CHR{}.
%The two models handle constraints relative to student sectioning and group assignment to classes, resource distribution over the sessions, session scheduling, and resource allocation.
%Table~\ref{table:cp-instance-data} lists the instance data shared by the two models.
%In the following, we first introduce the \MINIZINC{} model, then the \CHR{} model.
%\marc{ajouter un texte qui donne un exemple pour chacune des trois structures (part\_sessions faire le lien avec DPS) - supprimer la table}
The two models use the same arrays, functions and constants for representing input data. We do not list them here but they are easily understandable such as \texttt{part\_sessions} which gives the set of sessions constitutive of a part, \texttt{session\_rooms} which gives the set of allowed rooms for a session, \texttt{week} which gives the week of a slot, and \texttt{nr\_weekly\_slots} which is the number of slots in a week.

%\begin{table}[!ht]
%\resizebox{\textwidth}{!}{%
\centering
{\small
\begin{tabular}{|rl|}
%\hline
%\SLOT & ensemble des créneaux définissant l'horizon de temps \\
%%& (1..(nr\_daily\_slot $\times$ nr\_week\_day $\times$ nr\_week))\\ 
%\COURSE & ensemble des cours\\
%\PART & ensembles des parties de cours\\
%\CLASS & ensembles des classes\\
%\SESSION & ensembles des séances\\
%\ROOM & ensemble des salles\\
%\TEACHER & ensemble des enseignants\\
%\GROUP & ensemble des groupes d'étudiants\\
%\STUDENT & ensembles des étudiants\\
\hline
part\_sessions              & sessions of a part \\%séances d'une partie de cours\\
class\_sessions             & sessions of a class \\%séances d'une classe\\
part\_resources            & resources of a part\\%ressources allouables dans une partie de cours\\
part\_daily\_slots          & allowed dailyslots of a part\\%créneaux quotidiens autorisés pour une partie de cours\\
part\_weekdays              & allowed weekdays of a part\\%journées hebdomadaires autorisées pour une partie de cours\\
part\_weeks                 & allowed weeks of a part\\%semaines autorisées pour une partie de cours\\
part\_room\_use             & how rooms are used by a part (none, single, multiple)\\%régime d'utilisation des salles dans une partie de cours\\
part\_lecturer\_service      & number of sessions a lecturer has in a part\\%volume de séances par enseignant et partie de cours\\
room\_capacity              & maximum capacity of a room\\%capacité d'accueil d'une salle\\
class\_capacity             & maximum capacity of a class\\%effectif maximum d'une classe\\
class\_parent               & parent class of a class\\
%room\_max\_use              & \\
%teacher\_max\_use           & \\
%group\_max\_use             & \\
%\end{tabular}
%}
%\caption{Données d'instance}
%\label{table:input-data}
%\end{table*}
%
%\begin{table*}[h]
%%\resizebox{\textwidth}{!}{%
%\centering
%{\footnotesize
%\begin{tabular}{|rl|}
\hline
session\_course         & course of a session \\%cours d'une séance\\
session\_part           & part of a session \\%partie de cours d'une séance\\
session\_class          & class of a session \\%classe d'une séance\\
session\_rank           & rank of a session in its class \\%rang d'une séance dans sa classe\\
session\_length         & number of slots between the start and the end of a session \\%durée d'une séance\\
lecturer\_per\_session   & number of lecturers of a session \\%nombre d'enseignants d'une séance\\
room\_sessions          & set of possible sessions of a room \\%ensemble des séances possibles pour une salle\\
lecturer\_sessions       & set of possible sessions of a lecturer \\%ensemble des séances possibles pour un enseignant\\
group\_sessions         & set of possible sessions of a group \\%ensemble des séances possibles pour un groupe\\
class\_mandatory        & set of mandatory rooms of a class \\%ensemble de salles obligatoires pour une classe\\
slot\_dailyslot         & dailyslot of a slot \\%créneau quotidien caractérisant un créneau\\
slot\_weekday           & weekday of a slot \\%journée hebdomadaire caractérisant un créneau\\
slot\_week              & week of a slot \\%semaine caractérisant un créneau\\
\hline
is\_multi\_rooms        & is the session multi-room? \\%indique si une séance est multi-salles\\
has\_mandatory\_rooms   & has the session got mandatory rooms? \\%indique si une séance a des salles obligatoires\\
nr\_weekly\_slots       & number of slots in a week \\%nombre de créneaux dans une semaine\\
\hline
\end{tabular}
}
\caption{
%Données d'instances et fonctions utilitaires
Instance data shared by the \MINIZINC{} and \CHR{} models
}
\label{table:cp-instance-data}
\end{table}

%------------------------------------------------------------

%------------------------------------------------------------
%------------------------------------------------------------
\subsection{{\MINIZINC} model}
\label{sec:cp-mzn}
%------------------------------------------------------------
%------------------------------------------------------------
%------------------------------------------------------------

%{\MINIZINC} est un langage de modélisation haut-niveau de problèmes d'optimisation sous contraintes \cite{MZN,MINIZINC}. 
%Les modèles {\MINIZINC} sont traduits dans le langage cible {\FLATZINC} \cite{FLATZINC} qui permet d'interfacer différents types de solveurs dont les solveurs de programmation par contraintes sur domaines finis tels  {\GECODE} \cite{GECODE}.
%{\MINIZINC} intègre de nombreuses contraintes globales et le modèle {\UTP} présenté en Table~\ref{table:mzn-contraintes} et utilisant les variables de décision présentée en Table~\ref{table:cp-variables} s'appuie sur quelques contraintes dédiées aux problèmes d'ordonnancement.
%
\MINIZINC{} is a high-level language to model constrained optimization problems \cite{2007_nethercote_SPH,MINIZINC}.
\MINIZINC{} models are translated into \FLATZINC{} \cite{FLATZINC} which allows to interface different types of solvers including solvers on finite domain {\CSP}s such as \GECODE{} \cite{GECODE}.
The \MINIZINC{} model for {\UTP} is presented in Table~\ref{table:mzn-contraintes} and based on the decisions variables listed in Table~\ref{table:mzn-variables}.
The model uses some of the global constraints supported in \MINIZINC{} which are dedicated to scheduling problems.

\begin{table*}[ht]
%\resizebox{\textwidth}{!}{%
\centering
{\small
\begin{tabular}{|lll|}
\hline
array[\STUDENT] of var \GROUP: & $\xstudent$ & group assigned to a student \\%groupe attribué à un étudiant\\
array[\CLASS] of var set of \GROUP: & $\xgroup$ & set of groups assigned to a class \\%ensemble de groupes alloués à une classe\\
array[\SESSION] of var set of \ROOM: & \xroom & set of rooms allocated to a session \\%ensemble de salles allouées à une séance\\
array[\SESSION] of var set of \TEACHER: & $\xteacher$ & set of lecturers allocated to a session \\%ensemble d'enseignants alloués à une séance\\
array[\SESSION] of var \SLOT: & $\xslot$ & starting slot of a session \\%créneau de départ attribué à une séance\\
\hline
\end{tabular}
}
\caption{
%Variables de décision (\MINIZINC{})
Decision variables (\MINIZINC{}).
}
\label{table:mzn-variables}
\end{table*}

%\newcolumntype{N}{>{\refstepcounter{rowcntr}\therowcntr}r}
\newcounter{rowcntr}[table]
\renewcommand{\therowcntr}{(\arabic{rowcntr})}
\setcounter{rowcntr}{0}



\begin{table*}[!ht]
%\resizebox{\textwidth}{!}{%

\framebox[\textwidth][c]{%
\small
\begin{tabularx}{\textwidth}{>{\hsize=0.01\hsize\linewidth=\hsize}X>{\hsize=1.89\hsize\linewidth=\hsize}X>{\raggedleft\arraybackslash\hsize=.09\hsize\linewidth=\hsize}X}
%\begin{math}\STUDENT = \funcmzn{array\_union}(\xgroup) \end{math} & 
%partition\_set(\xgroup,\begin{math}\STUDENT\end{math}) & \refstepcounter{rowcntr} \therowcntr \label{mzn:grouppartition}\\
%
%
&$\forallmzn(u,v \inmzn \STUDENT \wmzn u\gqmzn v)$&\\
&\hspace*{2,8em}$(\arraymzn{student\_courses}[u]\neqmzn \arraymzn{student\_courses}[v] \arrowmzn \xstudent[u]\neqmzn \xstudent[v]) $&  \refstepcounter{rowcntr} \therowcntr \label{mzn:studentgrouping}\\
%
%
&$\forallmzn(u \inmzn \STUDENT,p \inmzn \funcmzn{student\_parts}[u])$&\\
&\hspace*{2,8em}$(\existmzn(k \inmzn \arraymzn{part\_classes}[p])(\xstudent[u] \inmzn \xgroup[k]))$& \refstepcounter{rowcntr} \therowcntr \label{mzn:allparts}\\
%forallmzn(u \inmzn \STUDENT, g \inmzn \GROUP)(\xstudent[u] = g \arrowmzn forall(p in \funcmzn{student\_parts}[u])(p = g in  xgroup ))
%
&$\forallmzn(p \inmzn \PART,k1,k2 \inmzn \arraymzn{part\_classes}[p] \wmzn k1\gqmzn k2)$&\\
%
%\hspace{2cm}all_disjoint
&\hspace*{2,8em}$(\xgroup[k1] \intermzn \xgroup[k2]=\{\})$& \refstepcounter{rowcntr} \therowcntr \label{mzn:exclusiveclass}\\
%
%
&$\forallmzn(k1 \inmzn \CLASS, k2 \inmzn \funcmzn{class\_parents}(k1))(\xgroup[k1] \subsetmzn \xgroup[k2])$ &  \refstepcounter{rowcntr} \therowcntr \label{mzn:parent}\\
%
&$\forallmzn(k \inmzn \CLASS)(\arraymzn{maxsize}[k] \geqmzn \summzn(g \inmzn \GROUP)$&\\
&\hspace*{2,8em}$(\funcmzn{bool2int}(g \inmzn \xgroup[k])*\summzn(u \inmzn \STUDENT)(\funcmzn{bool2int}(\xstudent[u]=g)))$ &  \refstepcounter{rowcntr} \therowcntr \label{mzn:classcapacity}\\
% 
%
\hline
%
%
&$\forallmzn(s \inmzn \SESSION)(\xroom[s] \subsetmzn \arraymzn{part\_rooms}[\funcmzn{session\_part}[s]])$& \refstepcounter{rowcntr} \therowcntr \label{mzn:allowedrooms}\\
%
%
&$\forallmzn(s \inmzn \SESSION)(\xteacher[s] \subsetmzn \arraymzn{part\_lecturers}[\funcmzn{session\_part}[s]]) $ &  \refstepcounter{rowcntr} \therowcntr \label{mzn:allowedteachers} \\
%
%
%$\forallmzn(k \inmzn \CLASS)(((\arraymzn{part\_room\_use}[\funcmzn{class\_part}(k)]=\text{none}) \Longleftrightarrow (\xroom[k] = \{\})) $&\\
&$\forallmzn(s \inmzn \SESSION, p \inmzn \PART \wmzn p=\funcmzn{session\_part}[s])($&\\ &\hspace*{2,8em}$(\arraymzn{part\_room\_use}[p]=\text{none} \arrowmzn \xroom[s] = \{\}) $&\\
&\hspace*{1em}$\landmzn (\arraymzn{part\_room\_use}[p]=\text{single} \arrowmzn \funcmzn{card}(\xroom[s]) = 1)  $&\\
&\hspace*{1em}$\landmzn(\arraymzn{part\_room\_use}[p]=\text{multiple} \arrowmzn \funcmzn{card}(\xroom[s]) \leqmzn 1))$
& \refstepcounter{rowcntr} \therowcntr 
\label{mzn:multiroom}\\
%
%
&$\forallmzn(s \inmzn \SESSION)( \funcmzn{card}(\xteacher[s]) = \funcmzn{team}[\funcmzn{session\_part}[s]])$ & \refstepcounter{rowcntr} \therowcntr 
\label{mzn:multiteacher}\\
%
%
&$\forallmzn(p \inmzn \PART,l \inmzn \arraymzn{part\_lecturers}[p])$&\\
&\hspace*{2,8em}$(\summzn{}(s \inmzn \funcmzn{part\_sessions}(p))(\funcmzn{bool2int}(l \inmzn \xteacher[s]) = \arraymzn{service}[l,p])) $& \refstepcounter{rowcntr} \therowcntr 
\label{mzn:partteacherservice}\\
%
%
\hline
%
%
%$\forallmzn(k \inmzn \CLASS)(\xroom[k] \subsetmzn \arraymzn{part\_rooms}[\funcmzn{class\_part}(k))$ & \refstepcounter{rowcntr} \therowcntr  \label{mzn:allowedsrooms}\\
%
%$\forallmzn(s \inmzn \SESSION)(\xteacher[s] \subsetmzn \arraymzn{part\_teachers}[\funcmzn{session\_part}[s]])$  & \refstepcounter{rowcntr} \therowcntr  \label{mzn:allowedsteacher}\\
%
&$\forallmzn(p \inmzn \PART,s \inmzn \funcmzn{part\_sessions}(p))$&\\
&\hspace*{2,8em}$(\funcmzn{week}(\xslot[s]) \inmzn \arraymzn{weeks}[p]$ &\\
&\hspace*{1em}$\landmzn \funcmzn{weekday}(\xslot[s]) \inmzn \arraymzn{weekdays}[p]$&\\
&\hspace*{1em}$\landmzn \funcmzn{dailyslot}(\xslot[s]) \inmzn \arraymzn{dailyslots}[p])$& \refstepcounter{rowcntr} \therowcntr 
\label{mzn:allowedslots}\\
%
%
&$\forallmzn(s \inmzn \SESSION)$&\\
&\hspace*{2,8em}$((\xslot[s] - 1) \divmzn \gconst{nr\_slots\_per\_day} =$&\\
&\hspace*{2,8em}$(\xslot[s] + \funcmzn{length}[s] - 1) \divmzn \gconst{nr\_slots\_per\_day})$& \refstepcounter{rowcntr} \therowcntr 
\label{mzn:nopreemption}\\
%
%
&$ \forallmzn(k \inmzn \CLASS, s1,s2 \inmzn \funcmzn{class\_sessions}[k] \wmzn \funcmzn{rank}(s1) \gqmzn \funcmzn{rank}(s2))$ &\\ 
&\hspace*{2,8em}$(\xslot[s1]+\funcmzn{length}[s] \leqmzn \xslot[s2]) $& \refstepcounter{rowcntr} \therowcntr 
\label{mzn:classsequencing}\\
%
%
%&$\forallmzn(s1 \inmzn \SESSION, r \inmzn \funcmzn{part\_rooms}[\funcmzn{session\_part}[s1]],s2 \inmzn %\funcmzn{room\_sessions}[r] $&\\
%&\hspace*{2,8em}$\wmzn \funcmzn{is\_multi\_rooms}[\funcmzn{session\_part}[s1]]) \landmzn s1 \neqmzn %s2)($&\\
%&\hspace*{3em}$\funcmzn{disjunctive}([\xslot[s1],\xslot[s2]],$&\\
%%
%&\hspace*{3em}$[\funcmzn{bool2int}(r \inmzn \xroom[s1])*\funcmzn{length}[s1],\funcmzn{bool2int}(r %\inmzn \xroom[s2])*\funcmzn{length}[s2]]))$ & \refstepcounter{rowcntr} \therowcntr  %\label{mzn:multiroomscheduling}\\
&$\forallmzn(p \inmzn \PART, s1 \inmzn \funcmzn{part\_sessions}[p], r \inmzn \funcmzn{part\_rooms}[p],s2 \inmzn \funcmzn{room\_sessions}[r] $&\\
&\hspace*{2,8em}$\wmzn \funcmzn{is\_multi\_rooms}[p] \landmzn s1 \neqmzn s2)$&\\
&\hspace*{3em}$(\funcmzn{disjunctive}([\xslot[s1],\xslot[s2]],$&\\
%
&\hspace*{3em}$[\funcmzn{bool2int}(r \inmzn \xroom[s1])*\funcmzn{length}[s1],\funcmzn{bool2int}(r \inmzn \xroom[s2])*\funcmzn{length}[s2]]))$ & \refstepcounter{rowcntr} \therowcntr  \label{mzn:multiroomscheduling}\\
%
%
&$\forallmzn(p \inmzn \PART, s \inmzn \funcmzn{part\_sessions}[p] \wmzn \funcmzn{is\_multi\_rooms}[p])$&\\
&\hspace*{2,8em}$(\summzn(r \inmzn \arraymzn{part\_rooms}[p])(\funcmzn{bool2int}(r \inmzn \xroom[s]) * \arraymzn{capacity}[r])$&\\
&\hspace*{2,8em}$\geqmzn\summzn(g \inmzn \arraymzn{class\_groups}[\funcmzn{session\_class}[s]])(\funcmzn{card}(\arraymzn{group\_students}[g])))$& \refstepcounter{rowcntr} \therowcntr  
\label{mzn:multiroomcapacity}\\
%
%
%$\forallmzn(s \inmzn \SESSION, r \inmzn \funcmzn{session\_rooms}[s])($&\\
%\multicolumn{1}{|c}{$\arraymzn{room\_capacity}[r] \geq sum(g \inmzn %\funcmzn{session\_group}[s])(\funcmzn{card}(\arraymzn{group\_students}[g]))))$} & \refstepcounter{rowcntr} \therowcntr  
%\label{mzn:cumulativeroomcapacity}\\
%
%
&$\forallmzn(p \inmzn \PART, s \inmzn \funcmzn{part\_sessions}[p])(\funcmzn{mandatory\_rooms}[p] \subsetmzn \xroom[s])$& \refstepcounter{rowcntr} \therowcntr 
\label{mzn:mandatoryrooms}\\
%
%
%&$\forallmzn(r \inmzn \ROOM  \wmzn \notmzn(\funcmzn{virtual}[r]))$&\\
%&\hspace*{2,8em}$(\funcmzn{cumulative}([\xslot[s] | s \inmzn \funcmzn{room\_sessions}[r]],$&\\
%&\hspace*{2,8em}$[\funcmzn{bool2int}(r \inmzn \xroom[s])* \funcmzn{length}[s] | s \inmzn \funcmzn{room\_sessions}[r]  ],$&\\
%&\hspace*{3em}$[\summzn(g \inmzn \GROUP)(\funcmzn{bool2int}(g \inmzn \xgroup[\funcmzn{session\_class}[s]])) * \summzn(u \inmzn \STUDENT)($&\\
%&$\funcmzn{bool2int}(g = \xstudent[u]))| s \inmzn \funcmzn{room\_sessions}[r]],\arraymzn{capacity}[r]))$& \refstepcounter{rowcntr} \therowcntr  
%\label{mzn:roomuse}\\
&$\forallmzn(r \inmzn \ROOM  \wmzn \notmzn(\funcmzn{virtual}[r]))($&\\
&\hspace*{2,8em}$\funcmzn{let \{}\funcmzn{set of \SESSION: RS= room\_sessions}[r]\intermzn\funcmzn{single\_room\_sessions;}\}\inmzn$&\\
&\hspace*{2,8em}$(\funcmzn{cumulative}([\xslot[s] | s \inmzn RS],$&\\
&\hspace*{2,8em}$[\funcmzn{bool2int}(r \inmzn \xroom[s])* \funcmzn{length}[s] | s \inmzn RS],$&\\
&\hspace*{3em}$[\summzn(g \inmzn \GROUP)(\funcmzn{bool2int}(g \inmzn \xgroup[\funcmzn{session\_class}[s]])) * \summzn(u \inmzn \STUDENT)($&\\
&$\funcmzn{bool2int}(g = \xstudent[u]))| s \inmzn RS],\arraymzn{capacity}[r]))$& \refstepcounter{rowcntr} \therowcntr  
\label{mzn:roomuse}\\
%
%
%\forallmzn(t \inmzn \TEACHER)(\funcmzn{cumulative}( & \\
%\multicolumn{1}{|c}{$[\xslot[s] | s \inmzn \funcmzn{teacher\_sessions}(t)],[\funcmzn{bool2int}(t \subsetmzn \xteacher[s])* \funcmzn{session\_length}[s] | s \inmzn \funcmzn{teacher\_sessions}(t)],$}&\\
%\multicolumn{1}{|c}{$[1| s \inmzn \funcmzn{teacher\_sessions}(t)],teacher\_max\_use[t])))$}& \refstepcounter{rowcntr} \therowcntr  
%\label{mzn:teacheruse}\\
%
%
%$\forallmzn(g \inmzn \GROUP)(\funcmzn{cumulative}($&\\
%\multicolumn{1}{|c}{$[\xslot[s] | s \inmzn \funcmzn{group\_sessions}(g)],[\funcmzn{session\_length}[s] | s \inmzn \funcmzn{group\_sessions}(g)],$}&\\
%\multicolumn{1}{|c}{$[1| s \inmzn \funcmzn{group\_sessions}(g)],group\_max\_use[g])))$}& \refstepcounter{rowcntr} \therowcntr  
%\label{mzn:groupuse}\\
%
%
\hline
%
%
&$\FORBIDDENPERIOD((r,S'),h1,h2) = \forallmzn(i \inmzn S')($&\\
&\hspace*{2,8em}$ r \inmzn \xroom[i] \arrowmzn (\xslot[i]+\funcmzn{length}[i] \geqmzn h_1 \lormzn  \xslot[i]\lqmzn h_2)) $& \refstepcounter{rowcntr} \therowcntr 
\label{mzn:forbiddenperiod}\\
%
%
&$\SAMEWEEKDAY((r,S')) = \forallmzn(i,j \inmzn S' \wmzn i\gqmzn j)($&\\
&\hspace*{2,8em}$ (r\inmzn \xroom[i] \intermzn \xroom[j]) \arrowmzn $&\\
&\hspace*{2,8em}$(\xslot[i] \divmzn \gconst{nr\_weekly\_slots}=\xslot[j] \divmzn  \gconst{nr\_weekly\_slots}))$ & \refstepcounter{rowcntr} \therowcntr 
\label{mzn:sameweekday}\\
%
%
&$\SAMEROOMS((r,S')) = \forallmzn(i,j \inmzn S' \wmzn i\gqmzn j)(($&\\
%
&\hspace*{2,8em}$r\inmzn \xroom[i] \intermzn \xroom[j]) \arrowmzn \xroom[i] = \xroom[j])$ & \refstepcounter{rowcntr} \therowcntr 
\label{mzn:samerooms}\\
%
%
&$\SEQUENCED((r1,S1),(r2,S2)) = \forallmzn(i \inmzn S1,j \inmzn S2)($&\\
&\hspace*{3em}$(r1\inmzn \xroom[i] \landmzn r2\inmzn \xroom[j]) \arrowmzn\xslot[i]\texttt{+}\funcmzn{length}[i] \geqmzn \xslot[j])$ & \refstepcounter{rowcntr} \therowcntr 
\label{mzn:sequenced}\\
%
%
&$\NOOVERLAP((r,S'))=$&\\
&\hspace*{2,8em}$\funcmzn{disjunctive}([\xslot[i]|i \inmzn S'],[\funcmzn{length}[i]*\funcmzn{bool2int}(r \inmzn \xroom[i])|i \inmzn S'])$ & \refstepcounter{rowcntr} \therowcntr 
\label{mzn:nooverlap}\\

\end{tabularx}%
}%
%}
\caption{
%Contraintes et prédicats du modèle \MINIZINC{}
Constraints and predicates of the \MINIZINC{} model.
}
\label{table:mzn-contraintes}
\end{table*}


%------------------------------------------------------------
%------------------------------------------------------------
%\subsection{Student Sectioning}
%\label{sec:model-sectioning}

%Les contraintes de sectionnement répartissent
%les étudiants dans les groupes et assigne chacun de ces groupes à différentes classes 
%conformément aux règles de sectionnement et aux seuils d'effectifs.
%%La contrainte~\ref{mzn:grouppartition} partitionne les étudiants en groupes.
%La contrainte \ref{mzn:studentgrouping} n'autorise le regroupement
%d'étudiants que s'ils sont inscrits aux mêmes cours.
%\ref{mzn:allparts} impose que tout étudiant, assimilé à son groupe, assiste à toute partie de cours dans lequel il est inscrit.
%\ref{mzn:exclusiveclass} assure que les classes d'une partie de cours n'ont aucun groupe en commun.
%\ref{mzn:parent} implémente la relation de parenté entre classes.
%Enfin, \ref{mzn:classcapacity} vérifie que l'effectif cumulé des groupes attribués à une classe ne dépasse pas le seuil autorisé. 
%A noter que cette contrainte utilise des variables auxiliaires pseudo-booléennes.
%
Sectioning constraints partition students into groups and assign each group to a class according to sectioning rules and class size thresholds.
Constraint~\ref{mzn:studentgrouping} allows students to be part of the same group only if they are registered to the same courses.
\ref{mzn:allparts} imposes that every student attends all the part of the courses to which he is registered.
\ref{mzn:exclusiveclass} ensures that classes from the same part do not have any common group.
\ref{mzn:parent} implements the parent-child relation between classes.
Lastly, \ref{mzn:classcapacity} checks that the groups fit in the class they have been assigned to.

%------------------------------------------------------------
%------------------------------------------------------------
%\subsubsection{Resource Distribution}
%\label{sec:model-distribution}

%La distribution des ressources s'appuie sur des contraintes de domaine, de cardinalité et de sommes.
%Les contraintes~\ref{mzn:allowedrooms} et \ref{mzn:allowedteachers} définissent les salles et enseignants allouables à chaque séance.
%\ref{mzn:multiroom} contraint le nombre de salles allouées à une séance selon que sa partie de cours est sans salles, mono-salle, ou multi-salles.
%\ref{mzn:multiteacher} attribue le nombre attendu d'enseignants à chaque séance
%et \ref{mzn:partteacherservice} vérifie que chaque enseignant dispense le volume de séances requis par partie de cours où il est pré-positionné.
%
Resource distribution relies on domain, cardinality and sum constraints.
Constraints~\ref{mzn:allowedrooms} and \ref{mzn:allowedteachers} define available rooms and lecturers for each session.
\ref{mzn:multiroom} forces the number of rooms allocated to a session according to the specific requirements of the course part (i.e., no room, single-room or multi-room).
\ref{mzn:multiteacher} allocates the required number of lecturers to a session and \ref{mzn:partteacherservice} checks that every lecturer has the right number of sessions in a part.

%------------------------------------------------------------
%------------------------------------------------------------
%\subsubsection{Session Scheduling and Resource Allocation}
%\label{sec:model-scheduling}

%La programmation des séances et l'allocation des ressources met en jeu des contraintes de positionnement, de séquencement, de non-chevauchement et de capacité. 
%%Constraint~\ref{mzn:allowedslots} defines the allowed slots per session
%%and 
%La contrainte~\ref{mzn:allowedslots} définit les créneaux autorisés pour chaque séance.
%\ref{mzn:nopreemption} interdit qu'une séance soit à cheval sur 2 journées. 
%\ref{mzn:classsequencing} séquence les séances d'une classe selon leur rangs.
%Les contraintes \ref{mzn:multiroomscheduling} et \ref{mzn:multiroomcapacity} modélisent les séances multi-salles et l'accès exclusif à leurs ressources.
%\ref{mzn:multiroomscheduling} impose qu'une ressource allouée à une séance multi-salles soit disjonctive le temps de son utilisation.
%\ref{mzn:multiroomcapacity} assure que le nombre d'étudiants attendus n'excède pas la capacité cumulée des salles allouées.
%À noter que cette contrainte est purement quantitative et autorise toute répartition d'étudiants dans les salles indépendamment de la structure de groupes.
%\ref{mzn:mandatoryrooms} modélise les salles à allouer obligatoirement à toute séance d'une partie de cours.
%\ref{mzn:roomuse} modélise la contrainte de capacité cumulative qui s'applique par défaut à toute salle allouée hors séances multi-salles.
%%\ref{mzn:groupuse}
%
Session scheduling and resource allocation involves positioning, sequencing, non-overlaping and capacity constraints.
Constraint~\ref{mzn:allowedslots} defines the allowed slots for each session.
\ref{mzn:nopreemption} forbids a session to be on two days.
\ref{mzn:classsequencing} sequences the sessions of a class according to their rank.
Constraints~\ref{mzn:multiroomscheduling} and \ref{mzn:multiroomcapacity} model multi-room sessions and the exclusive access to their rooms.
\ref{mzn:multiroomscheduling} makes disjunctive any resource that is allocated to a multi-room session while it is hosting the session.
\ref{mzn:multiroomcapacity} ensures that the number of students attending a multi-room session do not exceed the cumulated capacity of the allocated rooms.
%Note that this constraint is purely quantitative and allows all repartition of students in the allocated rooms without taking into account the group structure.
\ref{mzn:mandatoryrooms} models the mandatory rooms to be allocated.
\ref{mzn:roomuse} models the default cumulative capacity constraint controlling the allocation of non-virtual rooms to single-room sessions.
This constraint uses the \texttt{cumulative} global constraint of \MINIZINC{} (see \cite{2002_beldiceanu_CP} for the \GECODE{} implementation)
which \MINIZINC{} also reuses to rewrite the global \texttt{disjunctive} constraint. 

%------------------------------------------------------------
%------------------------------------------------------------

%\subsubsection{{\UTP} Predicates}
%\label{sec:model-predicates}

%%%%{\ADJACENTROOMS}
%%%%{\ATMOSTDAILY}
%%%%{\ATMOSTWEEKLY}
%%%%{\TRAVEL}
%{\FORBIDDENPERIOD}
%{\NOOVERLAP}
%%%%{\SAMEDAILYSLOT}
%%%%{\SAMEDAY}
%{\SAMEROOMS}
%%%%{\SAMESLOT}
%%%%{\SAMESTUDENTS}
%%%%{\SAMETEACHERS}
%{\SAMEWEEKDAY}
%%%%{\SAMEWEEKLYSLOT}
%%%%{\SAMEWEEK}
%{\SEQUENCED}
%%%%{\TEACHERDISTRIBUTION}
%%%%{\WEEKLY}

%%Due to lack of space, we just present a few {\UTP} constraint predicates, namely,
%La table~\ref{table:mzn-contraintes} présente les variantes de quelques prédicats \UTP{} dans le cas où les entités ciblées sont des salles.
%%, respectivement,
%%\texttt{\FORBIDDENPERIOD} (\ref{mzn:forbiddenperiod}),
%%\texttt{\SAMEWEEKDAY} (\ref{mzn:sameweekday}), 
%%\texttt{\SAMEROOMS} (\ref{mzn:samerooms}),
%%\texttt{\NOOVERLAP} (\ref{mzn:nooverlap}) 
%%et \texttt{\SEQUENCED} (~\ref{mzn:sequenced}).
%%Note that \texttt{\FORBIDDENPERIOD} accepts start and end point parameters.
%\ref{mzn:forbiddenperiod} implémente le prédicat \texttt{\FORBIDDENPERIOD} qui prend en paramètres les 2 créneaux modélisant la période interdite. 
%\ref{mzn:sameweekday}, \ref{mzn:samerooms} et \ref{mzn:sequenced} modélisent de manière directe les prédicats \texttt{\SAMEWEEKDAY}, \texttt{\SAMEROOMS} et \texttt{\SEQUENCED}.
%\ref{mzn:nooverlap} implémente le prédicat \texttt{\NOOVERLAP} en s'appuyant sur la contrainte globale \texttt{disjunctive}. %de \MINIZINC{}.
%
Table~\ref{table:mzn-contraintes} also presents some \UTP{} predicates when the targeted resources are rooms.
\ref{mzn:forbiddenperiod} implements the \texttt{\FORBIDDENPERIOD} predicate that takes the start and end time slots of the period as parameters.
\ref{mzn:sameweekday}, \ref{mzn:samerooms} and \ref{mzn:sequenced} model \texttt{\SAMEWEEKDAY}, \texttt{\SAMEROOMS} and \texttt{\SEQUENCED} predicates, respectively.
\ref{mzn:nooverlap} implements the \texttt{\NOOVERLAP} predicate that relies on the \texttt{disjunctive} global constraint.

% Note that entailed constraints may be enforced. 
% For instance, a room $r$ is necessarily disjunctive if a non-overlapping constraint is enforced on its set of compatible sessions, %\footnote{
%Il faut noter que des contraintes peuvent être renforcés. 
%Par exemple, une salle $r$ est nécessairement disjonctive si une contrainte de non-chevauchement est appliquée à son ensemble de sessions possibles, c'est-à-dire si l'instance inclut la contrainte ${no\_overlap}$.
% i.e., if the instance includes constraint ${no\_overlap}((r,\map{\ROOM}{\SESSION}{r}))$.
% If so, the default cumulative capacity constraint may be safely replaced with  Constraint~\ref{mzn:disjunctiveroomcapacity} for such resources where
% $\disjunctiverooms\subseteq{\ROOM}$         denotes the set of disjunctive rooms.
%Dans ce cas, la contrainte cumulative par défaut peut être remplacée en toute sécurité par la contrainte~\ref{mzn:disjunctiveroomcapacity} pour les ressources concernés.

%------------------------------------------------------------
%------------------------------------------------------------
\subsection{{\CHR} model}
\label{sec:cp-chr}

{\CHR} (for {\it Constraint Handling Rules}) \cite{1994_fruhwirth_Chap,1998_fruhwirth_JLP,2009_fruhwirth_Book,2011_fruhwirth_Book} are a committed-choice language consisting of multiple-heads guarded rules that replace constraints by more  simple constraints until they are solved.
{\CHR} are a special-purpose language concerned with defining declarative constraints in the sense of {\it Constraint logic programming} \cite{1991_vanHentenryck_KER,1994_jaffar_JLP}.
\CHR\ are a language extension that allows to introduce {\it user-defined} constraints, i.e. first-order predicates, into a given host language as {\PROLOG}, {\LISP}, {\JAVA}, or {\C}/{\CPP}.
\CHR{} have been extended to {\CHR}\ensuremath{^{\vee}} \cite{1998_abdennadher_Chap} that introduces the \emph{don't know} nondeterminism in \CHR{} \cite{2013_betz_ACMTCL}.
This nondeterminism is freely offered when the host language is {\PROLOG} and allows to specify easily problems from the {\NP} complexity class.

To model and solve {\UTP} instances with the {\CHR} language, we use the {\CHRPP} solver~\cite{2019_barichard_ICLP} (for Constraint Handling Rules in {\CPP}), which is an efficient integration of {\CHR} in the programming language \texttt{C++}.\\

The full model for \CHRPP\ is too long to be detailed here\footnote{The interested reader can download the sources of the model \cite{uspSite}.}. We give in Table~\ref{table:contrainte-tab-chr} the list of constraints taken into account by the solver. The decision variables to be instantiated are given in Table~\ref{table:chr-variables}. They are similar to those of the \MINIZINC\ model, only the end-of-session variables are added.\\

\begin{table*}[!ht]
\framebox[\linewidth][c]{%
\small
\begin{tabular}{ll}
$\forall s \in \SESSION: \xroom[s] \subseteq \ROOM$ & set of rooms allocated to a session\\
%array[\SESSION] of var set of \ROOM: $\xroom$ & set of rooms allocated to a session\\
$\forall s \in \SESSION: \xteacher[s] \subseteq \TEACHER$ & set of lecturers allocated to a session\\
%array[\SESSION] of var set of \TEACHER: $\xteacher$ & set of teachers allocated to a session\\
$\forall s \in \SESSION: \xslotstart[s] \in \SLOT$ & starting slot allocated to a session\\
%array[\SESSION] of int \SLOT: $\xslotstart$ & starting slot allocated to a session\\
$\forall s \in \SESSION: \xslotend[s] \in \SLOT$ & ending slot allocated to a session\\
%array[\SESSION] of int \SLOT: $\xslotend$ & ending slot allocated to a session\\
\end{tabular}%
}
\caption{Decision variables (\CHR).}
\label{table:chr-variables}
\end{table*}

\newcounter{rowcntrchr}[table]
\setcounter{rowcntrchr}{0}
\renewcommand{\therowcntrchr}{(\arabic{rowcntrchr})}

\begin{table*}[!ht]
\framebox[\textwidth][c]{%
\small
\begin{tabularx}{\textwidth}{>{\hsize=0.01\hsize\linewidth=\hsize}X>{\hsize=1.89\hsize\linewidth=\hsize}X>{\raggedleft\arraybackslash\hsize=.09\hsize\linewidth=\hsize}X}
%% CONTRAINTES STATIQUES
\multicolumn{3}{l}{Integrity constraint~:}\\
%
% La date de fin et la date de départ sont liées
& $\forall s \in \SESSION\ : \arraychr{\xslotend}[s] = \arraychr{\xslotstart}[s] + \funcchr{length}(s)$ %
& \refstepcounter{rowcntrchr} \therowcntrchr \label{ctrchr:startend} \\
%&&\\
\multicolumn{3}{l}{Static constraints (instance input filtering))~:}\\
%
% Restriction de domaine au parsing (allowed_xxx)
& $\forall s \in \SESSION\ : \arraychr{\xroom}[s] \subseteq \arraychr{part\_rooms}[\funcchr{session\_part(s)}]$ %
& \refstepcounter{rowcntrchr} \therowcntrchr \label{ctrchr:allowedroom} \\
%
& $\forall s \in \SESSION\ : \arraychr{\xteacher}[s] \subseteq \arraychr{part\_lecturers}[\funcchr{session\_part(s)}]$ %
& \refstepcounter{rowcntrchr} \therowcntrchr \label{ctrchr:allowedteacher} \\
%
& $\forall p \in \PART, \forall s \in \funcchr{part\_sessions}(p) :$ &\\
& \hspace*{3em}$\big(\funcchr{week}(\arraychr{\xslotstart}[s]) \in  \arraychr{weeks}[p]\big )$ &\\%
& \hspace*{3em}$\wedge \big(\funcchr{weekday}(\arraychr{\xslotstart}[s]) \in  \arraychr{days}[p]\big )$ &\\ %
& \hspace*{3em}$\wedge \big(\funcchr{dailyslot}(\arraychr{\xslotstart}[s]) \in  \arraychr{dailyslots}[p]\big )$ %
& \refstepcounter{rowcntrchr} \therowcntrchr \label{ctrchr:allowedslot} \\
%
% Une session commence et finit le même jour
& $\forall s \in \SESSION\ : \arraychr{\xslotstart}[s] / nr\_slots\_per\_day = \arraychr{\xslotend}[s] / nr\_slots\_per\_day$ %
& \refstepcounter{rowcntrchr} \therowcntrchr \label{ctrchr:startendsameday} \\
%
% Cardinalité nombre de teachers
& $\forall s \in \SESSION\ : \funcchr{card}(\arraychr{\xteacher}[s]) = \arraychr{team}[session\_part[s]]$ %
& \refstepcounter{rowcntrchr} \therowcntrchr \label{ctrchr:cardteacher} \\
%
% Cardinalité nombre de salles
& $\forall k \in \CLASS, \forall s \in \arraychr{class\_sessions}[k] :$&\\
& \hspace*{3em}If $\big( \arraychr{part\_room\_use}[\funcchr{class\_part}(k)] = \arraychr{none} \big )$ then $ \funcchr{card}(\arraychr{\xroom}[s]) = 0$ &\\%
& \hspace*{3em}If $\big( \arraychr{part\_room\_use}[\funcchr{class\_part}(k)] = \arraychr{single} \big )$ then $ \funcchr{card}(\arraychr{\xroom}[s]) = 1$ &\\%
& \hspace*{3em}If $\big( \arraychr{part\_room\_use}[\funcchr{class\_part}(k)] = \arraychr{multiple} \big )$ then $ \funcchr{card}(\arraychr{\xroom}[s]) \geq 1$ %
& \refstepcounter{rowcntrchr} \therowcntrchr \label{ctrchr:cardroom} \\
%
% Ranking des sessions
& $\forall k \in \CLASS, \forall s,s' \in \arraychr{class\_sessions}[k], s.t.\; \funcchr{rank}(s) < \funcchr{rank}(s') : \ctchr{before}(s,s')$ %
& \refstepcounter{rowcntrchr} \therowcntrchr \label{ctrchr:ranking} \\
%
% Un groupe dans deux classes
& $\forall k_1,k_2 \in \CLASS, s.t.\; \exists g_1 \in \arraychr{class\_groups}[k_1], \exists g_2 \in \arraychr{class\_groups}[k_2], \mbox{ avec } g_1 = g_2 :$ &\\%
& \hspace*{3em}$\forall s_1 \in \funcchr{class\_sessions}(k_1), s_2 \in \funcchr{class\_sessions}(k_2) : \ctchr{disjunct}(s_1,s_2)$%
& \refstepcounter{rowcntrchr} \therowcntrchr \label{ctrchr:disjunctgroups} \\
%
%&&\\
\multicolumn{3}{l}{Static predicates~:}\\
%
% Forbidden slots
& $forbidden\_period((e,S'),h,h') = \forall i \in S' : (\arraychr{\xslotstart}[i] + \funcchr{length}(i) \leq h) \vee (\arraychr{\xslotstart}[i] > h')$ %
& \refstepcounter{rowcntrchr} \therowcntrchr \label{ctrchr:forbiddenslot} \\
%
& $sequenced((e_1,S_1),(e_2,S_2)) = \forall i_1 \in S_1, \forall i_2 \in S_2 : \ctchr{before}(i_1,i_2)$ %
& \refstepcounter{rowcntrchr} \therowcntrchr \label{ctrchr:sequenced} \\
%
% Same rooms
& $same\_rooms((e,S')) = \forall s_1,s_2 \in S', s.t.\; s_1 < s_2 : \arraychr{\xroom}[s_1] \sim \arraychr{\xroom}[s_2]$%
& \refstepcounter{rowcntrchr} \therowcntrchr \label{ctrchr:samerooms} \\
%
%&&\\
\multicolumn{3}{l}{Dynamic constraints~:}\\
% Un teacher fait son bon nombre de sessions
& $\forall p \in \PART, \, \forall l \in \arraychr{part\_lecturers}[p] : \big|\big| \{ x \;|\; x \in \funcchr{part\_sessions}(p), l \in \arraychr{\xteacher}[x] \} \big | \big | = \arraychr{service}[l,p] $%
& \refstepcounter{rowcntrchr} \therowcntrchr \label{ctrchr:teacherservice} \\
%
% Capacité des salles à ne pas dépasser
& $\forall s \in \SESSION, \forall r \in \arraychr{session\_rooms}(s) :$ &\\%
& \hspace*{3em}$\sum \{ \funcchr{group\_students}[g] \;|\; g \in \funcchr{session\_room\_group}(s,r), r \in \arraychr{\xroom}[s] \} \leq \arraychr{capacity}[r] $%
& \refstepcounter{rowcntrchr} \therowcntrchr \label{ctrchr:roomcapacity} \\
%
% Salle mandatory
& $\forall s \in \SESSION, s.t.\; has\_mandatory\_room(s) :  \arraychr{session\_mandatory}[s] \subseteq \arraychr{\xroom}[s] $%
& \refstepcounter{rowcntrchr} \therowcntrchr \label{ctrchr:roommandatory} \\
%
%&&\\
\multicolumn{3}{l}{Dynamic predicate~:}\\
% Same weekday
& $same\_weekday((e,S')) = $&\\
& \hspace*{3em}$\forall s_1,s_2 \in S', s.t.\; s_1 < s_2 : \arraychr{\xslotstart}[s_1] / nr\_weekly\_slots = \arraychr{\xslotstart}[s_2] / nr\_weekly\_slots $%
& \refstepcounter{rowcntrchr} \therowcntrchr \label{ctrchr:sameweekday} \\
%
%&&\\
\multicolumn{3}{l}{Introspective constraints~:}\\
% Un teacher dans deux classes
& $\forall k_1,k_2 \in \CLASS, \forall s_1 \in \arraychr{class\_sessions}[k_1], \forall s_2 \in \arraychr{class\_sessions}[k_2], \mbox{ s.t. } s_1 \ne s_2 :$ &\\%
& \hspace*{3em}$ \arraychr{\xteacher}[s_1] \cap \arraychr{\xteacher}[s_2] \ne \emptyset \chrprop \ctchr{disjunct}(s_1,s_2)$%
& \refstepcounter{rowcntrchr} \therowcntrchr \label{ctrchr:disjunctteacher} \\
%
% Une salle dans deux groupes
& $\forall k_1,k_2 \in \CLASS, \forall s_1 \in \arraychr{class\_sessions}[k_1], \forall s_2 \in \arraychr{class\_sessions}[k_2] \mbox{ s.t. } s_1 \ne s_2 :$ &\\%
& \hspace*{3em}$ \arraychr{\xroom}[s_1] \cap \arraychr{\xroom}[s_2] \ne \emptyset \chrprop \ctchr{disjunct}(s_1,s_2)$%
& \refstepcounter{rowcntrchr} \therowcntrchr \label{ctrchr:disjunctroom} \\
%
\end{tabularx}%
}%
\caption{Constraints and predicates of the \CHR\ model.}
\label{table:contrainte-tab-chr}
\end{table*}

To simplify its implementation, the model is partly non-cumulative and some resources such as lecturers cannot be shared. It also considers that the sectioning and allocation of students to groups is done beforehand. Thus, computing a solution amounts to finding a consistent resource allocation while placing the schedules for all sessions.

Several constraints can be set at the instance analysis stage. This is the case for constraints \ref{ctrchr:startend} to \ref{ctrchr:disjunctgroups} of Table~\ref{table:contrainte-tab-chr}. Constraints \ref{ctrchr:allowedroom}, \ref{ctrchr:allowedteacher} and \ref{ctrchr:allowedslot} filter the domains by removing the rooms, lecturers or time slots which are impossible by construction of the instance. Constraint~\ref{ctrchr:startendsameday} ensures that a session starts and ends on the same day by removing from the domain values that contradict it.

Other constraints are set and managed by rules which monitor modifications to the domains of variables. This is the case for Constraint~\ref{ctrchr:startend} which ensures the integrity of the start and end of session variables. The same is true for \ref{ctrchr:cardteacher} which ensures that the number of lecturers teaching a session is valid and \ref{ctrchr:cardroom} which checks that the number of rooms allocated to a session corresponds to what is required in the instance.

We give as an example the \CHRPP\ rule which checks the integrity of the variables of beginning and end of session. The rule uses a \texttt{plus} propagator to ensure consistency of the constraint. This is triggered as soon as a domain of a variable is updated:

\begin{lstlisting}[style=custom, language=c++]
session_slot(_, S_Start, S_End, S_Length)
  =>> CP::Int::plus(S_Start, (*S_Length)-1, S_End);;
\end{lstlisting}

We use \CHRPP\ which allows us to manipulate values associated with logical variables and to wake up the corresponding rules as soon as a modification of the value occurs. This mechanism combined with the forward chaining of \CHR\ allows us to implement an efficient rule wake-up and domain propagation mechanism in the manner of a \CSP{} solver.

Constraints~\ref{ctrchr:ranking} and \ref{ctrchr:disjunctgroups} add new \CHR\ constraints to the model. Indeed, constraints \ctchr{before} and \ctchr{disjunct} are constraints ensuring the precedence and non-overlapping of two sessions. They are accompanied by rules verifying the coherence of the disjunctive graph created implicitly by the addition of all these constraints. The static predicates correspond to those read from the instance. They are processed and some new constraints (filtering constraints, \CHR{} constraints or unification of variables) are added.

Dynamic constraints ranging from \ref{ctrchr:teacherservice} to \ref{ctrchr:disjunctroom} are only triggered under certain conditions. \CHR\ guarded rules are used for this purpose. \ref{ctrchr:teacherservice} checks that a lecturer teaches the expected number of sessions in each course part. \ref{ctrchr:roomcapacity} ensures that the capacity of the rooms is respected and \ref{ctrchr:roommandatory} verifies that the rooms marked as mandatory are indeed found in the solution. Predicate~\ref{ctrchr:sameweekday} ensures that sessions subject to the same constraint $same\_weekday$ are set on the same day of the week.

Constraints \ref{ctrchr:disjunctteacher} and \ref{ctrchr:disjunctroom} add constraints when certain conditions are verified. Thus, \ref{ctrchr:disjunctteacher} adds a \ctchr{disjunct} between two sessions when the same lecturer participates. \ref{ctrchr:disjunctroom} adds a constraint between two sessions if they take place in the same room. These constraints enrich the disjunctive graph representing the sequencing of all the sessions.

%\begin{lstlisting}[style=custom, language=c++]
%session_slot(_, Start_slot, End_Slot, S_Length)
%    =>> CP::Int::plus(Start_Slot, (*S_Length)-1, End_Slot);;
%\end{lstlisting}
%
%
%\begin{lstlisting}[style=custom, language=c++]
%same_days(Modulo, Session_id),
%session_slot(Session_id, Start_slot, End_slot, S_Length)
%    =>> CP::Int::modulo_boundConsistency(Start_slot, slotsPerWeek, Modulo);;
%\end{lstlisting}

It should be noted that the \CHR\ model performs domain filtering but also analyses the disjunctive graph in order to eliminate non-solutions. The edges of the disjunctive graph are oriented as the resolution progresses and the decision variables are instantiated. 

